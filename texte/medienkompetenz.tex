
\section{Lesen und Medienkompetenz}

  Laut Postman haben Kommunikationsmöglichkeiten Kindheit erst entstehen und
  sind gleichzeitig in  der Lage sie wieder verschwinden zu lassen. Mit Harold
  Innis teilt er die Auffassung,  dass Veränderungen innerhalb der
  Kommunikationstechnik drei Auswirkungen haben: die Veränderung der
  Interessensstruktur (worüber wird nachgedacht?), den Charakter der Symbole
  (womit wird gedacht?) und das Wesen der Gemeinschaft (wo entwickeln sich die
  Gedanken?). \parencite[34]{Postman1985} Das \enquote{gesellschaftliche
  Kunstprodukt} Kindheit sei eine Erfindung der Renaissance, erst die
  Druckerpresse habe Kindheit möglich gemacht. Wenn er vom \enquote{Verschwinden
  der Kindheit} spricht, macht er die, durch die neuen elektronischen Medien
  vermittelten, Inhalte, die die kindliche Phantasie nicht mehr anregen,
  verantwortlich: Bilder und andere Darstellungsformen im Fernsehen, also
  vorrangig visuelle Medien, bieten der eigenen Vorstellungskraft, im Gegensatz
  zum Text in Büchern, wenig Entfaltungsmöglichkeiten. Gleichzeitig laufen
  Reflexions- wie Kritikfähigkeit Gefahr zu verkümmern, da nur elementare
  Fähigkeiten gebraucht würden. Außerdem kritisiert er, dass zunehmend für
  Erwachsene typische Wünsche transportiert werden, die die Neugier und
  Andersartigkeit des Kindseins gefährden, auch weil sie keine Geheimnisse mehr
  hüten. \parencite[93\psq]{Postman1985} Erfahrungsräume, die nur Literatur
  bietet, können verloren gehen.

  Selbstverständlich sollen hier die Entwicklungen im letzten
  Vierteljahrhundert, die das allumfassende Kommunikationsmedium Internet
  betreffen, nicht unberücksichtigt bleiben, welches seinen NutzerInnen auch
  spezifische Kompetenzen abverlangt. Trotzdem sind wir der Meinung, dass
  Postman hier einen Nerv getroffen hat und gerade die Lesekompetenz von
  entscheidender Wichtigkeit ist, wenn es darum geht, die Vorstellungskraft zu
  fördern oder ein kritisches Verständnis zu entwickeln.

  Lesesozialisation kann als Ausschnitt der Mediensozialisation gesehen werden:
  durch Lesen wird nämlich nicht nur die Fähigkeit zur Dekodierung von
  schriftlichen Texten gefördert, sondern es werden auch
  Kommunikationsinteressen und kulturelle Haltungen erworben.\footnote{Der
  Literatur wurde nicht immer eine positive Funktion zugeschrieben, gerade der
  Unterhaltungsliteratur warf man vor, Kinder von sinnvollen Tätigkeiten
  abzuhalten. Erst durch die Konkurrenz der elektronischen Medien schien der
  Umgang mit Texten förderungswürdig. } \parencite[220\psqq]{Weinkauff2010}