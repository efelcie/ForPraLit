\subsection{Forschung zu Geschlecht}


 Gender ist ein englischer Ausdruck, der das \emph{soziale} Geschlecht bezeichnet. In diesem Sinne ist es ein \hyphenquote{french}{fait social} im klassischen Sinne.\footnote{Leider geht das \emph{fait}, also \emph{gemacht} bei der Übersetzung verloren und im Englischen und Deutschen wird noch immer über konstruiert oder nicht gestritten. \parencite[152--161]{Latour2010}} \parencite[Kap.\,1]{Durkheim1970} Doch in der Genderforschung ist es weniger klar: Sie ist ein heterogenes Feld mit, wie in der Soziologie üblich, vielen, theoretisch gesehen, inkompatiblen Standpunkten. \parencite[67]{Nissen1998} 

Schon die Einteilung der Standpunkte und wie man mit ihnen umgehen soll, stellt ein Problem dar. \inparencite[86]{Nissen1998} teilt die Ansätze in die \enquote{\enquote{drei Räume} des Feminismus} ein: Gleichheit, Differenz und Dekonstruktion. Sie meint, man solle sich in den drei Räumen \enquote{einrichten}. Damit meint sie, man solle sich einem Mix der Theorien bedienen um möglichst viele Aspekte des Problems abzudecken. \inparencite[216]{Gildemeister2000} teilt die Positionen grob in \enquote{Geschlecht als \emph{Strukturkategorie} und Geschlecht als \emph{soziale Konstruktion}} ein. Jedoch ist eine Verbindung der Positionen auch für sie wichtig.
\blockcquote[223]{Gildemeister2000}{Umso wichtiger wird es, solche Verfahren zu entwickeln, in denen die interaktive Herstellung von Geschlecht verbunden wird mit der Analyse von Geschlechterordnungen in modernen Gesellschaften. Bislang steht weitgehend aus, Struktur- und Prozessanalysen miteinander zu verbinden oder, wie es auch heißt: Analyse sozialer Ungleichheit mit dem Fokus auf \enquote{soziale Konstruktion}.}
	
Geschlecht als Strukturkategorie heißt, Geschlecht ist ein messbares Merkmal der Gesellschaft wie Schicht oder Klasse. Der Ansatz verwendet Geschlecht als Analyse-Einheit, wodurch Aussagen über Ungleichheit oder Gleichheit möglich werden.  Die zwei Räume, Gleichheit und Differenz, von \citeauthor{Nissen1998}, fassen Geschlecht als Strukturkategorie auf. Jedoch haben beide Ansätze unterschiedliche Grundannahmen und unterschiedliche Ziele. 
Die Differenzpositionen gehen davon aus, dass es einen Unterschied zwischen Frauen und Männern gibt. Das rechtfertigt jedoch nicht, dass der Mann über der Frau steht. Ziel dieser Ansätze ist eine \emph{Aufwertung} der Weiblichkeit.
Der Gleichheitsansatz geht davon aus, dass von Geburt an alle Menschen gleich sind. 
Die, als Strukturkategorie messbaren, Unterschiede zwischen den Geschlechtern sind Konstruktionen, in die wir Menschen hineingepresst werden. Die Konstruktionen erzeugen eine (reale) Unterscheidung zwischen Frau und Mann, die dem Mann hilft, seine Stellung in der sozialen Hierarchie zu festigen. \parencite[181]{Hertz2007}
\textcquote[181]{Hertz2007}{Und die Männer, die sich heute an den Forderungen der Frau stören, berufen sich auf die \emph{natürliche} Unterlegenheit der Frau.}
Der Gleichheitsansatz verwendet Geschlecht als Strukturkategorie, jedoch sieht er Geschlecht auch als soziale Konstruktion.

Geschlecht als soziale Konstruktion ist eine problematische Einteilung, weil der Begriff \emph{Konstruktion} je nach erkenntnistheoretischer Position etwas anderes bedeutet. \parencite[219]{Gildemeister2000} Allen gmeinsam ist allerdings die Betonung des Werdens von Geschlecht. Um klar zu machen, dass man für das \emph{Werden} soziologische Erklärungen sucht, ist es wichtig sich von naturwissenschaftlichen zu Distanzieren. Am deutlichsten machen dies \inparencite[126]{West1987}. Sie unterscheiden zwischen dem naturwissenschaftlichen Geschlecht (sex), der Kategorie Geschlecht (sex category) und dem von der Geschlechts-Kategorie abhängigen Verhalten (gender). Gender ist ein Unterschied den man macht. Anders als bei Geschlecht als Strukturkategorie wird sich nicht auf die Beziehungen von Frauen zu Männern konzentriert, sondern wie und warum wir in Frauen und Männer denken. Gender ist nicht Folge von Struktur sondern Folge von Handlung. Um das zu betonen wird auch von \emph{doing gender} gesprochen. 
\hyphentextcquote{english}[137]{West1987}{Doing gender means creating differences between girls and boys and women and men, differences that are not natural, essential, or biological.}
Somit ist das soziale Geschlecht per Definition immer Ergebnis einer Tätigkeit. Das lenkt das Interesse auf die handelnden Personen und den Raum, der sie so handeln lässt.
Diese Prozesse werden de-, oder wie \inparencite{Gildemeister1992} schreiben, re-konstruiert.

Unser Ziel ist es, sichtbar zu machen, welche Rolle Bücher bei der Konstruktion von Geschlechterunterschieden zwischen Mädchen und Buben spielen. Wir versuchen eine Kette von Akteuren zu bauen von der Strukturkategorie Geschlecht, also den Unterschieden zwischen Mädchen und Buben, bis zur Konstruktion des Geschlechts durch Kinderbücher. 



	%Buch als Akteur

Bücher verknüpfen eine große Anzahl an Menschen, die Leserschaft, die Autorin oder den Autor, verschiedenste Inhalte, Theorien und Einstellungen.
Das Besondere an Akteur-Netzwerken, wie Büchern, die keine Menschen sind, ist, dass sie ihre \emph{Arbeit}, wenn sie einmal da sind, mit viel weniger Aufwand als menschliche Akteur-Netzwerke verrichten.
Ein gutes Beispiel dafür ist der Hirte, der mit viel Aufwand seine Herde hütet und der Weidezaun, der, ist er einmal gebaut, dieselbe Arbeit allein durch seine Existenz verrichtet.
In unserer Welt gibt es viele Akteure, die ihre Arbeit verrichten, ohne dass wir die Arbeit als solche wahrnehmen.
Diese Arbeit, auf die man sich verlassen kann, wie auf das Wasser, dass das Mühlrad antreibt, erscheint uns als \emph{Stabilität}.
Diese Stabilität ist für uns schon so gewöhnlich geworden, dass sie natürlich erscheint,dieser Umstand verdeckt, dass sie, das durch ständigen Aufwand Produzierte ist.
Veränderung ist demnach nicht das zu Erklärende, sondern die Stabilität bzw. Ordnung, die von Akteuren aufrecht erhalten wird.


Will man die \emph{Mächtigkeit} eines Akteur-Netzwerkes definieren, so könnte man sagen, dass je mehr Akteure durch ein Akteur-Netzwerk miteinander verknüpft werden, es umso mächtiger ist.
Bücher haben die Fähigkeit unzählige Akteure miteinander zu riesigen Akteur-Netzwerken zu verbinden.
Von der Bibel wurden \zB geschätzte 2 bis 3 Milliarden Exemplare unters Volk gebracht. Sie verknüpft seit rund 2000 Jahren verlässlich Menschen und Werte auf der ganzen Welt.
Nicht nur bei der Bibel sehen wir, dass das Buch nicht nur verknüpft, sondern auch differenziert. Wer dieselben Bücher liest, gehört zusammen und grenzt sich so, von denen die es nicht tun, ab. Differenzen wie Kind/Erwachsener oder der Zugehörigkeit zu einer Nation, werden mit differenziertem Leseverhalten in Verbindung gebracht.
\parencites[Kap.\,3 in][]{Postman2011}[50]{McLuhan2012}

Es gibt bestimmte Prinzipien oder  Regeln die, wie \inparencite[10]{McKee2001} schreibt, bstimmen wie Geschichten funktionieren, aber nicht wie eine Geschichte auszusehen hat.  Sie sind die Sprache die Leserschaft und Autorenschaft sprechen um sich zu verstehen. \parencite[30]{Daehnke2003} Doch wie jede Sprache ist sie auch eine Eingrenzung. Sie gibt den Rahmen, den Diskursraum vor, in dem sich die Geschichten bewegen werden.

Wohl eines der augenscheinlichsten Elemente der Prinzipien des Schreibens ist die der Hauptfiguren, der Protagonistin oder des Protagonisten. 
	
Die Hauptfigur oder die Hauptfiguren\footnote{\blockcquote[155]{McKee2001}{Im Allgemeinen ist der Protagonist eine einzelne Figur. \textelp{} In \film{Panzerkreuzer Potemkin} bildet eine ganze Gesellschaftsklasse, das Proletariat, einen massiven \emph{Plural-Protagonisten}}
Plural-Hauptfiguren unterliegen zwei Bedingungen: sie müssen denselben Wunsch haben und gemeinsam Leiden oder profitieren. \parencite[155]{McKee2001}} sind ein großer Teil von dem oben angesprochenen Draht zur Leserschaft. Im Idealfall erkennen wir uns in der Hauptfigur wieder und wollen das sie bekommt was sie will. \parencite[161]{McKee2001} \blockcquote[161]{McKee2001}{Ein Publikum\textelp{}vermag zwar , sich in jede Figur einzufühlen, in Ihren Protagonisten aber muß\textins{:sic} es sich einfühlen. Wenn nicht, dann ist das Band zwischen Publikum und Story gerissen}
%\nocite{Eisenstein1925} 
Geht man davon aus, dass ein Band zwischen Leserschaft und Geschichte notwendig ist, dann geht das nicht, ohne dass sich die Leserin oder der Leser in die Hauptfigur einfühlen. Die Hauptfigur ist die Seele der Geschichte.  \blockcquote[407]{McKee2001}{Im Wesentlichen bringt der Protagonist die übrigen Rollen hervor. Alle anderen Figuren sind in einer Story in der Hauptsache deshalb, um zum Protagonisten eine Beziehung einzugehen und dazu beizutragen, allen Dimensionen der komplexen Natur des Protagonisten Gestalt zu verleihen.}  
Wenn sich nun die Leserschaft in die Hauptfigur einfühlt, mit ihr die Geschichte erlebt, dann hat dieses Erleben natürlich einen Einfluss auf die Leserschaft. Wichtig ist also, was die Hauptfigur \emph{erlebt}, wie sie mit ihrer Umwelt inter\emph{agiert}. Da eine ganze Leserschaft durch eine Hauptfigur gleich agiert, verbindet sie das.
	
Diese Grundannahmen betreffen auch Kinderbücher. Im nächsten Schritt soll geklärt werden was Kinderbücher sind, vorher soll aber noch ein Input zu der Konstruktion der Kindheit gegeben werden.

Kindheit und Medien

Bei \inparencite{Postman2011} heißt es, dass dadurch, dass das Wissen, das Kindern durch Bücher zugänglich (und nicht zugänglich) gemacht wird, die Kindheit überhaupt erst erzeugt wird. Erst durch die gezielte Auswahl und Herstellung von Kinderbüchern, die gewisse Aspekte des Lebens zeigen und andere ausblenden entsteht Kindheit. Kindheit ist somit ein geschützter Raum ohne Krankheit, Sexualität und Tod.
Gleichzeitig sind Kommunikationsmöglichkeiten in  der Lage, Kindheit wieder verschwinden zu lassen. Mit Harold Innis teilt er die Auffassung,  dass Veränderungen innerhalb der Kommunikationstechnik drei Auswirkungen haben: die Veränderung der Interessensstruktur (worüber wird nachgedacht?), den Charakter der Symbole (womit wird gedacht?) und das Wesen der Gemeinschaft (wo entwickeln sich die Gedanken?). \parencite[34]{Postman1985} Wenn er vom \enquote{Verschwinden der Kindheit} spricht, macht er die, durch die neuen elektronischen Medien vermittelten, Inhalte, die die kindliche Phantasie nicht mehr anregen, verantwortlich: Bilder und andere Darstellungsformen im Fernsehen, also vorrangig visuelle Medien, bieten der eigenen Vorstellungskraft, im Gegensatz zum Text in Büchern, wenig Entfaltungsmöglichkeiten. Gleichzeitig laufen Reflexions- wie Kritikfähigkeit Gefahr zu verkümmern, da nur elementare Fähigkeiten gebraucht würden. Außerdem kritisiert er, dass zunehmend für Erwachsene typische Wünsche transportiert werden, die die Neugier und Andersartigkeit des Kindseins gefährden, auch weil sie keine Geheimnisse mehr hüten. \parencite[93\psq]{Postman1985} Erfahrungsräume, die nur Literatur bietet, können verloren gehen. Lesesozialisation kann als Ausschnitt der Mediensozialisation gesehen werden: durch Lesen wird nämlich nicht nur die Fähigkeit zur Dekodierung von schriftlichen Texten gefördert, sondern es werden auch Kommunikationsinteressen und kulturelle Haltungen erworben.\footnote{Der Literatur wurde nicht immer eine positive Funktion zugeschrieben, gerade der Unterhaltungsliteratur warf man vor, Kinder von sinnvollen Tätigkeiten abzuhalten. Erst durch die Konkurrenz der elektronischen Medien schien der Umgang mit Texten förderungswürdig. } \parencite[22\psqq]{Weinkauff2010}

Kinderliteratur 

Obwohl sich Kinder- von Jugendliteratur anhand eigener Attribute abgrenzen lässt, bilden sie in theoretischen und empirischen Arbeiten meist eine Einheit, die im Kontrast zur Erwachsenenliteratur steht. Kinderliteratur kann anhand spezifischer Textmerkmale, Inhalte und Funktionen in verschiedene Genres  eingeteilt werden, zu denen etwa Kriminalgeschichten, Abenteuer oder Märchen zählen.  Außerdem werden Kinderbücher  im Allgemeinen mit Altersempfehlungen versehen, die im deutschsprachigen Raum meistens in 2- Jahresstufen angegeben sind, bei Jugendbüchern sind die Abstände meist größer. \parencite[10]{Ewers2011}
Als Mädchen- oder Bubenliteratur werden die Kommunikationen bezeichnet, die vorwiegend von  weiblichem oder männlichem Lesepublikum angenommen werden, gleichzeitig  scheinen manche Genres, ebenso wie Inhalte oder Gestaltungsstile von Büchern, explizit unterschiedliche Vorlieben von Buben und Mädchen anzusprechen und zu betonen.
Kinderliteratur wird von den gesellschaftlichen, wirtschaftlichen und politischen Verhältnissen der jeweiligen Zeit geprägt: Inhalte, der (ästhetische) Gebrauch von Sprache, erzieherische Absichten und pädagogische Konzepte wie Ansichten der AutorInnen haben sich seit der Entstehung dieses Literaturkonzepts stark verändert.  Die zeitgenössische Auffassung von Kindheit, die ein individualistisches, postmodernes Menschenbild und das Ideal eines autoritativ-partizipativen\footnote{Der autoritativ-partizipative Erziehungsstil  zeichnet sich durch Wärme, Wertschätzung, dem Vereinbaren von Regeln und begründeter Sanktionierung aus. Das Kind kann die Eltern- Kindbeziehung mitgestalten, es wird zwar geleitet, lernt aber selbständig Verantwortung zu übernehmen. \parencite[35]{Kuttler2009}} Erziehungsstils verfolgt, kann mit ziemlicher Sicherheit nicht mit den Normen- und Wertvorstellungen anderer Epochen oder Kulturkreisen verglichen werden. 
Zur Veranschaulichung kann eine Literaturform, die Ende des achtzehnten Jahrhunderts in England entstanden ist und speziell an Mädchen gerichtet war – die sogenannte  \emph{Backfischliteratur}- dienen. Ihr Hauptziel war, Mädchen auf die spätere Rolle als Hausfrau, Mutter und Ehefrau vorzubereiten. Frauen sollten vor allem demütige und religiöse Eigenschaften besitzen, außerdem war das Finden eines geeigneten Ehemannes von entscheidender Bedeutung. Allerdings hat sich die Mädchenliteratur inzwischen stark verändert: Während im traditionellen Mädchenbuch vorherrschende Rollenstereotype und traditionelle Wertmaßstäbe verinnerlicht werden sollen, wird im nächsten Entwicklungsschritt gegen diese protestiert, um dann im emanzipierten Mädchenbuch vor allem die Identitätsfindung zu betonen und sämtliche Rollenerwartungen abzulehnen. Selbst in das eigene Handeln eingreifen zu können und eine aktive Lebensgestaltung stehen, soweit dies für das Kind möglich ist, im Vordergrund. Mädchen und Jungen müssen heute ähnliche Anforderungen bewältigen, wenn es darum geht ein konsistentes Selbstbild zu entwickeln. 



\subsection{Darstellungen der Hauptfiguren}

Bei der Analyse ausgewählter Kinderliteratur der 1990er Jahre legte Anita Schilcher besonderen Fokus auf das Verhalten, der in den Texten vorkommenden Hauptfiguren, das Familiensetting und Bewertungen, die in den Texten vorkamen. Sie kam auf folgende Ergebnisse: Traditionelle Mädcheneigenschaften, wie Passivität, Empfindlichkeit, körperliche Schwäche oder mädchentypische, unpraktische Kleidungsvorlieben werden durchgehend negativ bewertet, während eine selbstbewusste, aktive, durchsetzungsstarke Mädchenfigur als Leitbild wirkt. Auch Jungen, die ein moderneres Rollenbild und Eigenschaften wie Sensibilität, Kreativität und Kommunikationsfähigkeit, vereinen, werden bevorzugt. Auffallend ist, dass berufstätige Mütter gleichzeitig Familien- und Hausarbeit leisten und eine nahezu perfekte, alles vereinende und deswegen vielleicht sogar unrealistische Frauenrolle inne haben. Väter kommen in den meisten Texten seltener vor, da karrierebedingte Entscheidungen, die meist zu längeren Arbeitszeiten führen, öfter im Vordergrund stehen. Dadurch sind sie auch deutlich weniger ins alltägliche Familienleben eingebunden. Weiters gehen Männer kaum in Karenz und sind viel seltener geringfügig beschäftigt, was den tatsächlichen gesellschaftlichen Verhältnissen noch immer entspricht. Frauen spielen zwar durch ihre Berufstätigkeit in ehemalig reinen Männerdomänen mit\footnote{Gerade in höheren Positionen, sowie in naturwissenschaftlich- technischen Gebieten, sind wenig Frauen zu finden. Diese Tätigkeitsbereiche sind im Allgemeinen von sehr gutem Verdienst gekennzeichnet, während soziale (eher weiblich dominierte) Berufe vergleichsweise unterbezahlt sind. Dass die unterschiedliche Verteilung von Männern und Frauen auf die einzelnen Berufsgruppen, nicht der einzige Grund, für geschlechtsabhängige Lohndifferenzen sind, sei hier nur erwähnt.}, fallen aber nach der Ankunft ihres ersten Kindes in traditionelle Rollenmodelle zurück und widmen ihre Zeit in viel höherem Ausmaß als Väter (unbezahlter) Familien- und Hausarbeit, weshalb sie auch Teilzeitarbeitsmodelle  erheblich häufiger in Anspruch nehmen. Die Vermutung, dass Frauen vielfältigere, traditionelle wie moderne Eigenschaften vereinen (müssen) und Männer sich in einem weniger breiten Spektrum bewegen, wird, in der bereits analysierten modernen Kinderliteratur, bestätigt.
	

Allerdings bedeuten die geschlechtsspezifischen Rollenentwürfe der in der Literatur vorkommenden Figuren nicht, dass der/die junge LeserIn diese unvermittelt verinnerlichen. Sie werden natürlich (vorwiegend unbewusst) wahrgenommen, aber vor dem jeweiligen kindlichen Erfahrungshintergrund  in der Gedankenwelt konstruiert. Prädispositionen von Mädchen und Buben beeinflussen folglich auch die Akzeptanz oder Ablehnung eines Lesestoffs. Wenn Kinder also nicht gezwungen sind, sich mit einem bestimmten Lektüreangebot zu beschäftigen, hängt die Leseentscheidung von Belohnungen ab, die erwartet werden. Diese sind intrinsischer Natur und können auf emotionaler, sozialer oder kognitiver Ebene erfolgen: Der Wunsch, bei Themen, die gerade \emph{in} sind, mitreden zu können, kann die Motivation ein Buch zu lesen ebenso beeinflussen wie das Bedürfnis dabei die eigene Fantasie anzuregen und in andere Rollen zu schlüpfen, den persönlichen Wissensdurst zu stillen oder einfach Spaß bei dieser Form der Unterhaltung zu haben. \parencite[547\psq]{Kuhn2010}



Nach den bisherigen theoretischen Ausführungen ist es Aufgabe dieses Teilbereiches die methodischen Herangehensweisen rund um das Thema Kinderbuch vorzustellen. Schlussendlich beschäftigen wir uns hier mit den grundsätzlichen Fragen: Gibt es Mädchen- und Bubenbücher denn überhaupt und wenn ja, wie unterscheiden sie sich. Regalreihen mit den Hinweistafeln \enquote{Mädchen 8--12} sind ein praktisches Beispiel, wie bestimmte Kinderbücher einer Altersgruppe und vor allem einem Geschlecht zugeordnet werden.  Die dabei wohl ergiebigste und spannendste Forschungsmethode stellt die Inhaltsanalyse dar, die besonders geeignet ist um schriftliche und bildliche Darstellungen vergleichen zu können. Es liegt daher nahe die Vor- und Hinterseiten -- das Buchcover -- gesondert von den Inhalten der zahlreichen Geschichten, Abenteuer und Erzählungen zu untersuchen.

Konzentriert man sich auf den zuerst genannten Punkt einer unterschiedlichen \emph{Aufmachung} sind die Arbeiten von Erving Goffman, der bereits früh mit Untersuchungen von unterschiedlichen bildlichen Darstellungen der Geschlechter begonnen hat, eine fruchtbar Anregung. Goffman verwendete Werbegraphiken um aufzuzeigen, in welchen Rollen Männr und Frauen dargestellt werden. Seiner Interpretation nach machen immer wiederkehrende Konstellationen und dargestellte Situationen es möglich, Aussagen über Rollenerwartungen zu tätigen. Reklamebilder sind für ihn eine \textcquote[104]{Goffman1981}{verallgemeinernde Darstellung einer heimlichen Thematik der Geschlechter, vor allem des weiblichen Geschlechts.} Auch wenn es sich bei den Abbildungen um keine reale Situation handelt, sondern diese Bilder rein zum Zweck der Werbung inszeniert werden, kann festgehalten werden, dass die Reklame \textcquote[111]{Goffman1981}{von den Betrachtern als gar nicht ungewöhnlich, als etwas \enquote{ganz Natürliches} aufgefaßt wird.} Verknüpfen wir diese Erkenntnis mit unseren Buchcovern, so kann davon ausgegangen werden, dass Kinder die bildlich dargestellten Geschlechterrollen -- egal ob Geschlechterklischee oder nicht -- als gängig und konventionell interpretieren. Goffman legte zur Analyse der Reklame einige Indikatoren fest, wie beispielsweise die \emph{relative Größe}. Wollen Fotografen eine Person auf einem Bild besonders mächtig bzw. autoritär erscheinen lassen, so wird die Körpergröße gerne verwendet um den erwünschten Eindruck zu hinterlassen. Untergebene werden oftmals sitzend, hockend oder schlichtweg kleiner dargestellt. Dies könnte auch als Indikator für Kinderbuchcover übernommen werden, aber auch die restlichen von Goffman festgehaltenen Merkmale \enquote{weibliche Berührung}, \enquote{Rangordnung nach Funktion}, \enquote{Familie}, \enquote{Rituale der Unterordnung} und \enquote{zulässiges Ausweichen} können brauchbare Elemente für Buchdeckel-Analysen enthalten. 

Eine etwas andere Herangehensweise verfolgt eine Studie aus dem Jahre 1988 von \citeauthor{Schmerl1988}. Die Studie behandelt inhaltsanalytisch eine repräsentative Stichprobe von in Kindergärten und Vorschulen häufig gelesenen Bilderbüchern. Da Bilderbücher oftmals eine Kombination von Bild und Text sind, die in Bezug zu einander stehen, kann ein Vergleich zu einem Buchcover gezogen werden. Auch der Umstand, dass der bildlichen Darstellung ein besonderer Stellenwert zugesprochen wird, ähnelt einem Cover eines Kinderbuches sehr stark. Die Bücher wurden mittels einer Bestandsaufnahme von 29 Kindergärten der Umgebung ausgewählt, wobei nach einigen Ausschlussverfahren 52 Werke übrig blieben, die in späterer Folge analytisch auf 15 Inhaltskategorien untersucht wurden. Passenderweise wurden Texte und Bilder separat analysiert und ausgewertet. \parencite[133\psq]{Schmerl1988}

Beispielsweise wurden Geschlechterproportionen der handlungstragenden Figuren analysiert, was anhand des Buchcovers meistens kein Problem darstellt: Es konnten in dem Kinderbilderbücher-Sampling der Studie von \citeauthor{Schmerl1988} insgesamt 62 Hauptfiguren identifiziert werden, wovon jedoch nur 55 einem Geschlecht zugeordnet wurden. Davon wurden 17 (30,9\%) dem weiblichen und 38 (69,1\%) dem männlichen Geschlecht zugeschrieben. Diese Proportionen ähneln dem allgemeinen Geschlechterverhältnis aller Charaktere sehr stark. Auch bei den weiteren subkategorischen Vergleichen sind ähnliche Tendenzen festzuhalten. So ist der Unterschied bei Kindern um einiges geringer als bei erwachsenen Figuren (Frauen zu Männern = 1:3,8; Mädchen zu Buben = 1:1,4) und auch bei dem Verhältnis Bild zu Text zeigen die verschriftlichten Darstellungen eine geringere Differenz auf. 
Außerdem wurde der Frage nachgegangn, ob Autorinnen ein ähnliches Geschlechterverhältnis wie ihre männlichen Kollegen in ihren Werken wiedergeben. 
Auch wenn die Unterschiede zwischen bildlichen und schriftlichen Darstellungen etwas variieren, konnte einheitlich festgehalten werden, dass Autorinnen um ein ausgewogenes Geschlechterverhältnis bemüht waren, während ihre männlichen Kollegen in ihren Charakteren ein extremes Ungleichgewicht bei den Geschlechtern vorwiesen. Diese Erkenntnis macht es notwendig den Autor bzw. die Autorin bei einer Analyse eines Kinderbuches festzuhalten und die Inhalte ihrer Werke auf den Gebrauch von Geschlechterrollen zu untersuchen. 

Als klassisches Beispiel einer Inhaltsanalyse könnte die bereits oben beschriebene Studie von \citeauthor{Schmerl1988} genannt werden. So erheben \citeauthor{Schmerl1988} beispielsweise die unterschiedliche Darstellung von Männern und Frauen bei Beruf, Familienarbeit und Kommunikation, aber auch Differenzen bei Emotionalität, Bedürfnissen und Verhalten (aggressiv, passiv und/oder altruistisch). \parencite{Schmerl1988} 

Eine Möglichkeit wäre, die aktive Beschäftigung mit der Umwelt ins Blickfeld zu rücken. Dabei wurde in der vorligenden Studie zwischen „körperlichen“ (suchen, sammeln, essen, ernten, angeln, tanzen, sich verstecken, fotografieren) und „geistigen“ (denken, überlegen, träumen, sich erinnern, Ideen haben, etwas wissen, beschließen) Aktivitäten unterschieden. Die Proportionen waren besonders bei den körperlichen Betätigungen -- klar verteilt. Männliche Figuren wurden im Durchschnitt dreimal so oft wie weibliche Figuren bei derartigen Aktivitäten dargestellt.  Bei geistigen Tätigkeiten fallen diese Zahlen etwas ausgeglichener aus. 
Eigenschaften lassen sich am besten mittels Personenbeschreibungen ermitteln.  Unter äußeren Eigenschaften wurden körperliche Beschreibungen verstanden, so wurden Mädchen als klein, zierlich, anmutig und/oder leichtfüßig beschrieben und Frauen als schön, reizend, jung und/oder alt. Das männliche Geschlecht wurde äußerlich nur sehr wenig beschrieben, etwa als alt oder stark. Bei Jungen fehlte die Beschreibung äußerer Eigenschaften sogar gänzlich. Bei den inneren Eigenschaften, also Beschreibungen von Charakterzügen, wurde zwischen positiven und negativen Eigenschaften unterschieden.  \textcquote[145]{Schmerl1988}{Das absolute wie relative Überwiegen negativer Eigenschaftsbeschreibungen weiblicher Figuren ist vor dem Hintergrund der sonstigen durchgehenden Unterrepräsentierung von Mädchen und Frauen als besonders diskriminierende Konstellation zu bewerten.} %\parencite[144\psq]{Schmerl1988}

Als eine Beispielarbeit für modernere Analysen dient die Diplomarbeit von Carl Pick von 2009, die sich mit der seriellen Narration beschäftigt und dabei die Bedeutung von unterschiedlichen Handlungssträngen beschreibt. Es gibt zwei Arten von Handlungssträngen, die am besten mittels ihrer Verwendung in seriellen Narrationen erklärt werden. Unter seriellen Narrationen gibt es fünf Typen: Serie, Reihe, Fortsetzungsroman, Zyklen und Mehrteiler. \parencite[12--18]{Pick2009} Oftmals synonym verwendet sind sie medienwissenschaftlich von einander zu unterscheiden. Eines der Hauptmerkmale um sie voneinander trennen zu können, ist die unterschiedliche Verwendung der Handlungsstränge. Es gibt zwei Arten von Handlungssträngen. \parencite[23\psq]{Pick2009}

\pagebreak % um Aufzählung nicht zu zerreißen.
				\begin{itemize}
					\item Übergeordnete Handlungsstränge
					\item Untergeordnete Handlungsstränge
				\end{itemize}

Übergeordnete Handlungsstränge werden oftmals auch als Hauptkonflikte oder Hauptziel bezeichnet.Untergeordnete Handlungsstränge wären somit der Überbegriff für sämtliche Abenteuer, Erlebnisse die ein Protagonist durchleben muss um sein Hauptziel zu erreichen oder den Konflikt zu lösen. Untergeordnete Handlungsstränge werden oft auch folgenimmanente Handlungsstränge genannt. Sie müssen aber nicht immer auf genau ein Buch oder Kapitel einer seriellen Erzählung bezogen sein. Es kommt auch vor, dass untergeordnete Handlungsstränge über mehrere Teile beschrieben werden oder auch mehrere solcher Handlungsstränge in einem Teilwerk vorkommen. \parencite[23\psq]{Pick2009}
				
Diese Unterteilung in übergeordnete und folgenimmanente (bzw. untergeordnete) Handlungsstränge, kann vor allem bei seriellen Erzählungen verwendet werden und birgt daher für die inhaltliche Analyse von Kinderbüchern eine große Chance, da viele von Kindern gern gelesene Bücher Serien, Reihen, etc. sind. Versucht und damit geklärt, gehört vor allem auch, ob diese Unterteilung auch für Einzelwerke verwendet werden kann, da auch hier vermutet wird, dass es Haupt- und Nebenstränge gibt.
				
Im Hinblick auf unsre Forschungsfrage ist diese Herangehensweise besonders interessant und kann zeigen ob ein Geschlecht verstärkt in inem der beiden Handlungsstränge vorzufinden ist. Um jedoch eine Aussage über die Tragweite einer unterschiedlichen Ausprägung von weiblichen oder männlichen Charakteren in Haupt- und Nebensträngen erzielen zu können, muss erklärt werden, ob Leser sich mit dem Hauptprotagonisten oder mit Charakteren des eigenen Geschlechtes identifizieren. Um die Annahme zu berücksichtigen, dass ein Leser sich vorranging mit dem Hauptprotagonisten identifiziert, auch wenn dieser womöglich nicht dem eigenen Geschlecht angehört, sollen zusätzlich die Handlungsstränge zwischen Mädchen- und Bubenbüchern verglichen werden. Etwaige Unterschiede beim Aufbau, Ablauf, Inhalt und/oder Ausgang der Handlungsstränge könnten somit Aufschluss über die Bevorteilung eines Geschlechts zeigen.

Die Inhaltsanalyse bietet zahlreiche Möglichkeiten im Feld der Kinderbücher zu aussagekräftigen Erkenntnissen zu kommen. Oft wurde diese Möglichkeit jedoch nicht genutzt und nur an der Oberfläche gekratzt. Fest steht dennoch, dass sich Kinderbücher besonders gut dazu eignen, Geschlechterrollen aufzuzeigen und weiter zu vermitteln. Wie wir festgestellt haben wurden weibliche, im Vergleich zu männlichen Beschreibungen oft benachteiligt und unwahr präsentiert. Auch konnte gezeigt werden, dass Autorinnen sich viel mehr um eine ausgewogene Präsenz von weiblichen zu männlichen Figuren bemühten als ihre männlichen Berufskollegen. Zusammenfassend darf gesagt werden, dass Unterschiede festgestellt wurden und in der moderneren Kinderliteratur auffallend oft versucht wird, diese Klischees nicht länger zu verwenden. 

Der Schlüssel zu einer aussagekräftigen Analyse liegt wohl auch in der Offenheit der Forscher, neue Wege zu gehen. Einer dieser möglichen Ansätze wäre eine stärkere Gewichtung der HauptprotagonistenInnen und die reine Konzentration auf Geschlechter  zu erweitern. 
