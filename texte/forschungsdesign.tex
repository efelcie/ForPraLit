
\section{Forschungdesign}

  Im Rahmen unseres Forschungspraktikums mit dem Titel
  \emph{Geschlechtsidentität und Geschlechterrollen bei Kindern}, möchten wir
  untersuchen, wie sich Kinderbücher, je nachdem ob sie eher von Mädchen oder
  Buben gelesen werden, unterscheiden und welche Merkmale für unterschiedliche
  Lesepräferenzen entscheidend sein können. Wir vermuten, dass sich manche
  Kinderbücher klar als Mädchen- oder Bubenbücher einordnen lassen, während
  andere von beiden Geschlechtern in ähnlichem Ausmaß gelesen werden.

  Bei unserer bisherigen Recherche sind wir auf Forschungen zu
  Geschlechtsstereotypen in Bilderbüchern gestoßen. Hier wurde beispielsweise
  inhaltsanalytisch untersucht, welchem Geschlecht die Hauptfiguren zuzuordnen
  sind, welche Tätigkeiten die dargestellten Personen ausüben oder welche
  Eigenschaften Frauen und Männer haben (verhalten sie sich eher aktiv oder
  passiv, eher aggressiv oder empathisch). \parencite{Schmerl1988}

  Aufgrund des Umfangs dieses Projektes können wir nicht analysieren, wie
  Kinderbücher das geschlechtsspezifische Handeln von Personen beeinflussen. Wir
  können lediglich annehmen, dass Literatur und andere Medien, mit denen Kinder
  aufwachsen, sowie ihre Umgebung und persönlichen Dispositionen, insgesamt, in
  unterschiedlichem Gehalt, ihren Beitrag zur Veränderung oder Beibehaltung von
  Geschlechterrollenstereotypen leisten. Kinder reagieren aktiv auf Inputs und
  können Bücher annehmen oder ablehnen.

  Diverse Medien berichten immer häufiger von einem schulischen Leistungsabfall
  bei Buben, im Vergleich zu den eher \emph{braven} Mädchen. Tatsächlich lesen
  Mädchen tendenziell mehr Bücher als Buben, Buben greifen hingegen eher zu
  Sachbüchern und Comics. Dieses Ungleichgewicht müssen wir bei der Auswertung
  unseres Fragebogens berücksichtigen.

  \minisec{Definition}   Wir bezeichnen ein Kinderbuch erst als Mädchen- oder
Bubenbuch, wenn sich der Großteil seiner LeserInnen einem Geschlecht zuordnen
lässt.

  \minisec{Fragestellungen}

  \begin{itemize} 
      
    \item Lesen Buben andere Bücher als Mädchen?     
    \item Lassen sich Merkmale feststellen, die die Mehrheit der Mädchenbücher gemeinsam haben?     
    \item Sind diese Merkmale gleich erkennbar, wenn man ein Buch in die Hand nimmt? (Gestaltung des Umschlags (Farbe, Bild), Titel und Untertitel des  Buches, Klappentext, Geschlecht des/r AutorIn,\ldots)     
    \item Gibt es inhaltliche Unterschiede, wenn ein Buch als Mädchen- oder Bubenbuch eingeordnet werden kann?      
    \item Welche Eigenschaften kennzeichnen die Hauptfiguren in den Büchern? Sind Geschlechterrollen starr oder flexibel?   

  \end{itemize}

  \minisec{Hypothesen}   \begin{itemize}     \item Buben lesen tendenziell
andere Bücher als Mädchen.     \item Je dunkler das Buchcover gestaltet ist,
desto eher handelt es sich um ein Bubenbuch.     \item Autorinnen schreiben eher
Bücher, die Mädchen ansprechen als ihre männlichen Kollegen.     \item Je
wichtiger es für die Hauptfigur(en) ist, eine Aufgabe zu erfüllen, desto eher
handelt es sich um ein Bubenbuch.     \item Je stärker ein Buch von Mädchen
gelesen wir, desto mehr Adjektive findet man.     \item Wenn der Titel Namen der
Hauptfiguren enthält, weisen Mädchennamen auf Mädchenbücher hin.     \item Wenn
ein Buch thematisch Stereotype bei Vorlieben von Mädchen und Buben aufgreift,
können diese die Leseentscheidung beeinflussen: Je eher Themen wie Fußball,
Abenteuer, Indianer oder Drachen vorkommen, desto eher handelt es sich um ein
Bubenbuch und je eher die Geschichten  auch von Pferden oder Prinzessinnen
handeln, desto eher wird ein Buch von Mädchen gelesen.   \end{itemize}

  \subsection{Erhebungsmethoden}

    Orientiert an den Forschungsfragen und Hypothesen wurden zwei Methoden zur
    Beantwortung ausgewählt. Dabei handelt es sich um eine Fragebogenerhebung
    und die Inhaltsanalyse. Offen bleibt eine etwaige dritte Methode --
    Experteninterview -- die im Falle von \emph{unklaren} Ergebnissen in
    Betracht gezogen wird.

    \minisec{Fragebogenerhebung}

    Will man wissen ob Mädchen andere Bücher bevorzugen als Buben, so liegt es
    nahe dies mittels eines Fragebogens zu klären. Dabei ist das Ziel mindestens
    400 Kinder der dritten und vierten Klassen von Volksschulen in Graz zu
    befragen. Es ist hierbei eine Kooperation mit einer anderen Gruppe des
    Forschungspraktikums geplant, die an den Fernsehpräferenzen der Kinder
    interessiert war. Aufgrund der ähnlichen Thematiken sollte eine gemeinsame
    Gestaltung eines Fragebogens keine Probleme darstellen. Aufgrund der
    unzählbaren Lese-Möglichkeiten ist es nachvollziehbar, dass eine reine
    offene Fragestellung nur wenig zielführend ausfallen würde. Daher soll
    anhand von Bestsellerlisten, Listen von Klassikern der Kinderliteratur und
    Bibliotheksaufzeichnungen eine Liste von maximal 40 Kinderbüchern erstellt
    werden, die gerne und oft gelesen werden und das Lesespektrum von Kindern
    der befragten Altersgruppe repräsentiert. Obwohl bekannt ist, dass Buben
    weniger lesen und nur bei Sachbüchern die Nase vorn haben, soll auf diese
    Büchergruppe bewusst verzichtet werden, da sie für eine Auswertung von
    Geschlechterrollen nur wenig Gewinn verspricht. Ergänzt werden soll die
    Bücherliste noch mit einer offenen Frage nach dem Lieblingsbuch, die als
    Kontrollfrage verwendet werden kann, außerdem wollen wir die thematischen
    Präferenzen der Kinder erfragen.

    \minisec{Inhaltsanalyse}

    Es ist zu klären, ob sich bei den von Mädchen und Buben gelesenen Büchern
    tatsächlich Unterschiede finden lassen. Die Auswahl der Inhaltsanalyse als
    Methode liegt hier auf der Hand. Wir wollen hier jedoch versuchen, die
    Analyse in zwei Schritten durchzuführen. Neben dem Inhalt sollen
    Mädchenbücher und Bubenbücher auch auf \emph{oberflächliche} Merkmale und
    deren Unterschiede untersucht werden. Die beiden Analyseschritte werden
    daher \emph{Buchcover-Analyse} und \emph{Buchinhalt-Analyse} genannt.

    \minisec{Buchcover-Analyse}

    Wir vermuten, dass sich Bubenbücher bereits oberflächlich von Mädchenbüchern
    unterscheiden. Das Buchcover, das in unserer Analyse nicht nur als
    Vorderseite sonder als Summe aller oberflächlichen Merkmale eines
    Kinderbuches verstanden werden soll, bietet genügend Merkmale um eine
    Analyse zu ermöglichen. Ein Kinderbuchcover enthält üblicherweise ein
    farbenfrohes Design mit Abbildungen, die Bezug auf den Inhalt nehmen, den
    Namen des Autors oder der Autorin, einen Titel, der womöglich mit einem
    Untertitel ergänzt wird, Empfehlungen für das Alter und eine kleine
    Beschreibung des Inhalts auf der Rückseite. Außerdem kann eine Person
    bereits aufgrund der Stärke eines Buches auf den ersten Blick den Umfang
    desselbigen abschätzen. Für die Variablen die Erhoben werden siehe
    Tabelle~\ref{var}.

    %Tabelle Variable Erhebungsmerkma

    \ctable[       caption = {Variablen},       label   = var,       pos   =
htp,       width = \textwidth,     ]{l>{\raggedright}X}{}{       \FL   \small
Variable       & \small Erhebungsmerkmal       \ML   Geschlecht der Hauptfigur
& Klappentext, Buchtitel       \NN   Geschlecht von Autorin/Autor& Name
\NN   Helligkeit          & Mittelwert des Histogramms des Covers       \NN
Länge des Titels      &   Buchstabenanzahl       \NN   Bild--Text-Verhältnis   &
Flächenverhältnis zwischen Grafik und Text am Cover       \NN   Komplexität
&   Berechnung der Komplexität mit \emph{Graph-Cut-Algorismus} oder über
Fouriertransformation.       \NN   Dicke           &   Anzahl der Seiten
\LL      }

    Zusätzlich wird erwogen die bildlichen Darstellungen der Buchvorderseiten zu
    analysieren. Die dabei erwogene Methode lehnt sich an Goffmans
    \emph{Geschlecht und Werbung} an. Welche genauen Merkmale (und ob vielleicht
    sogar selbige wie bei Goffman) verwendet werden können, steht erst nach
    einer ersten Begutachtung aller relevanten Kinderbücher fest.

    Zur Erhebung werden alle Bücher verwendet, die bei der Fragebogenerhebung
    mehr als 50 Nennungen haben. (Siehe Tabelle~\ref{top30})

    %Tabelle relevante Bücher
      
      
      \ctable[       %  cap    = ,         caption = {Bücher die über 50 mal
genannt wurden},         label   = top30 ,         pos   = htp,       %  width
= \textwidth       ]{lD{,}{,}{0}D{,}{,}{0}D{,}{,}{0}D{,}{,}{3}}{
\tnote{1: 100\% Leserinnen; 0: gleich viele Leserinnen wie Leser; --1: 100\%
Leser}       }{                         \FL \small Bücher &
\multicolumn{1}{c}{\small Mädchen} & \multicolumn{1}{c}{\small Buben} &
\multicolumn{1}{c}{\small Gesamt} & \multicolumn{1}{c}{ \small w/m-Faktor\tmark}
\ML Die wilden Fußballkerle & 43 & 110 & 153 & -0,438       \NN Tiger-Team & 49
& 69 & 118 & -0,169       \NN Knickerbockerbande & 48 & 67 & 115 & -0,165
\NN Gregs Tagebuch & 86 & 117 & 203 & -0,153       \NN Harry Potter & 95 & 125 &
220 & -0,136       \NN Die drei ??? & 93 & 122 & 215 & -0,135       \NN Das
magische Baumhaus & 84 & 105 & 189 & -0,111       \NN Der kleine Ritter Trenk &
42 & 52 & 94 & -0,106       \NN Tom Turbo & 92 & 113 & 205 & -0,102       \NN
Der kleine Drache Kokosnuss & 46 & 52 & 98 & -0,061       \NN Der Räuber
Hotzenplotz & 92 & 101 & 193 & -0,047       \NN Sams & 63 & 67 & 130 & -0,031
\NN Fünf Freunde & 114 & 118 & 232 & -0,017       \NN Die Olchis & 47 & 48 & 95
& -0,011       \NN Der Grüffelo & 58 & 54 & 112 & 0,036       \NN Die Geggis &
36 & 31 & 67 & 0,075       \NN Peter Pan & 90 & 73 & 163 & 0,104       \NN Der
Regenbogenfisch & 122 & 95 & 217 & 0,124       \NN Baumhausgeschichten & 29 & 22
& 51 & 0,137       \NN Geschichten von Franz & 83 & 60 & 143 & 0,161       \NN
Pinocchio & 96 & 68 & 164 & 0,171       \NN Das kleine Wutmonster & 34 & 23 & 57
& 0,193       \NN Der kleine Eisbär & 91 & 56 & 147 & 0,238       \NN Pipi
Langstrumpf & 141 & 75 & 216 & 0,306       \NN Die kleine Hexe & 109 & 52 & 161
& 0,354       \NN Hexe Lilli & 162 & 53 & 215 & 0,507       \NN Die wilden
Hühner & 77 & 25 & 102 & 0,510       \NN Mini & 59 & 16 & 75 & 0,573       \NN
Conni & 94 & 22 & 116 & 0,621       \NN Prinzessin Lillifee & 109 & 14 &
123&0,772  \LL       }
      


    \minisec{Buchinhalt-Analyse}

    Die Analyse des Buchinhaltes wird wiederum in zwei unterschiedliche
    Erhebungen gesplittet. Während die quantitativ motivierte Inhaltsanalyse für
    alle 30 bereits oben festgehaltenen Bücher verwendet wird, soll eine
    qualitative Buchinhalt-Analyse nur auffällige Werke, die besonders viele als
    typisch erhobene Merkmale aufweisen, exemplarisch ausgewertet werden und zum
    strukturellen Aufbau der Abschlussarbeit verwendet werden.

    \minisec{Quantitative Buchinhalt-Analyse}

    Textkorpus der quantitativen Inhaltsanalyse sind die ersten 100 Wörter der
    Bücher die mindestens 50 Nennungen haben. (siehe Tabelle~\ref{top30}) Die
    Wörter in diesem Abschnitt werden gezählt und gruppiert. Es werden die
    Anzahl der Adjektive, Verben und Nomen verglichen. Weiters wird nach
    (stereo-)typischen weiblichen und männlichen Wörtern gesucht. Weiters wir
    Untersucht ob die Sätze, in denen die Hauptfiguren vorkommen, im Aktiv oder
    Passiv geschrieben sind.

    \minisec{Qualitative Buchinhalt-Analyse}

    Diese Form der Inhaltsanalyse bietet uns die Möglichkeit, nach den
    vorangegangenen Analysen, jene Bücher zu wählen, die besonders gut geeignet
    sind, um möglichst viele, der durch vorangegangenen Analysen festgehaltenen
    Merkmale, zu erklären. Diese Kinderbücher sollen qualitativ
    inhaltsanalytisch untersucht werden und zum strukturellen Aufbau einer
    Abschlussarbeit dienen.



    \singlespacing       \ctable[       caption = Zuordnung von Inhalten zu
Methoden, % Tabellen Überschrift       label   = zuo,       cap   = Zuordnung:
Inhalte--Methoden,   % Kurztitel f. Tabellenverz.       pos   = tbp,          %
Positon d. Tabelle       width   = \textwidth,       % Tab.br. \textwidth,
\columnwidth     ]{>{\raggedright}Xcccc}{            % Aufteilung d. Spalten
%\tnote{Die Spalten sind in der selben Reihenfolge wie die Methoden im Text
angegührt sind.}  % Fußnoten     }{              % Hier beginnt die Tabelle
\FL \small Fragestellung  & \begin{sideways}\small Fragebogen\end{sideways}&
\begin{sideways}\small Cover-Analyse\end{sideways}& \begin{sideways}\small
quant. Inhaltsa.\end{sideways}& \begin{sideways}\small Qual.
Inhaltsa.\end{sideways}        \ML Gibt es Unterschiede bei den Lesepräferenzen
von Mädchen und Buben? (Bubenbücher, Mädchenbücher)                         & x
&   &   &        \NN[0.5em] Gibt es Unterschiede zwischen Mädchen- und
Bubenbücher? (Erscheinung, Inhalt, Aufgeben, \ldots)                         &
& x & x & x       \NN[0.5em] Kann man Bubenbücher anhand von
\emph{oberflächlichen} Merkmalen von Mädchenbüchern unterscheiden? (Farben,
Themen, Umfang, Autorengeschlecht, \ldots)                        &   & x &   &
\NN[0.5em] Kann man Bubenbücher anhand von inhaltlichen Merkmalen von
Mädchenbüchern unterscheiden? (Schreibweise, Stereotype, \ldots)
&   &   & x &        \NN[0.5em] Sind Unterschiede tatsächlich in den bevorzugten
Büchern auffindbar? (Rollensettings, Lösungen von Aufgaben, \ldots)
&   &   &   & x       \NN[0.5em] Gibt es Bücher die die Einteilungen (Mädchen-
und Bubenbuch) besonders gut repräsentieren? (Welche)                         &
& x & x & x       \LL     }     \onehalfspacing


  \subsection{Zeitplan}     Die Studie wird innerhalb der Lehrveranstaltung
durchgeführt. Die Lehrveranstaltung dauert vom Sommersemester 2012 bis zum
Wintersemester 2012/13. Für eine Überblick siehe Tabelle~\ref{zeit}.
\singlespacing     \ctable[       caption = Zeitplan,       label   = zeit,
cap   = Zeitplan,       pos   = htp,       width   = \textwidth,
]{l>{\raggedright}X}{}{       \FL     Mai & Literaturstudium, Kontaktaufnahme zu
Schulen, Konstruktion der Erhebungsbögen, Leitfäden für Fokusgruppengespräche,
Kontaktaufnahme zu Experten.       \NN[0.5em] Juni & Durchführung der Erhebungen
und Fokusgruppengespräche, Zusammenfassung von Literatur.       \NN[0.5em]Juli--
Spt.  & Dateneingabe und Analyse der Erhebungsfragebögen, Expertengespräche,
Umschlaganalyse, Tiefenanalyse, Analyse der Interviews.       \NN[0.5em] Oktober
& Abschluss der Analyse, Ergänzende Schritte.

      \NN[0.5em] November & Anfertigung der Erstfassung des Forschungsbericht.
\NN[0.5em] Dez.--Jän.& Endfassung anfertigen, Präsentation erstellen.
\NN[0.5em] Februar  & Ergebnisse den Beteiligten zukommen lasse,
Projektabschluss.       \LL     }     \onehalfspacing
