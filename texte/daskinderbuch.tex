
\section{Das Kinderbuch}   \emph{Lisa Weiler}   \smallskip

  %1. Kinderliteratur

  \noindent Bevor wir uns mit der historischen Entwicklung der Kindheit und
  \ihrer Literatur, dem Wandel von Ansprüchen und Zielen der Kinderlektüre und
  \geschlechtsspezifischen Rollenangeboten, die in den Texten vermittelt werden,
  \auseinandersetzen, möchten wir versuchen zu definieren, was unter einem
  \Kinderbuch zu verstehen ist.
    
  \subsection{Begriffsbestimmung}     Obwohl sich Kinder- von Jugendliteratur
anhand eigener Attribute abgrenzen lässt, bilden sie in theoretischen und
empirischen Arbeiten meist eine Einheit, die im Kontrast zur
Erwachsenenliteratur steht. In der Regel richten sich kinder- und
jugendliterarische Kommunikationen an ihre jungen Adressaten. Zwischen infrage
kommender Literatur, die ausdrücklich für Kinder und Jugendliche verfasst wird
und dem lediglichen Konsum dieser, kann eine Grenze gezogen werden: Auf der
einen Seite steht die intendierte\footnote{Intentionale Kinderliteratur kann
auch die Gesamtheit aller Kinder- und Jugendliteratur, die in einer Epoche als
solche ausgegeben wird, meinen. Die Problematik dieser Definition besteht darin,
dass es nahezu unmöglich ist, die gesamte Kinder- und Jugendlektüre einer Epoche
als homogene Textsorte bestimmen zu können. \parencite[192\psq]{Ewers2008}}  und
auf der anderen die nicht- intendierte Kinder- und Jugendliteratur. Die
intendierte Kinder- und Jugendliteratur muss nicht von Anfang an an die junge
Zielgruppe adressiert sein, sondern kann auch weitergeleitet oder im Nachhinein
kindgerecht zugeschnitten werden. Neben dem Kinderbuch mit all seinen weiteren
Ausdifferenzierungen werden auch andere Publikationsmedien, wie etwa
Zeitschriften und Comics, genutzt, um das kindliche Publikum zu begeistern und
eine Fülle an Angeboten zu gewährleisten.  \parencite[5]{Ewers2011}  Im Hinblick
auf unsere Fragestellung und empirische Vorgehensweise, werden wir uns in
weiterer Folge begrifflich auf Kinderliteratur beschränken.

    Da Kinderliteratur stets auch einen pädagogischen Anspruch verfolgte, wurde
    das Medium Buch zur gezielten Weitergabe von Werten und Normen genutzt:
    Dem/r jungen LeserIn wurden somit (in)direkte Handlungsempfehlungen gegeben.
    Laut Hurrelmann kommt der Literatur die zentrale Aufgabe der Vermittlung von
    Wissen und Werten zu; es kann behauptet werden, dass sie durchgehend
    \textcquote[218]{Weinkauff2010}{ein Reservoir an Weltwissen} enthält. Eine
    Einteilung der Kinderliteratur kann einerseits anhand  verschiedener
    Gattungen, zu denen etwa das Bilderbuch, die Kinderbibel, das
    Kindersachbuch, der Abenteuerroman, die Kriminalgeschichte, der Comic oder
    das Märchen zählen, getroffen werden. Außerdem werden Kinderbücher  im
    Allgemeinen mit Altersempfehlungen versehen, die im deutschsprachigen Raum
    meistens in 2- Jahresstufen angegeben sind, bei Jugendbüchern sind die
    Abstände meist größer. \parencite[10]{Ewers2011} Geschlechterrollenmodelle
    bieten in vielen Texten, die sich an die kindliche Altersgruppe richten,
    Handlungsentwürfe an, auch in Bilderbüchern und verstärkt in
    Kinderzeitschriften kann oft ein klarer Mädchen- oder Bubenbezug hergestellt
    werden.

    Als Mädchen- oder Bubenliteratur werden die Kommunikationen bezeichnet, die
    vorwiegend von  weiblichem oder männlichem Lesepublikum angenommen werden,
    gleichzeitig  scheinen manche Genres, ebenso wie Inhalte oder
    Gestaltungsstile von Büchern, explizit unterschiedliche Vorlieben von Buben
    und Mädchen anzusprechen und zu betonen. Manchmal vermag schon der Titel
    eines Werks (wenn er etwa Vornamen der Figuren enthält), eine Gruppe für
    sich einzunehmen.

  \subsection{Historische Einblicke in Kindheit und Kinderliteratur}

    In diesem Abschnitt soll der historische Wandel des Kindheitsverständnisses
    skizziert und ein grober geschichtlicher Überblick von Problematiken und
    Strömungen in der Kinderliteratur seit ihren Anfängen bis in die heutige
    Zeit beschrieben werden.

    Die Kinderliteratur wird von den gesellschaftlichen, wirtschaftlichen und
    politischen Verhältnissen der jeweiligen Zeit geprägt: Inhalte, der
    (ästhetische) Gebrauch von Sprache, erzieherische Absichten und pädagogische
    Konzepte wie Ansichten der AutorInnen haben sich seit der Entstehung dieses
    Literaturkonzepts stark verändert.  Die zeitgenössische Auffassung von
    Kindheit, die ein individualistisches, postmodernes Menschenbild und das
    Ideal eines autoritativ-partizipativen\footnote{Der autoritativ-
    partizipative Erziehungsstil  zeichnet sich durch Wärme, Wertschätzung, dem
    Vereinbaren von Regeln und begründeter Sanktionierung aus. Das Kind kann die
    Eltern- Kindbeziehung mitgestalten, es wird zwar geleitet, lernt aber
    selbständig Verantwortung zu übernehmen. \parencite[35]{Kuttler2009}}
    Erziehungsstils verfolgt, kann mit ziemlicher Sicherheit nicht mit den
    Normen- und Wertvorstellungen anderer Epochen oder Kulturkreisen verglichen
    werden. Während die biologischen Charakteristiken des kindlichen
    Entwicklungsprozesses hier eine untergeordnete Rolle spielen, liegt der
    Fokus auf gesellschaftshistorischen Faktoren. Außerdem bleiben die
    biologischen Bedingungen beim Prozess des \emph{Erwachsenwerdens} im
    Zeitverlauf und  interkulturellen Vergleich relativ konstant, weshalb sich
    unser Interesse  auf Verhaltensanforderungen, Rollenentwürfe oder
    Förderungsangebote von  Kompetenzen richtet.

    Beim Betrachten der Darstellungen von Kindern in mittelalterlichen
    Malereien, fällt auf, dass sich diese, abgesehen von der Größe, nicht
    wesentlich von Erwachsenen abheben, weil typisch kindliche Proportionen
    schlicht unberücksichtigt blieben. \parencite[53]{Sigmund2010} Beobachtungen
    dieser Art können auf die gesellschaftlichen Verhältnisse in einem
    bestimmten historischen Kontext hinweisen: In diesem Fall kann angenommen
    werden, dass die Jahre der Kindheit im Gegensatz zum gegenwärtigen
    Verständnis von einer kostbaren, eigenständigen und unwiederbringlichen
    Lebensspanne, wenig geschätzt worden ist. Die kindliche Lebensphase wurde
    auf die Entwicklung zum erwachsenen Menschen reduziert und  kindliche
    Bedürfnisse und Interessen wurden kaum beachtet. Die Einführung der
    Schulpflicht war ein Mitgrund dafür, dass sich die kindliche Lebenswelt
    schärfer von den Erfahrungsräumen der Erwachsenen abzugrenzen begann als
    zuvor, auch durch die immer weiter fortschreitende Ersetzung von Arbeits-
    durch Lernzeit, stand einem Wandel des vorherrschenden Kinderparadigmas
    nichts mehr im Weg.

    Sozioökonomische Verhältnisse prägen  schon in früher Kindheit: Die
    Lebenswelt von ArbeiterInnenkindern mit dem Erfahrungsraum von Kindern aus
    bürgerlichen Verhältnissen mit dem gleichen Maß zu messen, entzieht sich
    jeder Sinnhaftigkeit. Ähnliche Differenzen lassen sich beim Vergleich der
    Welt von Stadtkindern und der am Land erlebten Kindheit feststellen. Die
    Erwartungen einer, an vorrangig traditionellen und religiösen Normen
    orientierten Gesellschaft, waren, gerade im Bezug auf Geschlechterrollen,
    streng:  Zukünftige Frauen und Männer hatten sich, diesen Erwartungen
    entsprechend, zu verhalten und die Handlungsspielräume waren, im Vergleich
    zu heute, deutlich geringer.

    Ende des achtzehnten Jahrhunderts entstand in England eine Literaturform,
    die speziell an Mädchen gerichtet war und diese moralisch belehren wollte,
    sie wurde später auch \emph{Backfischliteratur} genannt. Ihr Hauptziel war,
    Mädchen auf die spätere Rolle als Hausfrau, Mutter und Ehefrau
    vorzubereiten. Frauen sollten vor allem demütige und religiöse Eigenschaften
    besitzen, außerdem war das Finden eines geeigneten Ehemannes von
    entscheidender Bedeutung. Hier muss hinzugefügt werden, dass   die
    vermittelten gesellschaftlichen Ideale, wie auch die kindliche Leserschaft,
    vorrangig aus dem bürgerlichen Milieu stammte. Allerdings können
    Veränderungen in der Mädchenliteratur beobachtet werden: Während im
    traditionellen Mädchenbuch vorherrschende Rollenstereotype und traditionelle
    Wertmaßstäbe verinnerlicht werden sollen, wird im nächsten
    Entwicklungsschritt gegen diese protestiert, um dann im emanzipierten
    Mädchenbuch vor allem die Identitätsfindung zu betonen und sämtliche
    Rollenerwartungen abzulehnen.

    Etwa zur selben Zeit wie die Backfischliteratur entstand auch die
    Kolonialerzählung, die sich an Jungen richtete und das Ideal des deutschen
    Soldaten, der moralischen Überlegengenheit der europäischen Länder vertrat
    und die Notwendigkeit, insbesondere afrikanische Kulturen zu
    \emph{domestizieren} heraus strich.\footnote{Die Kolonialerzählung kann als
    eine Form der Abenteuerliteratur gesehen werden. Als bekanntes Beispiel
    können die Romane von Karl May dienen: Protagonisten reisen in die ferne,
    teilweise unentdeckte Fremde, während die deutsche Heimat stets hochgehoben
    wird.} \parencite[23]{Kaminski1989}

    In der Mitte des neunzehnten Jahrhunderts fanden auch reine Fantasie- und
    Abenteuergeschichten Eingang in die Kinderliteratur. Nach dem ersten
    Weltkrieg veränderten sich die Motive: Schon früher forderte Wolgast,
    Kinderbücher nicht für Kinder zu schreiben, sondern besonders auf die
    literarische Ästhetik Acht zu geben und den guten Geschmack nicht in dem
    Maße zu vernachlässigen. Außerdem entstanden neben bürgerlichen auch
    proletarische Geschichten, die von Klassenkampf oder sozialem Aufstieg
    handelten und deren ProtagonistInnen aus den unteren Gesellschaftsschichten
    stammten. Später entstanden nationalsozialistische Propagandatexte, die
    rassistische und sozial- darwinistische  Theorien, sowie das Ideal von
    Gehorsam und Härte verbreiteten und eine Verharmlosung des Tötens und des
    Krieges betrieben. Hinzu kommt, dass auch Kinderliteratur nicht von Zensur
    verschont blieb. In der Nachkriegszeit entstanden phantastische Erzählungen,
    die teils auf religiöse Inhalte zurückgriffen, Klassiker der
    Kinderliteratur, etwa  Romane von Karl May, Johanna Spyris \emph{Heidi} oder
    Heinrich Hoffmans \emph{Struwwelpeter} wurden neu aufgelegt, aber auch
    Bücher und Autoren, die von den Besatzungsmächten verboten wurden, weil sie
    einen Beitrag zur nationalsozialistischen  Mobilisierung leisteten, waren
    wieder verfügbar. Astrid Lindgrens \emph{Pippi Langstrumpf} kam Anfang der
    vierziger Jahre auf den Markt und stellte die Autorität der Erwachsenen in
    Frage. Erich Kästner veröffentlichte beispielsweise die inhaltlich und
    sprachlich nieveauvollen Zeitschrift \emph{Pinguin} und wirkte beim
    antifaschistischen Widerstand mit. \parencite[37]{Kaminski1989} In den
    ausgehenden sechziger Jahren wurden neue Verlage gegründet, die
    antiautoritäre Konzepte, sexuelle Aufklärung, einen reflektierten Umgang mit
    dem Nationalsozialismus und linkere Positionen vertraten. Es entstanden zwei
    Strömungen der Kinder- und Jugendliteratur, zum einen die
    sozialrealistische, wie Erzählungen von Christine Nöstlinger und zum
    anderen, die psychologisch- phantastische, wie \emph{Der Herr der Ringe} von
    Tolkien oder \emph{Die unendliche Geschichte} von Michael Ende, die sich
    mehr auf Gefühle, die sich der Gesellschaft entziehen, konzentrieren und
    beim Individuum ansetzen.  Mit der Veränderung der Mädchenliteratur entstand
    auch ein anderer Umgang mit Kolonialerzählungen oder Erfahrungen mit
    Fremdheit: Die Überlegenheit der europäischen, insbesondere der deutschen
    Kultur wurde als weniger selbstverständlich angenommen, Kolonial- und
    Kriegserfahrungen von anderer Seite beleuchtet. Außerdem schwebte das Gefühl
    vom Versagen der menschlichen Vernunft und Ethik, die Uneinschätzbarkeit der
    Folgen von neuen technischen Errungenschaften (Atomenergie) und die Gefahr
    neuer Umweltkatastrophen in der Luft.

    Im Zuge eines \emph{Pädagogisierungsschubs}, induziert durch neue
    Forschungsergebnisse wurde die Verbreitung des erzieherischen Wissens massiv
    ausgebaut und erreichte neben Eltern und Lehrern auch AutorInnen und
    Kinderzimmer. Die Studenten-, Friedens-, und Emanzipationsbewegungen hatten
    ebenfalls eine starken Einfluss auf das vorherrschende Kindheitsbild: die
    Ideale von Mündigkeit und Gleichheit setzten voraus, dass Kinder Zugang zu
    vielfältigeren Informationen erhielten und man ihnen zutraute, mit der
    gesellschaftliche Wirklichkeit umgehen zu lernen.
    \parencite[88]{Daubert2011}

  \subsection{Genres in der Kinderliteratur}

    Kinderliteratur kann anhand spezifischer Textmerkmale, Inhalte und
    Funktionen in verschiedene Genres  eingeteilt werden, zu denen etwa
    Kriminalgeschichten, Abenteuer oder Märchen zählen. Exemplarisch sollen hier
    der moderne Kinderroman und die phantastische Literatur vorgestellt werden.

    \minisec{Der moderne Kinderroman}       Der moderne Kinderroman ist in den
neunzehn siebziger Jahren entstanden und behandelt Themen, die der empirisch
erfahrbaren Wirklichkeit des jungen Lesepublikums entsprechen und verzichtet auf
phantastische Elemente, was bedeutet, dass Handlungsstränge zwar
unwahrscheinlich sein, aber trotzdem einen real möglichen Charakter besitzen
können. Wie im Entwicklungsüberblick schon erwähnt, lassen sich grob drei
Hauptströmungen unterscheiden, wobei diese sich weiter ausdifferenzieren können
und Überschneidungen möglich sind. Hannelore Daubert führt den
sozialkritischen-, den problemorientierten, den psychologischen und den
komischen beziehungsweise tragikomischen Kinder- und Jugendroman an.
\parencite[89]{Daubert2011} Sozialkritische Romane brachten den Durchbruch,
indem sie negative Thematiken, wie Krankheit, Tod, Ungerechtigkeiten, Rassismus
oder schwierige  Familienbeziehungen offen angesprochen haben. Im
psychologischen Kinderroman werden innere, subjektive Wahrnehmungen der
ProtagonistInnen in den Vordergrund gestellt, unterschiedliche Perspektiven
können in Monologen eingenommen werden und der Erzähler begegnet dem/der LeserIn
auf der gleichen Ebene. Außerdem werden Konflikte in ihrer Komplexität ernst
genommen, mögliche Lösungswege eröffnet, ohne mit einem glatten, meist
unrealistischen Happy End abzuschließen. ProtagonistInnen des (tragik-)komischen
Kinderromans berichten in selbstironischer, skeptischer Art von ihren
Erlebnissen und stellen Krisenerfahrungen, veränderte familiäre
Beziehungsverhältnisse und andere Problemen heiter und zugleich oft spöttisch
dar, wodurch eine gewisse Leichtigkeit vermittelt wird, die auch die Freude beim
Lesen anregt. Der moderne Kinderroman trägt nicht nur wegen seines hohen
sprachlichen Anspruchs, sondern auch durch die Möglichkeit der
Perpektivenübernahme und das Verstehen von Handlungsmotiven zum Ausbau von
Mitgefühl und zur Identitätsentwicklung bei.

    \minisec{Fantasy}       Die Abgrenzung der phantastischen Kinderliteratur
gestaltet sich nicht immer einfach, da  verschiedene Begriffe, wie Phantastik,
Fantasy, Fiktionalität und sogar Science Fiction, oft uneindeutig verwendet
werden. Hinzu kommt, dass auch Jugendliche und Erwachsene von diesem breiten
Genre begeistert sind. Wir halten Fantasy für die geeignetste Variante, da sie
relativ eng gefasst ist und Science Fiction, unter anderem aufgrund der
naturwissenschaftlich-technischen Erklärungsmöglichkeiten für unrealistische
Ereignisse, ausschließt. Auch Märchen werden trotz ihrer phantastischen Elemente
als eigene Gattung betrachtet.

      Nach Nikolajeva ist das Vorhandensein von zwei Welten, der primären und
      sekundären, das auszeichnende Merkmal. So können drei unterschiedliche
      Arten von Sekundärwelten unterschieden werden. \parencite[176]{Rank2011}
      Die Sekundärwelt kann eine \emph{geschlossene} Welt sein, in der es keinen
      Kontakt zur Primärwelt gibt, diese kommt auch nicht explizit vor. Eine
      \emph{offene} Sekundärwelt ist von der Primärwelt räumlich und zeitlich
      abgetrennt, beide Welten kommen im Werk vor und auch der Übergang von
      einer in die andere kann Thema sein. Ein Beispiel hierfür wäre Endes
      \emph{Die unendliche Geschichte}: Protagonist ist der Schüler Bastian, der
      bei seinem Vater lebt und Probleme mit seinen Mitschülern hat. Als er in
      einer Bücherei das geheimnisvolle Buch \emph{Die unendliche Geschichte}
      entdeckt und zu lesen beginnt, befindet er sich bald selbst in der
      Fantasiewelt \emph{Phantasien}. Die dritte Art ist die \emph{implizierte}
      Sekundärwelt. Hier kommen nur Elemente aus der  Sekundärwelt in der
      primären vor. Vampire oder Werwölfe können beispielsweise in der
      \emph{realen} Welt der ProtagonistInnen auftauchen.

      Dieses Genre kann gleichzeitig unterhaltsam wie anspruchsvoll sein und
      somit einen Beitrag zur Leseförderung und zum literarischen Lernen leisten
      und ebenso das Fiktionalitätsbewusstsein erweitern. Außerdem kann das
      Bedürfnis, in eine andere Welt einzutauchen, eher gestillt werden als
      durch realistische Literatur, Lösungen im phantastischen Raum entlasten
      den/die LeserIn.

  \subsection{Die Position des/r AutorIn}

    Es ist unmöglich, die Rolle der AutorInnen zu verallgemeinern oder auf ein
    paar wenige Eigenschaften zu reduzieren, da sie den jeweiligen historischen
    und gesellschaftlichen Einflüssen ihrer Zeit unterliegt. Außerdem bestimmen
    eine Reihe von Faktoren, wie das Selbstbild und der Erfahrungshintergrund
    des/r AutorINs, die Bevorzugung eines kinderliterarischen Anspruchs (siehe
    unten), sowie Gattung und Form den Inhalt und Schreibstil der Lektüre.
    Weiters wird eine Unterscheidung zwischen einer allwissenden Erzählerfigur
    und einer Erzählposition, die dem Publikum auf derselben Augenhöhe begegnet,
    getroffen.

    Verantwortungsbewusste ErziehungsschriftstellerInnen unterscheiden sich
    beispielsweise wesentlich von kinderliterarischen ErzählerInnen. Während
    Erstere sich hauptsächlich um die Vermittlung von moralischen und religiösen
    Werten bemühen und gleichzeitig Benimm- und Verhaltensregeln für ihre Leser
    einbauen, geht es dem/r ErzählerIn eher darum, ein Unterhaltungswerk zu
    verfassen, das zum Lesen anregen soll und dabei auch noch Spaß macht.
    Ein(e) Schriftsteller(In) kann auch in mehreren Genres aktiv sein, die
    vielleicht nicht dieselben Motivationen ansprechen  oder eindeutige
    Ansprüche verfolgen, ebenso können AutorInnen auch für Erwachsene schreiben
    oder diesen Beruf nur als Hobby verfolgen.

    Für mehr Einfachheit bei der Beschreibung werden drei Grundsatzkomplexe
    gebildet: der ästhetische-, entwicklungspsychologische- und pädagogische
    Grundsatz. Kinderliteratur enthält oft Elemente aller drei
    Anspruchskomplexe, eine strikte Trennung dieser Einteilungsbereiche ist
    nicht vorgesehen, da es sinnvoller und realistischer ist, eine persönliche
    Reihung der Ansprüche vorzunehmen  und einen der Grundsätze zu priorisieren,
    die anderen aber nicht außer Acht zu lassen.

    Natürlich sind AutorInnen nicht die Einzigen, die Einfluss auf ein Buch
    nehmen, da Verlage über die Veröffentlichung eines Werks entscheiden und
    MarketingexpertInnen die Gestaltung übernehmen.

  \subsection{Darstellungen der Hauptfiguren}

    Bei der Analyse ausgewählter Kinderliteratur der 1990er Jahre legte Anita
    Schilcher besonderen Fokus auf das Verhalten, der in den Texten vorkommenden
    Hauptfiguren, das Familiensetting und Bewertungen, die in den Texten
    vorkamen. Sie kam auf folgende Ergebnisse: Traditionelle
    Mädcheneigenschaften, wie Passivität, Empfindlichkeit, körperliche Schwäche
    oder mädchentypische, unpraktische Kleidungsvorlieben werden durchgehend
    negativ bewertet, während eine selbstbewusste, aktive, durchsetzungsstarke
    Mädchenfigur als Leitbild wirkt. Auch Jungen, die ein moderneres Rollenbild
    und Eigenschaften wie Sensibilität, Kreativität und Kommunikationsfähigkeit,
    vereinen, werden bevorzugt. Auffallend ist, dass berufstätige Mütter
    gleichzeitig Familien- und Hausarbeit leisten und eine nahezu perfekte,
    alles vereinende und deswegen vielleicht sogar unrealistische Frauenrolle
    inne haben. Väter kommen in den meisten Texten seltener vor, da
    karrierebedingte Entscheidungen, die meist zu längeren Arbeitszeiten führen,
    öfter im Vordergrund stehen. Dadurch sind sie auch deutlich weniger ins
    alltägliche Familienleben eingebunden. Weiters gehen Männer kaum in Karenz
    und sind viel seltener geringfügig beschäftigt, was den tatsächlichen
    gesellschaftlichen Verhältnissen noch immer entspricht. Frauen spielen zwar
    durch ihre Berufstätigkeit in ehemalig reinen Männerdomänen
    mit\footnote{Gerade in höheren Positionen, sowie in naturwissenschaftlich-
    technischen Gebieten, sind wenig Frauen zu finden. Diese Tätigkeitsbereiche
    sind im Allgemeinen von sehr gutem Verdienst gekennzeichnet, während soziale
    (eher weiblich dominierte) Berufe vergleichsweise unterbezahlt sind. Dass
    die unterschiedliche Verteilung von Männern und Frauen auf die einzelnen
    Berufsgruppen, nicht der einzige Grund, für geschlechtsabhängige
    Lohndifferenzen sind, sei hier nur erwähnt.}, fallen aber nach der Ankunft
    ihres ersten Kindes in traditionelle Rollenmodelle zurück und widmen ihre
    Zeit in viel höherem Ausmaß als Väter (unbezahlter) Familien- und
    Hausarbeit, weshalb sie auch Teilzeitarbeitsmodelle  erheblich häufiger in
    Anspruch nehmen. Die Vermutung, dass Frauen vielfältigere, traditionelle wie
    moderne Eigenschaften vereinen (müssen) und Männer sich in einem weniger
    breiten Spektrum bewegen, wird, in der bereits analysierten modernen
    Kinderliteratur, bestätigt.

    Passend zu postmodernen Theorien, die Individualisierung und Pluralisierung
    als Leitmotive der heutigen Gesellschaft feststellen, scheinen sich auch für
    die kindlichen ProtagonistInnen, die Bedingungen im Vergleich zu
    vorangegangenen zu ändern: Selbst in das eigene Handeln eingreifen zu können
    und eine aktive Lebensgestaltung stehen, soweit dies für das Kind möglich
    ist, im Vordergrund. Mädchen und Jungen müssen ähnliche Anforderungen
    bewältigen, wenn es darum geht ein konsistentes Selbstbild zu entwickeln.

    Allerdings bedeuten die geschlechtsspezifischen Rollenentwürfe der in der
    Literatur vorkommenden Figuren nicht, dass der/die junge LeserIn diese
    unvermittelt verinnerlichen. Sie werden natürlich (vorwiegend unbewusst)
    wahrgenommen, aber vor dem jeweiligen kindlichen Erfahrungshintergrund  in
    der Gedankenwelt konstruiert. Prädispositionen von Mädchen und Buben
    beeinflussen folglich auch die Akzeptanz oder Ablehnung eines Lesestoffs.
    Wenn Kinder also nicht gezwungen sind, sich mit einem bestimmten
    Lektüreangebot zu beschäftigen, hängt die Leseentscheidung von Belohnungen
    ab, die erwartet werden. Diese sind intrinsischer Natur und können auf
    emotionaler, sozialer oder kognitiver Ebene erfolgen: Der Wunsch, bei
    Themen, die gerade \emph{in} sind, mitreden zu können, kann die Motivation
    ein Buch zu lesen ebenso beeinflussen wie das Bedürfnis dabei die eigene
    Fantasie anzuregen und in andere Rollen zu schlüpfen, den persönlichen
    Wissensdurst zu stillen oder einfach Spaß bei dieser Form der Unterhaltung
    zu haben. \parencite[547\psq]{Kuhn2010}