\begin{abstact}
  Der Beitrag untersucht erstens den Zusammenhang zwischen dem Verhalten von Hauptfiguren in Kinderbüchern und dem Geschlecht der Kinder, die die Bücher lesen.
  Zweitens untersucht der Beitrag den Zusammenhang zwischen äußeren Merkmalen, wie Titel und Aussehen der  Bücher und dem Geschlecht der Kinder, die das Buch lesen.
  Das Geschlecht der Kinder die ein Buch lesen, wurde in einer Umfrage unter Volksschulkindern erhoben und es wurden die Hauptfiguren der 30 meistgenannten Bücher analysiert.
  Die Ergebnisse zeigen, dass es einen Zusammenhang zwischen dem Geschlecht der Lesenden und dem Verhalten der Hauptfiguren gibt.
  In Büchern die hauptsächlich von Mädchen gelesen werden handeln Hauptfiguren femininer als in Büchern die hauptsächlich von Buben gelesen werden.
  Das Verhältnis von Leserinnen zu Leser kann anhand äußerer Merkmale sehr gut erklärt werden.
\end{abstact}

\section{Einleitung} % (fold)
\label{sec:einleitung}

Bei der Untersuchungen die die Zusammenhänge von Kinderbücher und Geschlecht untersuchen gibt es im Grunde zwei Zugänge.
Die Lesesozialisationsforschung beschäftigt sich mit dem, wie Kinder lesen lernen.
Hier geht es voranging um die Unterschiede beim Lesen zwischen Mädchen und Buben.
Mädchen lesen anders und anderes, heißt es hier oft.
Der andere Ansatz beschäftigt sich mit der Darstellung von Gender, also dem vom Geschlecht abhängigen Verhalten, in Kinderbüchern.
Für diesen Ansatz ist relevant, ob sich weibliche Figuren anders als männliche Figuren verhalten.
Der zweite Ansatz untersucht eine mögliche Wirkung von Büchern auf Kinder.
Dabei geht er davon aus, dass das Verhalten weiblicher Figuren einen Einfluss auf das Verhalten von Leserinnen und das Verhalten männlicher Figuren auf das Verhalten von Lesern hat.
Folgt man der Literatur des Schreibens, sind Geschichten um Protagonisten aufgebaut.
Die Kunst ist es Hauptfiguren zu schaffen, mit der sich möglichst alle Leserinnen und Leser identifizieren um Geschichten möglichst gut funktionieren zu lassen.

Der vorliegende Beitrag kombiniert die drei Ansätze.
Wir gehen davon aus, das es durch aus plausibel ist, dass sich Leserinnen und Leser hauptsächlich mit der Hauptfigur der Bücher identifizieren, die sie lesen.
Auch wir interessieren uns für das Gender, das vom Geschlecht abhängige Verhalten, der Figuren.
Jedoch wir beziehen uns nicht auf das Geschlecht der Figuren sondern auf den Zusammenhang zwischen dem Verhalten der Hauptfiguren eines Buchs und dem Geschlecht der Kinder, die die Bücher lesen.
Wir fragen ob, sich Hauptfiguren in Büchern die hauptsächlich von Mädchen gelesen werden anders verhalten, als Hauptfiguren in Büchern die hauptsächlich von Buben gelesen werden?
Anders als der oben geschilderten Ansatz, der sich mit der Wirkungsebene von Kinderbüchern beschäftigt, ist in unserem theoretischen Modell das Leseverhalten zwischen geschaltet.



% section einleitung (end)





\section{Gender und Kinderbücher} % (fold)
\label{sec:gender}



% section gender (end)
