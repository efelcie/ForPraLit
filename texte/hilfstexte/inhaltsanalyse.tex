
\section{Inhaltsanalyse}     \emph{Lukas Kaiser}   \smallskip

  \noindent Nach den bisherigen theoretischen Ausführungen ist es Aufgabe dieses
  \Teilbereiches die methodischen Herangehensweisen rund um das Thema Kinderbuch
  \vorzustellen. Schlussendlich beschäftigen wir uns hier mit den
  \grundsätzlichen Fragen: Gibt es Mädchen- und Bubenbücher denn überhaupt und
  \wenn ja, wie unterscheiden sie sich. Der Schwerpunkt liegt hier auf der
  \zweiten Frage, da außer Zweifel steht, ob es denn Unterschiede in der
  \Lesepräferenz zwischen den Geschlechtern im Kindesalter gibt. Wer diese
  \Tatsache anzweifelt, dem sollte nahegelegt werden einen etwas größeren
  \Buchladen zu besuchen und sich in die Kinderbuchabteilung zu begeben. Die
  \Präsentiertische sind oftmals fein säuberlich, nach augenscheinlichen
  \Geschlechterpräferenzen sortiert um unentschlossenen Kunden die Auswahl für
  \die nächsten Geburtstage ihrer Neffen und Nichten so einfach wie möglich zu
  \gestalten. Regalreihen mit den Hinweistafeln \enquote{Mädchen 8--12} sind
  \wohl auch ein praktisches Beispiel, wie bestimmte Kinderbücher einer
  \Altersgruppe und vor allem einem Geschlecht zugeordnet werden. Auch von
  \wissenschaftlicher Seite ist man sich einig, dass Unterschiede vorherrschen.
  \Die Backfischliteratur -- wie Bücher die speziell für junge Frauen, die an
  \der Schwelle zum Erwachsenwerden stehen, auch genannt werden -- ist bereits
  \seit den 1970ern ein wissenschaftlich untersuchtes Phänomen. Uneinigkeit
  \herrscht jedoch um die Frage von Sinn und Unsinn einer derartigen
  \Zweigleisigkeit in der Kinderliteratur. Diese schriftliche Einführung in die
  \Methodik rund um die Forschung von Unterschieden in Kinderbüchern wird jedoch
  \keinen Beitrag leisten, um die zahlreichen Diskussionen rund um die
  \Sinnhaftigkeit einer Einteilung in geschlechterspezifische Literatur zu
  \unterstützen. Da es für die Beantwortung der oben gestellten Fragen keine
  \Relevanz hat, wird lediglich festgestellt, dass es Unterschiede gibt, um
  \diese in späterer Folge zu nennen. Die dabei wohl ergiebigste und spannendste
  \Forschungsmethode stellt die Inhaltsanalyse dar, die besonders geeignet ist
  \um schriftliche und bildliche Darstellungen vergleichen zu können. Es liegt
  \daher nahe die Vor- und Hinterseiten -- in Zukunft Buchcover genannt --
  \gesondert von den Inhalten der zahlreichen Geschichten, Abenteuer und
  \Erzählungen zu untersuchen.

  \subsection{Buchcover -- Analyse \emph{oberflächlicher} Merkmale}

    Ein Kinderbuchcover enthält üblicherweise ein farbenfrohes Design mit
    Abbildungen, die Bezug auf den Inhalt nehmen, den Namen der Autorin oder des
    Autoren, einen Titel, der womöglich mit einem Untertitel ergänzt wird,
    Empfehlungen für das Alter und eine kleine Beschreibung über den Inhalt auf
    der Rückseite. Somit steht fest, dass ein Kinderbuch bereits oberflächlich
    genügend Merkmale aufzeigt die auch verglichen werden können.

    Konzentriert man sich auf den zuerst genannten Punkt einer unterschiedlichen
    \emph{Aufmachung} fällt einer Sozialforscherin bzw. einem als Sozialforscher
    unweigerlich Erving Goffman ein, der bereits früh mit Untersuchungen von
    unterschiedlichen bildlichen Darstellungen der Geschlechter begonnen hat.
    Goffman verwendete Bilder die für den Zweck der Werbung -- in Zukunft
    Reklame genannt -- geschaffen wurden und will aufzeigen, in welchen Rollen
    die Geschlechter dargestellt werden. Seiner Interpretation nach machen immer
    wiederkehrende Konstellationen und dargestellte Situationen es möglich,
    Aussagen über Rollenerwartungen zu tätigen. Reklamebilder sind für ihn eine
    \textcquote[104]{Goffman1981}{verallgemeinernde Darstellung einer heimlichen
    Thematik der Geschlechter, vor allem des weiblichen Geschlechts.} Auch wenn
    es sich bei den Abbildungen um keine reale Situation handelt, sondern diese
    Bilder rein zum Zweck der Werbung inszeniert werden, kann festgehalten
    werden, dass die Reklame \textcquote[111]{Goffman1981}{von den Betrachtern
    als gar nicht ungewöhnlich, als etwas \enquote{ganz Natürliches} aufgefaßt
    wird.} Verknüpfen wir diese Erkenntnis mit unseren Buchcovern, so kann davon
    ausgegangen werden, dass Kinder die bildlich dargestellten
    Geschlechterrollen -- egal ob Geschlechterklischee oder nicht -- als gängig
    und konventionell interpretieren. Goffman legte zur Analyse der Reklame
    einige Indikatoren fest, wie beispielsweise die \emph{relative Größe}.
    Wollen Fotografen eine Person auf einem Bild besonders mächtig bzw.
    autoritär erscheinen lassen, so wird die Körpergröße gerne verwendet um den
    erwünschten Eindruck zu hinterlassen. Untergebene werden oftmals sitzend,
    hockend oder schlichtweg kleiner dargestellt. Dies könnte auch als Indikator
    für Kinderbuchcover übernommen werden, aber auch die restlichen von Goffman
    festgehaltenen Merkmale \enquote{weibliche Berührung}, \enquote{Rangordnung
    nach Funktion}, \enquote{Familie}, \enquote{Rituale der Unterordnung} und
    \enquote{zulässiges Ausweichen} können brauchbare Elemente für Buchdeckel-
    Analysen enthalten.

    Eine etwas andere Herangehensweise verfolgt eine Studie aus dem Jahre 1988
    von \citeauthor{Schmerl1988}. Sie kann zwar auf Grund ihres Alters nicht mit
    den neuesten Ergebnissen dienen, trotzdem werden auch Zahlen präsentiert, um
    einen gedanklichen Vergleich mit Heute entstehen zu lassen. Die Studie
    behandelt inhaltsanalytisch eine repräsentative Stichprobe von in
    Kindergärten und Vorschulen häufig gelesenen Bilderbüchern. Da Bilderbücher
    oftmals eine Kombination von Bild und Text sind, die in Bezug zu einander
    stehen, kann ein Vergleich zu einem Buchcover gezogen werden. Auch der
    Umstand, dass der bildlichen Darstellung ein besonderer Stellenwert
    zugesprochen wird, ähnelt einem Cover eines Kinderbuches sehr stark. Die
    Bücher wurden mittels einer Bestandsaufnahme von 29 Kindergärten der
    Umgebung ausgewählt, wobei nach einigen Ausschlussverfahren 52 Werke übrig
    blieben, die in späterer Folge analytisch auf 15 Inhaltskategorien
    untersucht wurden. Passenderweise wurden Texte und Bilder separat analysiert
    und ausgewertet. \parencite[133\psq]{Schmerl1988}

    \minisec{Kategorie: Geschlechterproportionen der handlungstragenden Figuren}

      Da es zumeist kein Problem darstellt, auf einem Buchcover --
      einschließlich des auf der Rückseite befindlichen Klappentextes -- die
      Hauptperson eines Kinderbuches zu identifizieren, könnte eine Analyse der
      Verhältnisse ohne große Umstände durchgeführt werden. Es konnten in dem
      Kinderbilderbücher-Sampling der Studie von \citeauthor{Schmerl1988}
      insgesamt 62 Hauptfiguren identifiziert werden, wovon jedoch nur 55 einem
      Geschlecht zugeordnet wurden. Davon wurden 17 (30,9\%) dem weiblichen und
      38 (69,1\%) dem männlichen Geschlecht zugeschrieben. Diese Proportionen
      ähneln dem allgemeinen Geschlechterverhältnis aller Charaktere sehr stark.
      Auch bei den weiteren subkategorischen Vergleichen sind ähnliche Tendenzen
      festzuhalten. So ist der Unterschied bei Kindern um einiges geringer als
      bei den Erwachsenen Figuren (Frauen zu Männern = 1:3,8; Mädchen zu Buben =
      1:1,4) und auch bei dem Verhältnis Bild zu Text zeigen die
      verschriftlichten Darstellungen eine geringere Differenz auf. Generell
      konnte bei der Analyse der Geschlechterproportionen ein eindeutiges
      Übergewicht an männlichen Figuren festgehalten werden.
      \parencite[136\psq]{Schmerl1988}

    \minisec{Kategorie: Autorengeschlecht}

      Hier sollte der Frage auf den Grund gegangen werden, ob Autorinnen ein
      ähnliches Geschlechterverhältnis wie ihre männlichen Kollegen in ihren
      Werken wiedergeben. Diese Untersuchung wurde wiederum mit allen
      vorkommenden und einem Geschlecht zurechenbaren Charakteren aber auch
      speziell für die Hauptfiguren durchgeführt. Dabei kamen folgende
      Verhältnisse zum Vorschein. (Siehe Tabelle~\ref{verh})

      %Tabelle 1       \singlespacing       \ctable[       %  cap    = ,
caption = {Verhältnis - Weibliche und männliche Figuren in Bild und Text},
label   = verh ,         pos   = h,       ]{lcccc}{         \tnote[]{Quelle:
\inparencite[146\psq]{Schmerl1988}}       }{                  \FL         &
\multicolumn{2}{c}{\small Bild}& \multicolumn{2}{c}{\small Text}      \NN
& \small Autorinnen & \small Autoren  & \small Autorinnen & \small Autoren  \NN
\cmidrule(rl){2-3}\cmidrule(rl){4-5}       weiblich zu männlich & 1:1,2  & 1:3,2
& 1:1,1   & 1:2,8   \NN       Mädchen zu Buben & 1:1,3  & 1:3,0   & 1:1,2   &
1:5,1   \NN       Fauen zu Männern & 1:1,1  & 1:3,4   & 1:1,0   & 1:2,4   \LL
}       \onehalfspacing

      Auch wenn die Unterschiede zwischen bildlichen und schriftlichen
      Darstellungen etwas variieren, konnte einheitlich festgehalten werden,
      dass Autorinnen um ein ausgewogenes Geschlechterverhältnis bemüht waren,
      während ihre männlichen Kollegen in ihren Charakteren ein extremes
      Ungleichgewicht bei den Geschlechtern vorwiesen. Diese Erkenntnis macht es
      notwendig den Autor bzw. die Autorin bei einer Analyse eines Kinderbuches
      festzuhalten und die Inhalte ihrer Werke auf den Gebrauch von
      Geschlechterrollen zu untersuchen.

      Ergänzend sollen hier noch die Zahlen für die handlungstragenden Personen
      genannt werden, da bei einer Analyse der Buchcovers lediglich diese
      ausgewertet werden können:  Geschlechterverhältnis der Hauptcharaktere
      (weiblich zu männlich) bei Autorinnen: 1:1,2. Geschlechterverhältnis der
      Hauptcharaktere (weiblich zu männlich) bei Autoren: 1:5,8.
      \parencite[146\psq]{Schmerl1988}

  \subsection{Buchinhalte -- Analyse inhaltlicher Merkmale}

    Betreten wir nun das Terrain einer inhaltsanalytischen Untersuchung von
    Geschlechterrollen in Kinderbüchern, so sollte eines vorne weg fest stehen:
    Mädchen lesen andere Bücher als Jungen. Diese Bücher müssen zuerst ermittelt
    werden, wenn eine aussagekräftige Analyse das Ergebnis sein soll. Oftmals
    vollzogen, von sozialwissenschaftlicher Sicht jedoch zweifelhaft, sind
    Inhaltsanalysen von Kinder- und Jugendbüchern die kaum gelesen werden. Gerne
    werden Kinderbücher von AutorenInnen verwendet die dafür bekannt sind, in
    der Darstellung ihrer Charaktere Klischees oder auch besonders Anti-
    Klischees zu erfüllen. Aussagen aus derartigen Analysen über
    gesellschaftlichen Wandel zu tätigen, erscheint oft willkürlich. Doch selbst
    wenn wir wissen welche Bücher viel gelesen werden und dadurch viel
    wahrscheinlicher die Entwicklung und Erziehung von Kindern beeinflussen
    bleibt bei einer inhaltsanalytischen Untersuchung die Auswahl der Kriterien
    -- aufgrund derer sie verglichen werden -- abhängig von der Intention des
    Forschers. Oftmals wird nur an der Oberfläche gekratzt, wie etwa bei reinen
    Zählverfahren von weiblichen und männlichen Charakteren. Die Auswertung von
    Geschlechterverhältnissen kann nur wenig über klischeehaftes Verwenden von
    Geschlechterrollen aussagen. Besonders in der neueren Kinderliteratur werden
    alte Geschlechterklischees in der Darstellung immer weniger verwendet.
    \textcquote[488]{Haller2004}{Der Bruch mit traditionellen
    geschlechtsspezifischen Rollenbildern zählt, zumindest in anspruchsvolleren
    Texten, mittlerweile zum Standard. Es dominieren selbstbewusste, witzige und
    aufmüpfige Mädchen, oft begleitet von sensiblen, nachdenklichen Jungs
    \textelp{}}. Und selbst wenn all diese Veränderungen mitberücksichtigt
    werden, ist nicht vollkommen klar ob sich Mädchen nur mit den weiblichen
    Protagonisten und Charakteren identifizieren und Buben ausschließlich von
    männlichen Erscheinungen Anleihen für ihre Entwicklung entnehmen. Es bleibt
    also die Frage offen, ob womöglich Mädchen auch das Verhalten eines
    männlichen Hauptprotagonisten in ihren Lieblingsbüchern zumindest teilweise
    übernehmen, kopieren oder sich von ihnen in ihrer Entwicklung inspirieren
    lassen. Aufgabe dieses Kapitels soll daher sein, in den folgenden
    Unterkapiteln einige Herangehensweisen vorzustellen. Beginnen wird eine
    klassische Auszählung von Geschlechterverhältnissen und ihrer Darstellung,
    gefolgt von einer moderneren Verwendung der Kinderbuchanalyse inspiriert von
    zwei jungen Arbeiten, die sich dadurch auszeichnen, entweder Rücksicht auf
    die Verwendung von Klischees genommen zu haben oder die Gewichtung von
    Handlungssträngen hervorzuheben. Als Drittes und letztes Unterkapitel wird
    der Ansatz einer inhaltsanalytischen Untersuchung auf Basis der Hauptfiguren
    erläutert, der jedoch ohne Unterstützung einer Beispielsarbeit stattfinden
    muss, da ein derartiger Ansatz noch nicht verfolgt wurde.

    \subsubsection{Klassiche Analyse}

      Als klassisches Beispiel einer Inhaltsanalyse könnte die bereits oben
      beschriebene Studie von \citeauthor{Schmerl1988} genannt werden. Kurz
      zusammengefasst wurden Kindergartenbücher auf die Verwendung von
      Geschlechterrollen untersucht und in Verhältnissen wiedergegeben. Es
      wurden hier jedoch nicht nur die bereits beschriebenen Kategorien
      \emph{Geschlechterproportionen der handlungstragenden Figuren} und
      \emph{Autorengeschlecht} erhoben, die wohl zu den klassischsten
      Erhebungskategorien im Forschungsfeld Kinderliteratur zählen. Es sind hier
      auch etwas weitreichendere und interessantere Kategorien möglich. So
      erheben \citeauthor{Schmerl1988} beispielsweise die unterschiedliche
      Darstellung von Männern und Frauen bei Beruf, Familienarbeit und
      Kommunikation, aber auch Differenzen bei Emotionalität, Bedürfnissen und
      Verhalten (aggressiv, passiv und/oder altruistisch).
      \parencite{Schmerl1988} Beispielhaft sollen hier noch zwei Kategorien
      (aktive Tätigkeiten und Eigenschaften) auf ihre Vorgehensweise und ihre
      Ergebnisse vorgestellt werden.

      \minisec{Kategorien: Aktive Tätigkeiten}

        \textcquote[139]{Schmerl1988}{Unter dieser Kategorie wurden alle Arten
        \von aktiver Umweltauseinandersetzung oder -- Bewältigung erfasst.}
        \Dabei wurde zwischen „körperlichen“ und „geistigen“ Aktivitäten
        \unterschieden. Unter körperlichen Aktivitäten wurden etwa
        \Beschäftigungen wie suchen, sammeln, essen, ernten, angeln, tanzen,
        \sich verstecken, fotografieren etc. verstanden. Getrennt davon wurde
        \denken, überlegen, träumen, sich erinnern, Ideen haben, etwas wissen,
        \beschließen etc. als geistige Aktivität interpretiert und festgehalten.
        \Die Proportionen waren besonders bei der ersten Unterkategorie – also
        \den körperlichen Betätigungen -- klar verteilt. Männliche Figuren
        \wurden im Durchschnitt dreimal so oft wie weibliche Figuren bei
        \derartigen Aktivitäten dargestellt. Wobei das Buben-Mädchen-Verhältnis
        \um einiges geringer ausfällt. Dies bedeutet jedoch auch, dass der
        \Durchschnittswert vor allem von den erwachsenen Darstellungen gehoben
        \wird. Männer werden somit sogar vier bis fünfmal (variiert in Bild und
        \Text) sooft bei körperlich-aktiven Tätigkeiten dargestellt wie Frauen.
        \Bei geistigen Tätigkeiten fallen diese Zahlen etwas ausgeglichener aus.
        \Durchschnittlich wurde ein Verhältnis von 1:2,3 (weiblich zu männlich)
        \festgehalten. \parencite[139]{Schmerl1988}
  
      \minisec{Kategorie: Eigenschaften}

        Eigenschaften lassen sich am besten mittels Personenbeschreibungen
        ermitteln. Hierbei wurden äußere von inneren Eigenschaften
        unterschieden. Unter äußeren Eigenschaften wurden körperliche
        Beschreibungen verstanden, so wurden Mädchen als klein, zierlich,
        anmutig und/oder leichtfüßig beschrieben und Frauen als schön, reizend,
        jung und/oder alt. Das männliche Geschlecht wurde äußerlich nur sehr
        wenig beschrieben, etwa als alt oder stark. Bei Jungen fehlte die
        Beschreibung äußerer Eigenschaften sogar gänzlich. Bei den inneren
        Eigenschaften, also Beschreibungen von Charakterzügen, wurde zwischen
        positiven und negativen Eigenschaften unterschieden. Männliche Figuren
        steigen dabei um einiges besser aus als weibliche Darstellungen. So
        liegt das Verhältnis bei den positiven Beschreibungen (wie etwa nett,
        freundlich, wachsam, munter, klug, ausgeglichen usw.), bei 1:2,5
        (weiblich zu männlich). Bei der Analyse der negativen Eigenschaften
        wurden die Frauen besonders stark benachteilig. Weibliche Figuren wurden
        als dumm, zerstreut, nervös, empfindlich, schüchtern, verträumt, streng,
        usw. beschrieben. \textcquote[145]{Schmerl1988}{Das absolute wie
        relative Überwiegen negativer Eigenschaftsbeschreibungen weiblicher
        Figuren ist vor dem Hintergrund der sonstigen durchgehenden
        Unterrepräsentierung von Mädchen und Frauen als besonders
        diskriminierende Konstellation zu bewerten.}
        %\parencite[144\psq]{Schmerl1988}

    \subsubsection{Moderne Analyse}

      Betritt man das Feld der qualitativen Inhaltsanalyse von Kinderbüchern,
      fällt auf, dass die großen Studien renommierter SozialwissenschaftlerInnen
      fehlen. Es scheint sich jedoch dabei um ein beliebtes Forschungsgebiet für
      junge, angehende und ambitionierte WissenschafterInnen zu handeln, die
      jedoch zumeist in den pädagogischen Instituten beheimatet sind. Zumeist
      verbindet diese Arbeiten eine spürbare Motivation, zugleich aber auch eine
      kritisierbare Herangehensweise.

      \minisec{Analyse der Geschlechterklischees}

        Stellvertretend für derartige junge Arbeiten könnte die Bachelorarbeit
        von Marika Grigat von der Hochschule Neubrandenburg genannt werden, die
        im Folgenden kurz vorgestellt wird. Inspiriert von Tim Rohrmanns
        „Checklist für Kinderbücher“ \parencite{Rohrmann2012} versucht sie
        qualitative Unterschiede von Darstellungen der Geschlechter in der
        Lektüre für Kinder zu finden. Obwohl ihre Herangehensweise, nach dem
        Zufallsprinzip die Bücher aus dem Regal eines Buchladens zu ziehen,
        etwas kritisch betrachtet werden darf, sind die Ergebnisse und
        Interpretationen aufschlussreich und interessant. Besonders ihre
        Erkenntnis, dass erwachsene Figuren im Vergleich zu den kindlichen
        Darstellungen klischeereicher dargestellt werden, zeigt auf, dass das
        reine Zählen vom Vorkommen der Geschlechter nur wenig Aussagekraft
        aufweist. Ihr Auswahlverfahren wird jedoch durch eine weitere Erkenntnis
        noch kritischer betrachtet, wenn sie feststellt, dass zwar in neuerer
        Kinderliteratur ein Trend zur klischeelosen bzw. -armen Darstellung von
        Geschlechtern tendiert wird, das Alter bzw. Erscheinungsjahr des Buches
        aber keine Garantie darstellt. Umso wichtiger wird der Umstand zuerst zu
        erfahren, welche Bücher heute tatsächlich von den Kindern gelesen
        werden. \parencite[22\psq]{Grigat2009}

      \minisec{Analyse der Handlungsstränge}

        Als zweite Beispielarbeit dient die Diplomarbeit von Carl Pick von 2009,
        die sich mit der seriellen Narration beschäftigt und dabei die Bedeutung
        von unterschiedlichen Handlungssträngen beschreibt. Es gibt zwei Arten
        von Handlungssträngen, die am besten mittels ihrer Verwendung in
        seriellen Narrationen erklärt werden. Unter seriellen Narrationen gibt
        es fünf Typen: Serie, Reihe, Fortsetzungsroman, Zyklen und Mehrteiler.
        \parencite[12--18]{Pick2009} Oftmals synonym verwendet sind sie
        medienwissenschaftlich von einander zu unterscheiden. Eines der
        Hauptmerkmale um sie voneinander trennen zu können, ist die
        unterschiedliche Verwendung der Handlungsstränge. Es gibt zwei Arten von
        Handlungssträngen. \parencite[23\psq]{Pick2009}

        \pagebreak % um Aufzählung nicht zu zerreißen.         \begin{itemize}
\item Übergeordnete Handlungsstränge           \item Untergeordnete
Handlungsstränge         \end{itemize}

        Übergeordnete Handlungsstränge werden oftmals auch als Hauptkonflikte
        Üoder Hauptziel bezeichnet. Als Beispiel eines solchen übergeordneten
        ÜHandlungsstranges könnte hier etwa, Pinocchios Wunsch zu seinem Vater
        Üzurück zu finden, verwendet werden. Untergeordnete Handlungsstränge
        Üwären somit der Überbegriff für sämtliche Abenteuer, Erlebnisse die ein
        ÜProtagonist durchleben muss um sein Hauptziel zu erreichen oder den
        ÜKonflikt zu lösen. Um bei demselben literarischen Exemplar zu bleiben,
        Ükönnte hier Pinocchios Flucht aus dem Spielzeugland als Beispiel
        Üdienen. Untergeordnete Handlungsstränge werden oft auch folgenimmanente
        ÜHandlungsstränge genannt. Sie müssen aber nicht immer auf genau ein
        ÜBuch oder Kapitel einer seriellen Erzählung bezogen sein. Es kommt auch
        Üvor, dass untergeordnete Handlungsstränge über mehrere Teile
        Übeschrieben werden oder auch mehrere solcher Handlungsstränge in einem
        ÜTeilwerk vorkommen. \parencite[23\psq]{Pick2009}
        
        Diese Unterteilung in übergeordnete und folgenimmanente (bzw.
        untergeordnete) Handlungsstränge, kann vor allem bei seriellen
        Erzählungen verwendet werden und birgt daher für die inhaltliche Analyse
        von Kinderbüchern eine große Chance, da viele von Kindern gern gelesene
        Bücher Serien, Reihen, etc. sind. Versucht und damit geklärt, gehört vor
        allem auch, ob diese Unterteilung auch für Einzelwerke verwendet werden
        kann, da auch hier vermutet wird, dass es Haupt- und Nebenstränge gibt.
        
        Geschlechter inhaltsanalytisch mit dieser Unterscheidung zu untersuchen,
        kann zeigen ob ein Geschlecht verstärkt von tragender Rolle ist oder ob
        die Charaktere eines Geschlechtes verstärkt bei untergeordneten
        Handlungssträngen zu finden sind. Um jedoch eine Aussage über die
        Tragweite einer unterschiedlichen Ausprägung von weiblichen oder
        männlichen Charakteren in Haupt- und Nebensträngen erzielen zu können,
        muss erklärt werden, ob Leser sich mit dem Hauptprotagonisten oder mit
        Charakteren des eigenen Geschlechtes identifizieren. Um die Annahme zu
        berücksichtigen, dass ein Leser sich vorranging mit dem
        Hauptprotagonisten identifiziert, auch wenn dieser womöglich nicht dem
        eigenen Geschlecht angehört, sollen zusätzlich die Handlungsstränge
        zwischen Mädchen- und Bubenbüchern verglichen werden. Etwaige
        Unterschiede beim Aufbau, Ablauf, Inhalt und/oder Ausgang der
        Handlungsstränge könnten somit Aufschluss über die Bevorteilung eines
        Geschlechts zeigen.

    \subsubsection{Denkanstoß zur Analyse auf Basis der Protagonisten}

      Bisherige Inhaltsanalysen von Kinderbüchern weisen eine Gemeinsamkeit auf.
      Es wird von vorne weg angenommen, dass Kinder sich lediglich mit den
      Charakteren ihres eigenen Geschlechtes identifizieren. Verändern sich die
      Darstellung eines Geschlechtes im Laufe der Geschichte, wird dies all zu
      gern verwendet um Aussagen über Veränderungen in der Gesellschaft zu
      formulieren. Es kommt jedoch durchaus vor, dass Kinder Lieblingsbücher
      haben, deren Protagonisten dem anderen Geschlecht zugehörig sind. Dies
      wäre auch weder ungewöhnlich oder von weiterem Belang, wenn man nicht
      wüsste, dass Mädchen andere Bücher lesen als Jungen. Vermuten wir, dass
      männliche Protagonisten von Büchern, die bevorzugt von Mädchen gelesen
      werden öfter, einen \emph{Anti-Klischee-Anstrich} -- sprich sehr
      gefühlvoll, kommunikativ, zärtlich, etc. -- verpasst bekommen als jene in
      der von Buben bevorzugten Lektüre, so könnte dies dazu führen, dass
      Mädchen und Buben unterschiedliche Rollenbilder eines \emph{wahren} Mannes
      entwickeln. Dies gilt natürlich auch umgekehrt. Da jedoch nicht bekannt
      ist, ob die Darstellungen des gleichen Geschlechts bei den beiden
      Lesegruppen -- also Buben und Mädchen -- tatsächlich Unterschiede
      aufweisen, kann hier nur gemutmaßt werden. Es handelt sich hierbei um eine
      große Lücke in der Forschung rund um den Gegenstand der
      geschlechterspezifischen Untersuchung von Kinderliteratur. Unterlässt man
      es, die Protagonisten passend zur dazugehörigen Leserschaft zu
      untersuchen, besteht die Gefahr einer verfälschten Aussage.
      Zusammengefasst auf einen Satz bedeutet das, dass die oftmalige
      Feststellung, dass die Charaktere von Jungen und Mädchen in Büchern immer
      klischeeärmer oder emanzipierter dargestellt werden, ihre Aussagekraft als
      Spiegel der Gesellschaft verlieren, wenn sie von fast keinem oder nur von
      einem Geschlecht gelesen werden.

      Denkbar wäre auch ein Vergleich der Protagonisten von Mädchen- und
      Bubenliteratur, ohne auf das Geschlecht des Hauptcharakters zu achten.
      Dies würde bedeuten, dass die Protagonisten -- egal ob Mädchen oder Junge
      -- auf ihre Eigenschaften und Wesenszüge analysiert werden. Spannend wären
      etwa Ergebnisse bezüglich der Verwendung von Klischees. So könnte etwa
      vermutet werden, dass Mädchenbücher klischeeärmer formuliert werden als
      jene Bücher die bevorzugt von Buben gelesen werden.

  \subsection{Resümee}

    Die Inhaltsanalyse bietet zahlreiche Möglichkeiten im Feld der Kinderbücher
    zu aussagekräftigen Erkenntnissen zu kommen. Oft wurde diese Möglichkeit
    jedoch nicht genutzt und nur an der Oberfläche der Wahrheit gekratzt. Fest
    steht dennoch, dass sich Kinderbücher besonders gut dazu eignen,
    Geschlechterrollen aufzuzeigen und weiter zu vermitteln. Wie wir
    festgestellt haben wurden weibliche, im Vergleich zu männlichen
    Beschreibungen oft benachteiligt und unwahr präsentiert. Auch konnte gezeigt
    werden, dass Autorinnen sich viel mehr um eine ausgewogene Präsenz von
    weiblichen zu männlichen Figuren bemühten als ihre männlichen
    Berufskollegen. Zusammenfassend darf gesagt werden, dass Unterschiede
    festgestellt wurden und in der moderneren Kinderliteratur auffallend oft
    versucht wird, diese Klischees nicht länger zu verwenden. Der Schlüssel zu
    einer aussagekräftigen Analyse liegt wohl auch in der Offenheit der
    Forscher, neue Wege zu gehen. Einer dieser möglichen Ansätze wäre eine
    stärkere Gewichtung der HauptprotagonistenInnen um sich nicht immer nur rein
    auf die Geschlechter zu konzentrieren.