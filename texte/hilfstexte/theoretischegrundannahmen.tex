
\section{Theoretische Grundannahmen}   \emph{Peter Flucher}   \smallskip

  \noindent Mädchen und Buben verhalten sich schon früh anders, aber woher kommt
das? Wir untersuchen welchen Einfluss Kinderbücher auf das
geschlechtsspezifische Verhalten von Kindern haben können. Wir gehen davon aus,
dass sich Kinder mit den Hauptfiguren der Bücher, die sie lesen, identifizieren
und somit von ihnen Verhalten übernehmen.    Wenn sich das Verhalten der
Hauptfiguren der Bücher, die Buben lesen, von dem der Bücher die Mädchen lesen,
unterscheiden, könnte das einen Beitrag zur Erklärung der Unterschiede im
Verhalten von Mädchen und Buben darstellen.

   


  \subsection{Das Buch als Akteur}

  %Netzwerke   In einer Face-to-Face Situation bringt eine Person eine andere
zum Handeln.    Wir haben zwei echte Menschen, die sich beeinflussen.   Doch
viele Situationen der gegenseitigen Beeinflussung sind heutzutage keine Face-to-
Face-Situationen mehr.   Wir werden von Personen in Werbefilmen beeinflusst.
Doch wer beeinflusst hier? Das Team, das den Film gedreht hat? Die Firma, die
den Spot bezahlt? Oder die fiktive Person, die uns etwas über ein Produkt
erzählt?   Oder wenn wir mit Otello in Franco Zeffirellis Film
\citefilm{zeffirelli1986} mitleiden. Wer bringt uns zum Weinen? Ist es
Shakespeare, Verdi, Zeffirelli oder Domingo?   Aber auch in Situationen, die auf
den ersten Blick klarer scheinen, wie dieser Text, den Sie gerade lesen,
offenbart ähnliche Probleme.   Ist es für Sie als Leserin oder Leser relevant
wer diesen Text geschrieben hat? Wen stellen Sie, sich vor wenn Sie den Text
lesen? Eine Frau oder einen Mann? Oder spricht zu Ihnen einfach ein Text?   Oder
wen stellen wir uns als Adressaten vor? %Glauben Sie wirklich, wir denken die
ganze Zeit an Sie?   Wenn wir gegenseitige Beeinflussungen analysieren wollen,
die keine Face-to-Face-Situationen sind, entsteht, wie oben gezeigt, die
Problematik, dass Sender und Empfänger einer Kommunikation nicht mehr klar
zugeordnet werden können. Für gewöhnlich wird dieser Problematik nicht viel
Aufmerksamkeit geschenkt. Die Situation wird in etwas, das wie eine Face-to-
Face-Situation behandelt werden kann, transformiert. Als Sender wird dann
einfach die Regisseurin oder der erstgenannte Autor verwendet und mit dem
Empfänger wird ähnlich verfahren.   Will man jedoch die Komplexität der
Situation mit in die Analyse einbeziehen, kommt man so nicht weiter.
\parencite[252\psq]{Johnson2006}    % Problematisch   Eine Möglichkeit um solche
Komplexitäten darzustellen sind Akteur-Netzwerke. Akteur-Netzwerke sind
Beschreibungen von Situationen, die aus vielen kleinen Beeinflussungen bestehen.
Die Punkte an denen sich mindestens drei Beeinflussungen \emph{treffen}, also
der Werbespot, Otello oder dieser Text hier, nennen wir \emph{Akteure}.
Akteure deshalb, weil sie etwas tun. Sie beeinflussen, sie bringen zum Weinen
oder Nachdenken.   Um zu betonen, dass kein Akteur ohne andere Akteure
existieren kann, sprechen wir von Akteur-Netzwerken. \parencite[82]{Latour2010}
Ohne Team, Auftraggeberin und Adressaten wäre der Spot kein Spot. So verknüpft
er verschiedene Dinge zu einem Akteur Netzwerk.

  %\blockcquote[22]{Law2011}{STS \textins{Science, Technology and Society
  %\P.\,F.} behauptet \textelp{}, dass Worte oder Texte nicht nur beschreiben,
  %\sondern in den Beschreibungen etwas \emph{tun}: Sie sind performativ; STS
  %\sagt uns, dass Worte oder Texte dabei helfen, das, was sie beschreiben, in
  %\die Welt zu bringen und zu festigen.}


  %Buch als Akteur   Das Netzwerk, an dem wir hier gerade arbeiten, verknüpft
außer Leserschaft und Schreibende mit all der notwendigen Technik, auch noch
Kinder, Geschlechterunterschiede und --- Bücher. All diese Akteure sind wiederum
Akteur-Netzwerke. Man kann sich das wie \emph{Wikipedia} vorstellen. Durch den
Klick auf einen Link auf Wikipedia springt man von einem Netzwerk zum nächsten.
Jede Definition ist wieder ein Netzwerk, eine Versammlung oder Verknüpfung von
neuen Definitionen. Auch Bücher verknüpfen. Bücher verknüpfen eine große Anzahl
an Menschen, die Leserschaft, die Autorin oder den Autor, und verschiedenste
Inhalte, Theorien oder vielleicht Einstellungen.   Das Besondere an Akteur-
Netzwerken, wie Büchern, die keine Menschen sind, ist, dass sie ihre
\emph{Arbeit}, wenn sie einmal da sind, mit viel weniger Aufwand als menschliche
Akteur-Netzwerke verrichten.   Ein gutes Beispiel dafür ist der Hirte, der mit
viel Aufwand seine Herde hütet und der Weidezaun, der, ist er einmal gebaut,
dieselbe Arbeit allein durch seine Existenz verrichtet.   In unserer Welt gibt
es viele Akteure, die ihre Arbeit verrichten, ohne dass wir die Arbeit als
solche wahrnehmen.   Diese Arbeit, auf die man sich verlassen kann, wie auf das
Wasser, dass das Mühlrad antreibt, erscheint uns als \emph{Stabilität}.   Diese
Stabilität ist für uns schon so gewöhnlich geworden, dass sie so natürlich
scheint, wie die Lünneburger Heide.   Dieser Umstand verdeckt, dass die
Stabilität, das durch stetigen Aufwand Produzierte ist.   Veränderung ist
demnach nicht das zu Erklärende, sondern die Stabilität bzw. Ordnung, die von
Akteuren aufrecht erhalten wird.


  Will man die \emph{Mächtigkeit} eines Akteur-Netzwerkes definieren, so könnte
  man sagen, umso mehr Akteure durch ein Akteur-Netzwerk verknüpft werden umso
  mächtiger ist es. Bücher haben die Fähigkeit unzählige Akteure miteinander zu
  riesigen Akteur-Netzwerken zu verbinden. Von der Bibel wurden \zB geschätzte 2
  bis 3 Milliarden Exemplare unters Volk gebracht. Sie verknüpft seit rund 2000
  Jahren verlässlich Menschen und Werte auf der ganzen Welt. %Ein mächtiger
  Akteur Nicht nur bei der Bibel sehen wir, dass das Buch nicht nur verknüpft,
  sondern auch differenziert. Wer dieselben Bücher liest, gehört zusammen und
  grenzt sich so, von denen die es nicht tun, ab. Differenzen wie
  Kind/Erwachsener oder der Zugehörigkeit zu einer Nation, werden mit
  differenziertem Leseverhalten in Verbindung gebracht.  \parencites[Kap.\,3
  in][]{Postman2011}[50]{McLuhan2012} Wir begeben uns in dieser Arbeit auf die
  Suche nach Hinweisen, wie das Buch Unterschiede zwischen Mädchen und Buben
  aufrecht erhalten kann. Bei \inparencite{Postman2011} heißt es, dass dadurch,
  dass das Wissen, das Kindern durch Bücher zugänglich (und nicht zugänglich)
  gemacht wird, die Kindheit überhaupt erst erzeugt wird. Erst durch die
  gezielte Auswahl und Herstellung von Kinderbüchern, die gewisse Aspekte des
  Lebens zeigen und andere ausblenden entsteht Kindheit. Kindheit ist somit ein
  geschützter Raum ohne Krankheit, Sexualität und Tod. Folgt man der Spur der
  Kinderbücher ins 15. Jahrhundert, findet man andere wichtige Bücher, die die
  Regeln, für die später entstehenden Kinderbücher definierten.
  \hyphenblockcquote{english}[loc.\,963]{Postman2011}{Locke, for example,
  exerted enormous influence on childhood's growth through his remarkable book
  \emph{Some Thoughts Concerning Education}, published in 1693} Durch die
  stetige Konfrontation der jungen Menschen der damaligen Zeit mit speziellen
  Büchern wurden sie zu Kindern. Auch die Entstehung von Nationen wird mit
  Büchern oder in dem Fall auch Zeitungen in Verbindung gebracht. In diesem Fall
  sind es jedoch keine Erziehungsratgeber, die den unterschiedlichen Raum
  bilden, sondern Sprachen und Distributionswege sind für die gleichbleibenden
  Inhalte verantwortlich. \parencite[39]{Anderson2006}


  %Cultural Studies   Wechseln wir in eine Zeit, in der Geschichten noch nicht
in Büchern aufgeschrieben wurden, sehen wir den Zusammenhang von Buch und der
Herstellung von Stabilität noch deutlicher.   Zu Homers Zeiten im alten
Griechenland, vor rund 3000 Jahren, gab es noch keine Bücher. Hier war das
Herstellen einer gemeinsamen Kultur noch ein großer Aufwand. Sänger zogen von
Stadt zu Stadt und maßen sich in Wettstreiten. Es ging darum, dass die
Performance möglichst der Performance des letzten Auftritts glich. Es war ein
Wettstreit der Kontinuität. Das Schwierigste, in einer Zeit ohne Bücher, war die
Unveränderlichkeit. Mit der Entwicklung des Schreibens wurde ein Teil, der
Inhalt, aus der Performance gelöst. Das Konstrukt des Inhalts bekam schnell
einen \emph{stabilen Charakter}. Anders als die Performance, scheint der Inhalt
sich nicht zu verändern. Diese Übersetzung von Bewegung in die materielle Form
eines Textes ermöglichte es den Inhalt auf seine Substanz hin zu untersuchen. Er
ist etwas Gemachtes. Die Möglichkeit den Inhalt einer Performance jetzt über
Zeit und Raum stabil transportieren zu können, verändert auch den Zugang zu den
Geschichten. Der Inhalt wurde beurteilt und vergleichbar. Die Analyse von
Büchern war eine Analyse von Inhalten. Auch bei der Wirkung von Büchern
konzentrierte man sich anfangs nur auf den Inhalt. Das Buch war ein
Transportgefäß, wie die biblische Bundeslade.

  Als in der Mitte des 20. Jahrhunderts neue Medien entstanden, und Alternativen
  zum Buch aufkamen, wurde das Buch erstmals auch aus einer gewissen Distanz
  wahrgenommen.  \Citeauthor{McLuhan2012} stellte fest, das der Inhalt von
  Medien von der Form der Medien abhängig ist. Er plädiert dafür, dass sich die
  Medienwissenschaft mehr mit der Form der Medien beschäftigt.


  %Er zeigt im Grunde, dass die Trennung von Performance und Text oder wie es
  %bei ihm heißt, von \emph{medium} und \emph{message}, nicht konsequent machbar
  %ist. Inhalt und Form sind von einander Abhängig und die Wissenschaft solle
  %sich mehr auf das Medium an sich konzentrieren.

  Ein anderer Kritikpunkt an der Forschung zu Massenmedien kommt von
  \inparencite{Hall1980}.
  \foreignblockcquote{english}[117]{Hall1980}{Traditionally, mass-communications
  research has conceptualised the process of communication in terms of a
  circulation circuit or loop. This model has been criticised for ists linearity
  ---sender/message/reciever---for its concentration on the level of message
  exchange and for the absence of a structured conception of the different
  moments as a complex structure of relations. But it is also possible (and
  useful) to think of this process in terms of a structure produced an sustained
  through the articulation of linked but distinctive moments---production,
  circulation, distribution, consumption, reproduction. This would be to think
  of the process as a \enquote{complex structure in dominance}, sustained
  through the articulation of connected practices, each of which, however,
  retains its distinctiveness and has its own specific modality, its own forms
  and conditions of existence.} Die Cultural Studies sind ein transdisziplinärer
  Ansatz, der darauf aus ist den Menschen die Kontrolle über die Macht und die
  Strukturen, die ihr Leben bestimmen, zurück zu geben. \parencite[2]{Hipfl2004}
  Niemand kann Texten eine fixe Bedeutung zuschreiben. Jedoch man kann die
  Wahrscheinlichkeit erhöhen, dass ein Film auf eine gewisse Weise gedeutet
  wird. Die Cultural Studies wollen den Einfluss der Leserschaft, oder zumindest
  ihr Bewusstsein für den Vorgang der individuellen Deutung stärken.
  \parencite[1:15\,Min.]{Hall2010} Sie wollen diese Machtfaktoren und ihr Wirken
  analysieren. \parencite[29]{Daehnke2003}
  \hyphentextcquote{english}[119]{Winter2004}{For example, a semiotic analysis
  of a Hollywood film \textelp{} with no mention of the relation between culture
  and power do not belong to cultural studies.}

  Beide Ansätze haben gemeinsam, dass sie die Konzentration auf den Inhalt als
  Analyseobjekt kritisieren. \citeauthor{McLuhan2012} lenkt das Interesse auf
  die Einflüsse abseits des Inhalts. \hyphenquote{english}{The Medium is the
  Message.} Er stellt aber nicht die Frage nach Machtverhältnissen. Er stellt
  einfach fest, welche Veränderungen welche Konsequenzen haben. Für die Cultural
  Studies sind gerade die Machtverhältnisse wesentlich. Ziel ist die Beseitigung
  von Ungerechtigkeit. Uns interessiert die Rolle, die Bücher bei der Produktion
  von Geschlechterunterschieden spielen. Dabei ist auch für uns der Inhalt, in
  seiner gewohnt analysierten Form, nicht Gegenstand unserer Interesse. Uns geht
  es um die Sichtbarmachung des Aufwands der Produktion der Unterschiede. Es
  geht, mehr oder weniger, um eine theoretische Umkehr der \emph{Textwerdung}
  aus der Performance. Wir wollen die Performance im Text wieder sichtbar
  machen. Um diese Aktivität sichtbar zu machen, fragen wir, was Bücher machen
  und warum sie es machen. Wir behandeln Bücher als Akteur-Netzwerke.

  Bevor wir fragen was Bücher eigentlich machen, überlegen wir, warum sie es
  machen. Wir haben jetzt schon einiges über Funktionen von Büchern als
  Produzenten von Ordnungen gehört, jedoch wird es, außer in der Werbung, wohl
  nur wenige Autorinnen oder Autoren geben, die sagen würden, dass sie an
  Büchern schreiben um Ordnung zu erzeugen. Warum sie genau schreiben oder was
  sie damit wollen weicht vielleicht oft von einander ab, jedoch haben sie alle
  ein Ziel --- sie wollen ihre Leserschaft erreichen. Sie wollen einen Draht zu
  ihrer Leserschaft bekommen. Um seine Leserschaft zu erreichen, muss man sie
  kennen, sie verstehen. Es kommt nicht von ungefähr, dass
  \inparencite[13]{Mamet1991} drei Bücher empfiehlt, wenn man lernen will, wie
  man Geschichten erzählt: \hyphenquote{english}{The Uses of Enchantment: The
  Meaning and Importance of Fairy Tales} von Bettelheim,
  \hyphenquote{english}{The Interpretation of Dreams} von Freud und
  \hyphenquote{english}{Memories, Dreams, Reflections} von Jung. Drei Bücher,
  die die Psyche der Menschen, ihre innersten Triebe, verstehen will. Die
  Regeln, die sich mit dem Zugang zu seiner Leserschaft beschäftigen, haben sich
  interessanterweise, in den letzten 2000 Jahren kaum verändert. Der Text um den
  auch die modernste Drehbuch-Schreib-Fibel nicht herum kommt ist über 2000
  Jahre alt. Die \citetitle{Aristoteles1991} von \inparencite{Aristoteles1991}.
  Aufbauend auf diesen Klassiker gibt es ein stabiles Paket an Büchern zu
  Schreibkunst und Werke, die auf diese Regeln aufbauen. Dieser stabile
  \emph{Stammbaum}, der die Struktur der meisten Geschichten vorgibt. Und selbst
  wenn wir auf Godot warten, tun wir das nur weil wir Leserinnen und Leser an
  die Regeln der Schreibkunst gewohnt sind. Somit werden Bücher nach den Regeln
  der Schreibkunst in \emph{Schwingung versetzt}. Diese Regeln, oder wie
  \inparencite[10]{McKee2001} schreibt, die Prinzipien, bestimmen nicht wie eine
  Geschichte auszusehen hat. Sie beschreiben einfach wie Geschichten
  funktionieren. Sie sind die Sprache die Leserschaft und Autorenschaft sprechen
  um sich zu verstehen. \parencite[30]{Daehnke2003} Doch wie jede Sprache ist
  sie auch eine Eingrenzung. Sie gibt den Rahmen, den Diskursraum vor in dem
  sich die Geschichten bewegen werden.

  Wohl eines der augenscheinlichsten Elemente der Prinzipien des Schreibens ist
  die der Hauptfiguren, der Protagonistin oder des Protagonisten.  % Die
  Hauptfigur oder die Hauptfiguren\footnote{\blockcquote[155]{McKee2001}{Im
  allgemeinen ist der Protagonist eine einzelne Figur. \textelp{} In
  \film{Panzerkreuzer Potemkin} bildet eine ganze Gesellschaftsklasse, das
  Proletariat, einen massiven \emph{Plural-Protagonisten}} Plural-Hauptfiguren
  unterliegen zwei Bedingungen: sie müssen denselben Wunsch haben und gemeinsam
  Leiden oder profitieren. \parencite[155]{McKee2001}} sind ein großer Teil von
  dem oben angesprochenen Draht zur Leserschaft. Im Idealfall erkennen wir uns
  in der Hauptfigur wieder und wollen das sie bekommt was sie will.
  \parencite[161]{McKee2001} \blockcquote[161]{McKee2001}{Ein
  Publikum\textelp{}vermag zwar , sich in jede Figur einzufühlen, in Ihren
  Protagonisten aber muß\textins{:sic} es sich einfühlen. Wenn nicht, dann ist
  das Band zwischen Publikum und Story gerissen} %\nocite{Eisenstein1925}  Geht
  man davon aus, dass ein Band zwischen Leserschaft und Geschichte notwendig
  ist, dann geht das nicht, ohne dass sich die Leserin oder der Leser in die
  Hauptfigur einfühlen. Die Hauptfigur ist die Seele der Geschichte. Wenn wir an
  unsere Lieblingsbücher denken, denken wir an die Hauptfiguren und wenn eine
  Autorin oder ein Autor ein Problem designt, dann um die Hauptfigur zum
  leuchten zu bringen. \blockcquote[407]{McKee2001}{Im Wesentlichen bringt der
  Protagonist die übrigen Rollen hervor. Alle anderen Figuren sind in einer
  Story in der Hauptsache deshalb, um zum Protagonisten eine Beziehung
  einzugehen und dazu beizutragen, allen Dimensionen der komplexen Natur des
  Protagonisten Gestalt zu verleihen.}   Wenn sich nun die Leserschaft in die
  Hauptfigur einfühlt, mit ihr die Geschichte erlebt, dann hat dieses Erleben
  natürlich einen Einfluss auf die Leserschaft. Das wichtige ist also, was die
  Hauptfigur \emph{erlebt}, wie sie mit ihrer Umwelt inter\emph{agiert}. Da eine
  ganze Leserschaft durch eine Hauptfigur gleich agiert, verbindet das eine
  Leserschaft. Einer Leserschaft ist eine Gruppe von Menschen die über dieselben
  Hauptfiguren miteinander verbunden sind. Wir untersuchen in diesem Artikel
  Zusammenhänge zwischen Leserschaft und Gender. Doch zunächst gilt es zu
  ergründen was Gender eigentlich ist.

  \subsection{Forschung zu Geschlecht}


  Gender ist ein Ausdruck aus dem Englischen, der das \emph{soziale} Geschlecht
  bezeichnet. In diesem Sinne ist es ein \hyphenquote{french}{fait social} im
  klassischen Sinne.\footnote{Leider geht das \emph{fait}, also \emph{gemacht}
  bei der Übersetzung verloren und im Englischen und Deutschen wird noch immer
  über Konstruiert oder nicht gestritten. \parencite[152--161]{Latour2010}}
  \parencite[Kap.\,1]{Durkheim1970} Doch so klar ist es in der Genderforschung
  nicht. Die Genderforschung ist ein heterogenes Feld mit, wie in der Soziologie
  üblich, vielen, theoretisch gesehen, inkompatiblen Standpunkten.
  \parencite[67]{Nissen1998}

  Schon die Einteilung der Standpunkte und wie man mit ihnen umgehen soll,
  stellt ein Problem dar. \inparencite[86]{Nissen1998} teilt die Ansätze in die
  \enquote{\enquote{drei Räume} des Feminismus} ein: Gleichheit, Differenz und
  Dekonstruktion. Sie meint, man solle sich in den drei Räumen
  \enquote{einrichten}. Damit meint sie, man solle sich einem Mix der Theorien
  bedienen um möglichst viele Aspekte des Problems abzudecken.
  \inparencite[216]{Gildemeister2000} teilt die Positionen grob in
  \enquote{Geschlecht als \emph{Strukturkategorie} und Geschlecht als
  \emph{soziale Konstruktion}} ein. Jedoch auch für sie ist eine Verbindung der
  Positionen wichtig. \blockcquote[223]{Gildemeister2000}{Umso wichtiger wird
  es, solche Verfahren zu entwickeln, in denen die interaktive Herstellung von
  Geschlecht verbunden wird mit der Analyse von Geschlechterordnungen in
  modernen Gesellschaften. Bislang steht weitgehend aus, Struktur- und
  Prozessanalysen miteinander zu verbinden oder, wie es auch heißt: Analyse
  sozialer Ungleichheit mit dem Fokus auf \enquote{soziale Konstruktion}.}
  Entgegengesetzt dazu sieht \inparencite[75]{Riegraf2008} die
  \enquote{\textelp{} Herausforderung für eine theoretisch anspruchsvolle und
  anwendungsbezogene Forschung zu Geschlecht \textelp{} darin, entlang der
  skizzierten Pole eine Standortbestimmung vorzunehmen.}

  Geschlecht als Strukturkategorie heißt, Geschlecht ist ein messbares Merkmal
  der Gesellschaft wie eine Schicht oder eine Klasse. Der Ansatz verwendet
  Geschlecht als Analyse-Einheit. Dadurch werden Aussagen über Ungleichheit oder
  Gleichheit möglich.  Die zwei Räume, Gleichheit und Differenz, von
  \citeauthor{Nissen1998}, fassen Geschlecht als Strukturkategorie auf. Jedoch
  haben beide Ansätze unterschiedliche Grundannahmen und unterschiedliche Ziele.
  Die Differenzpositionen gehen davon aus, dass es einen Unterschied zwischen
  Frauen und Männer gibt. Das rechtfertigt jedoch nicht, dass der Mann über der
  Frau steht. Ziel dieser Ansätze ist eine \emph{Aufwertung} der Weiblichkeit.
  Der Gleichheitsansatz geht davon aus, dass von Geburt an alle Menschen gleich
  sind.  Die, als Strukturkategorie messbaren, Unterschiede zwischen den
  Geschlechtern sind Konstruktionen, in die wir Menschen hineingepresst werden.
  Die Konstruktionen erzeugen eine (reale) Unterscheidung zwischen Frau und Mann
  die dem Mann hilft, seine Stellung in der sozialen Hierarchie zu festigen.
  \parencite[181]{Hertz2007} \textcquote[181]{Hertz2007}{Und die Männer, die
  sich heute an den Forderungen der Frau stören, berufen sich auf die
  \emph{natürliche} Unterlegenheit der Frau.} Der Gleichheitsansatz verwendet
  Geschlecht als Strukturkategorie, jedoch sieht er Geschlecht auch als soziale
  Konstruktion.

  Geschlecht als soziale Konstruktion ist eine problematische Einteilung, weil
  der Begriff \emph{Konstruktion} je nach erkenntnistheoretischer Position etwas
  anderes bedeutet. \parencite[219]{Gildemeister2000} Wichtig ist jedoch allen
  Positionen, die Geschlecht als Konstruktion bezeichnen, die Betonung des
  Werden von Geschlecht. Um klar zu machen, dass man für das \emph{werden}
  soziologische Erklärungen sucht, ist es wichtig sich von
  naturwissenschaftlichen zu Distanzieren. An deutlichsten machen dies
  \inparencite[126]{West1987}. Sie unterscheiden zwischen dem
  naturwissenschaftlichen Geschlecht (sex), der Kategorie Geschlecht (sex
  category) und dem von der Geschlechts-Kategorie abhängigen Verhalten (gender).
  Gender ist ein Unterschied den man macht. Anders als bei Geschlecht als
  Strukturkategorie wird sich nicht auf die Beziehungen von Frauen zu Männern
  konzentriert, sondern wie und warum wir in Frauen und Männer denken. Gender
  ist nicht Folge von Struktur sondern Folge von Handlung. Um das zu betonen
  wird auch von \emph{doing gender} gesprochen.
  \hyphentextcquote{english}[137]{West1987}{Doing gender means creating
  differences between girls and boys and women and men, differences that are not
  natural, essential, or biological.} Somit ist das soziale Geschlecht per
  Definition immer Ergebnis einer Tätigkeit. Das lenkt das Interesse auf die
  handelnden Personen und den Raum der sie so handeln lässt. Diese Prozesse
  werden de-, oder wie \inparencite{Gildemeister1992} schreiben, re-konstruiert.

  Unser Ziel ist es, sichtbar zu machen, welche Rolle Bücher bei der
  Konstruktion von Geschlechterunterschieden zwischen Mädchen und Buben spielen.
  Wir versuchen eine Kette von Akteuren zu bauen von der Strukturkategorie
  Geschlecht, also den Unterschieden zwischen Mädchen und Buben, bis zur
  Konstruktion des Geschlechts durch Kinderbücher.
