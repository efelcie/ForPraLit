\documentclass[]{article}
\usepackage[T1]{fontenc}
\usepackage{lmodern}
\usepackage{amssymb,amsmath}
\usepackage{ifxetex,ifluatex}
\usepackage{fixltx2e} % provides \textsubscript
% use microtype if available
\IfFileExists{microtype.sty}{\usepackage{microtype}}{}
\ifnum 0\ifxetex 1\fi\ifluatex 1\fi=0 % if pdftex
  \usepackage[utf8]{inputenc}
\else % if luatex or xelatex
  \usepackage{fontspec}
  \ifxetex
    \usepackage{xltxtra,xunicode}
  \fi
  \defaultfontfeatures{Mapping=tex-text,Scale=MatchLowercase}
  \newcommand{\euro}{€}
\fi
% Redefine labelwidth for lists; otherwise, the enumerate package will cause
% markers to extend beyond the left margin.
\makeatletter\AtBeginDocument{%
  \renewcommand{\@listi}
    {\setlength{\labelwidth}{4em}}
}\makeatother
\usepackage{enumerate}
\ifxetex
  \usepackage[setpagesize=false, % page size defined by xetex
              unicode=false, % unicode breaks when used with xetex
              xetex]{hyperref}
\else
  \usepackage[unicode=true]{hyperref}
\fi
\hypersetup{breaklinks=true,
            bookmarks=true,
            pdfauthor={},
            pdftitle={},
            colorlinks=true,
            urlcolor=blue,
            linkcolor=magenta,
            pdfborder={0 0 0}}
\setlength{\parindent}{0pt}
\setlength{\parskip}{6pt plus 2pt minus 1pt}
\setlength{\emergencystretch}{3em}  % prevent overfull lines
\setcounter{secnumdepth}{0}


\usepackage{ctable}
\usepackage{float} % provides the H option for float placement
%\usepackage{dcolumn}   % Ermöglicht das Ausrichten von Tabellen am Dezimalpunkt
\usepackage{% 
  ellipsis,   % Korrigiert den Weißraum um Auslassungspunkte
% ragged2e,   % Ermöglicht Flattersatz mit Silbentrennung
  marginnote, % Für bessere Randnotizen mit \marginnote statt marginline
  microtype,  % Mikrotypografische Feinheiten 
      % Andere Einstellungen siehe microtype-Handbuch
  eurosym,  % echte Euro-Symbol
  xspace,   % ermöglich Lehrzeichen, dass nicht vor Satzzeichen ist
% wasysym,  % Sammlung von Symbolen
  blindtext,  % Einfügen von Blindtext
  lmodern,
  rotating, % Erlaubt vertikalen Text (Tabellen)
  dcolumn   % Ermöglicht das Ausrichten von Tabellen am Dezimalpunkt
}



\author{}
\date{}

\begin{document}

Inhalt ======

\begin{enumerate}[<+->][1.]
\item
  Was sind Mädchenbücher?
\item
  Welche Merkmals haben Mädchenbücher?

  \begin{enumerate}[<+->][1.]
  \item
    Verpackung

    \begin{enumerate}[<+->][1.]
    \item
      Kaufentscheidung
    \item
      Farbe
    \item
      Text

      \begin{enumerate}[<+->][1.]
      \item
        Autor\_in
      \item
        Hauptfigur
      \item
        Titel
      \item
        Klappentext

        Genre
      \end{enumerate}
    \end{enumerate}
  \item
    Inhalt

    \begin{enumerate}[<+->][1.]
    \item
      Figur

      \begin{enumerate}[<+->][1.]
      \item
        Geschlecht
      \item
        Doing Gender
      \end{enumerate}
    \item
      Kommunkationsstile
    \item
      Handeln

      Ziel-Orientierung
    \item
      Figur
    \end{enumerate}
  \item
    Verbindung zwischen Verpackung und Inhalt
  \end{enumerate}
\item
  Wo entstehen Mädchenbücher?

  \begin{enumerate}[<+->][1.]
  \item
    Autor\_innen
  \item
    Verlage
  \item
    Eltern
  \item
    Peers
  \end{enumerate}
\end{enumerate}

Text

\end{document}