%\singlespacing

\ctable[       
    caption = Zuordnung von Inhalten zu Methoden, % Tabellenüberschrift       
    label   = zuo,       
    cap   = Zuordnung: Inhalte--Methoden, % Kurztitel f. Tabellenverz.  
    pos   = tbp, % Positon d. Tabelle       
    width   = \textwidth, % Tab.br. \textwidth, \columnwidth     
    ]
    {>{\raggedright}Xcccc}{  % Aufteilung d. Spalten
  % Fußnoten     
}{              % Hier beginnt die Tabelle
\FL \small Fragestellung  & \begin{sideways}\small Fragebogen\end{sideways}&
\begin{sideways}\small Inhaltsanalyse\end{sideways}& \begin{sideways}\small
quant. Inhaltsa.\end{sideways}& \begin{sideways}\small Qual.
Inhaltsa.\end{sideways}        
\ML Gibt es Unterschiede bei den Lesepräferenzen von Mädchen und Buben? (Bubenbücher, Mädchenbücher) & x &   &   &        
\NN[0.5em] Gibt es Unterschiede zwischen Mädchen- und Bubenbücher? (Erscheinung, Inhalt, Aufgeben, \ldots)                         &
& x & x & x       
\NN[0.5em] Kann man Bubenbücher anhand von \emph{oberflächlichen} Merkmalen von Mädchenbüchern unterscheiden? (Farben, Themen, Umfang, Autorengeschlecht, \ldots)
&   & x &   &
\NN[0.5em] Kann man Bubenbücher anhand von inhaltlichen Merkmalen von
Mädchenbüchern unterscheiden? (Schreibweise, Stereotype, \ldots)
&   &   & x &        
\NN[0.5em] Sind Unterschiede tatsächlich in den bevorzugten Büchern auffindbar? (Rollensettings, Lösungen von Aufgaben, \ldots)
&   &   &   & x       
\NN[0.5em] Gibt es Bücher die die Einteilungen (Mädchen- und Bubenbuch) besonders gut repräsentieren? (Welche)                         
& & x & x & x       
\LL }     

%\onehalfspacing