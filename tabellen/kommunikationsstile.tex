
\newcommand{\minitab}[2][l]{\begin{tabular}{#1}#2\end{tabular}}
\ctable[
  %  cap    = ,
    caption = {Kommunikationsstile der Geschlechter},
    label   = stile ,
    %pos   = htp,
    width    = \textwidth
]{X<{\raggedright}X<{\raggedright}X<{\raggedright}X<{\raggedright}}{
    \tnote{Quelle: Klaus/Röser (1996, S. 289)}
}{                  
  \FL \multirow{3}{*}{\small Fokus der Analyse} &  \multicolumn{2}{c}{\small grundlegende Aspekte} & \multirow{3}{*}{\small Beispiel: Soap Opera}
  \NN & \multicolumn{2}{c}{\small kommunikativen Handelns} &
  \NN[1em] & \multicolumn{1}{c}{\small \enquote{männlich}} & \multicolumn{1}{c}{\small \enquote{weiblich}} &
  \ML Lebenszusammen-hang \small (Prokop 1976) & Orientierung an: Arbeit und Beruf & Orientierung an: Familie und Beziehungen & Bezugspunkt: familiäre Reproduktionsarbeit
  \NN[1em] & sachbezogenes Handeln & bedürfnisbezogenes Handeln & Blickwinkel von Gemeinschaft
  \NN[2em] Moral \small (Gilligan 1991) & Position an der Spitze: Trennung/Abgrenzung & Knoten im Beziehungsnetz: Zuwendung/Fürsorge & Phantsie von Gemeinschaft
  \NN[3em] Denken \small (Belnky u.a. 1989) & abgelöstes Denken & vebundenes Denken & mehrsträngig und zyklich statt linear und kontinuierlich
  \NN[2em] Sprechen \small (Trämel-Plötz 1983) & Wettstreit und Hierachie & Kooperation und Gleichrangigkeit & spezifisch weibliche  Erzählform: Zeit  zum Austausch  und um Folgen %nachzuspüren
  \NN Kommunikative Gegenkultur \small (Modelmog 1991) & räumlihce zeitliche und inhaltiche Distanz & Sinnlichkeit, Ruhe und Intimität & nachzuspüren\LL
}