\onehalfspace

\chapter{Einleitung}

\begin{verbatim}
Im Rahmen des gemeinsamen Forschungsprojekts, das sich mit der Enstehung von Geschlechteridentitäten im Kindesalter und Rollenangeboten für Mädchen und Buben beschäftigt, konzentrierten wir uns auf die Bedeutung von Kinderbüchern.

Mögliche Einflussfaktoren auf die Bildung des sozialen Geschlechts zu ermitteln, ist sicher keine neue Idee. Dennoch haben wir in vorangegangenen Studien keine Antwort auf unsere Fragen, was Mädchen und Buben nun wirklich lesen und warum sie das tun, bekommen. Wir nehmen an, dass es  Unterschiede im Leseverhalten gibt, die neben vielen anderen Faktoren auf die Geschlechterrollenbildung von Kindern Einfluss nehmen, gleich wie  sich bereits vorhandene *Rollenspezifika* umgekehrt auf die Lesepräferenzen auswirken können. Oft werden Bücher lediglich auf die Häufigkeit von weiblichen und männlichen Charakteren und die Art, wie diese dargestellt werden, untersucht. Die Erweiterung dieses Ansatzes auf Merkmale, die die Leseentscheidung von Kindern beeinflussen können, bietet für uns einen interessanten Zugang, der auch verfolgt wurde. Relevant wäre auf jeden Fall eine weitere Untersuchung, die herauszufinden versucht, wie das Verhalten von Charakteren in Büchern oder von anderen Vorbildern auf das geschlechtsspezifische Handeln von Kindern Einfluss nehmen kann.
Zu Beginn soll auf den Genderbegriff und unser Verständnis von *Doing Gender* näher eingegangen werden, wobei auch unterschiedliche Strömungen in der Geschlechterforschung kurz angeschnitten werden sollen. 
\end{verbatim}

Wie wir im ersten Kapitel sehen werden, bietet die vorhandene Literatur,
Theorien an, die sich um den Einfluss von Büchern im Allgemeinen drehen
und unsere Interpretation von einem Buch als Akteur darstellen. Außerdem
soll ein kurzer Überblick über die Kinderliteratur, ihre Geschichte und
unterschiedlichen Funktionen, in das Thema einführen. Bevor wir unsere
selbstständige Analyse starteten, war uns ebenfalls wichtig, vorhandene
Forschungen, die sich mit dem Thema Gender in Büchern auseinandergesetzt
hatten, zu sichten, um brauchbare Methoden zu verwenden und Ergebnisse
in unsere Hypothesen miteinfließen zu lassen. Im Forschungsdesign werden
unsere Fragestellungen vorgestellt und mit verschiedenen Analysemethoden
verknüpft. Der zweite Teil dieser Arbeit beschäftigt sich mit der
genauen Vorgehensweise und schließlich den Ergebnissen unserer
Untersuchung, wobei wir hier schrittweise verfahren wollen, um auch den
Forschungsprozess sichtbar zu machen.
