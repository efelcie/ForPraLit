\onehalfspace

\chapter{Einleitung}

Im Rahmen des gemeinsamen Forschungsprojekts, das sich mit der Enstehung
von Geschlechteridentitäten im Kindesalter und Rollenangeboten für
Mädchen und Buben beschäftigt, konzentrierten wir uns auf die Bedeutung
von Kinderbüchern. Zu Beginn soll auf den Genderbegriff und unser
Verständnis von \emph{Doing Gender} näher eingegangen werden, wobei auch
unterschiedliche Strömungen in der Geschlechterforschung kurz
angeschnitten werden sollen. Wie wir im ersten Kapitel sehen werden,
bietet die vorhandene Literatur, Theorien an, die sich um den Einfluss
von Büchern im Allgemeinen drehen und stellen unsere Interpretation von
einem Buch als Akteur dar. Außerdem soll ein kurzer Überblick über die
Kinderliteratur, ihre Geschichte und unterschiedlichen Funktionen, in
das Thema einführen. Bevor wir unsere selbstständige Analyse starteten,
war uns ebenfalls wichtig, vorhandene Forschungen, die sich mit dem
Thema Gender in Büchern auseinandergesetzt hatten, zu sichten um
brauchbare Methoden zu verwenden und Ergebnisse in unsere Hypothesen
miteinfließen zu lassen. Im Forschungsdesign werden unsere
Fragestellungen vorgestellt und mit verschiedenen Analysemethoden
verknüpft. Der zweite Teil dieser Arbeit beschäftigt sich mit der
Vorgehensweise und den Ergebnissen unserer Untersuchung, wobei wir hier
schrittweise verfahren wollen um den Forschungsprozess zu
repräsentieren.

\section{Relevanz}

Mögliche Einflussfaktoren auf die Bildung des sozialen Geschlechts zu
ermitteln, ist sicher keine neue Idee. Dennoch haben wir in
vorangegangenen Studien, keine Antwort auf unsere Fragen, was Mädchen
und Buben nun lesen und warum sie das tun, bekommen. Häufig werden
Bücher lediglich auf die Häufigkeit von weiblichen und männlichen
Charakteren und die Art, wie diese dargestellt werden, untersucht. Die
Erweiterung dieses Ansatzes auf Merkmale, die die Leseentscheidung von
Kindern (die zum Großteil aktiv entscheiden, was sie lesen) beeinflussen
können, bietet für uns einen interessanten Zugang, der auch verfolgt
wurde. Relevant wäre auf jeden Fall eine weitere Untersuchung, die
herauszufinden versucht, wie das Verhalten von Charakteren in Büchern
oder von anderen Vorbildern auf das geschlechtsspezifische Handeln von
Kindern Einfluss nehmen kann, wobei hierbei wieder gilt: Kinder agieren
aktiv mit ihrer Umwelt und stellen keine \emph{Black Box} dar.
