\documentclass[
%	fontsize=10pt,	% Standard-Schriftgröße
	% fontsize=12pt,	% Für Korrekturversionen
%	fontsize=14pt,	% Ideale Schriftgröße für 3. Klasse VS
%	paper=a5,
%	twoside,
	% DIV=9,
	pagesize,
	DIV=15,
	headinclude=true,
	footinclude=false,
	twocolumn,
	% headings=small,
%	notitlepage	
%      ]{scrartcl}		% Koma-Script Artikel
     ]{scrreprt}		% Koma-Script Report (nötig für paperandpencil)
%      ]{scrbook}		% Koma-Script Report (nötig für paperandpencil)
		

%\usepackage{
%	array,		% Array Paket wird für paperandpencil benötigt
%      	paperandpencil	% Zur Erstellung von Fragebögen
%	papi		% Für Papiergestützte Befragungen
%}

%\usepackage[
%	top=2.5cm, 
%	left=2.5cm, 
%	bottom=2.5cm,
%	right=2.5cm
%	]{geometry}	% Einstellungen für Fragebogen (paperandpencil)



\usepackage[utf8]{inputenc}		% Ermöglicht das Umgehen mit Unicode (Umlaute).
\usepackage[french, english, ngerman]{babel}	% Lädt Spracheinstellungen.
\usepackage[				% Verwaltet Anführungszeichen.
  	babel,				%
	german=quotes,
	english=british]{csquotes}	% Deutsche Anführungszeichen.
\usepackage{ctable}			% Für coolere Tabellen
%\usepackage[just=true]{stdpage} 	% Normseite
\usepackage[T1]{fontenc}		% Schriften auf mehr Zeichen erweitern.

\usepackage{% 
	ellipsis, 	% Korrigiert den Weißraum um Auslassungspunkte
%	ragged2e, 	% Ermöglicht Flattersatz mit Silbentrennung
	marginnote,	% Für bessere Randnotizen mit \marginnote statt marginline
	microtype, 	% Mikrotypografische Feinheiten 
			% Andere Einstellungen siehe microtype-Handbuch
	eurosym, 	% echte Euro-Symbol
	xspace,		% ermöglich Lehrzeichen, dass nicht vor Satzzeichen ist
%	wasysym, 	% Sammlung von Symbolen
	blindtext,	% Einfügen von Blindtext
	lmodern,
	rotating,	% Erlaubt vertikalen Text (Tabellen)
	dcolumn		% Ermöglicht das Ausrichten von Tabellen am Dezimalpunkt
}
\usepackage{tikz}
\usepackage{Sweave}

\usepackage[
  %	style=authoryear-icomp,	% Autor:Jahr Stil
  	style=fiwi2,		% Autor:Jahr Stil
	backend=biber,		% Eine BibTeX Alternative
%	maxbibnames=99,
%	maxcitenames=2,		% Namen in Zitaten
%	mincitenames=1,		% Namen in Zitat bei et al.
	uniquelist=false,
	origyearwithyear=false,
	citefilm=full,
	series=true,
%	url=false,
	doi=true,
	pages=bib,
	publisher=false
      ]{biblatex}		% Literaturverwaltung
\bibliography{literatur/primaer}
\bibliography{literatur/sekundaer}		% Pfad zur Literatur-Datei '<name.bib>' ohne .bib
\title{Mädchenbücher -- Bubenbücher}
\author{Peter Flucher \and Lukas Kaiser \and Lisa Weiler}
\subtitle{Eine empirische Untersuchung über das Verhältnis von Geschlecht der Lesenden und Kinderbücher}
\subject{Forschungsbericht}
\date{Universität Graz 2013}

% 				% Anpassung von authoryear
%\renewcommand{\postnotedelim}{\addcolon\addspace}	% : statt ,
%\DeclareFieldFormat{postnote}{#1}			% kein S.
%\renewcommand{\labelnamepunct}{\addcolon\addspace}	% : statt .
							% in LitListe

%\usepackage{DejaVuSans}
%\usepackage{DejaVuSansCondensed}
%% Another possibility is
%% \usepackage{dejavu}
%% which loads the DejaVu Serif and DejaVu Sans Mono fonts as well
%\renewcommand*\familydefault{\sfdefault} %% Only if the base font of the document is to be sans serif
%\usepackage[T1]{fontenc}
\usepackage{setspace}
% \onehalfspacing
\usepackage{color}
\usepackage[]{xcolor}
\usepackage[
	pdftitle={Das Gender von Kinderbüchern},
	pdfsubject={Forschungsdesign},
	pdfauthor={Flucher, Kaiser, Weiler},	
	pdfkeywords={},
	colorlinks=true,
	linkcolor=darkgray,
	citecolor=teal,
	filecolor=darkgray,
	urlcolor=blue,
	]{hyperref}

\newcommand{\zB}{\mbox{z.\,B.}\xspace}	% Definiert einen Befehl für z.B.

%\renewcommand{\familydefault}{\sfdefault}
%\renewcommand{\familydefault}{\rmdefault} 
					%Standardschrift ist Roman (wegen stdpage)


\SetCiteCommand{\parencite}		% Notwendig für (Zitieren)
\renewcommand{\mkccitation}[1]{ #1}	%


\usepackage{pdfpages}


\usepackage[thref,hyperref]{ntheorem}




\newtheorem{frage}{Forschungsfrage}


% \theoremindent4em
\newtheorem{hyp}{Hypothese}
% \theoremindent4em
% \theoremnumbering{alph}
\newtheorem{subhyp}{Hypothese}[hyp]
% \renewcommand{\thesubhyp}{\thehyp\alph{subhyp}}


\usepackage[labelfont={color=teal}]{caption}


\newenvironment{myquote}%
   {\begin{quote}\small}%
   {\end{quote}}%
\SetBlockEnvironment{myquote}


\setkomafont{subsection}{
	\normalfont \normalsize
}

\setkomafont{subsubsection}{
	\normalfont \normalsize
}


\setkomafont{chapter}{\huge \textcolor{teal}
}


\setkomafont{section}{
	\normalfont \bfseries \normalsize \textcolor{teal}
}

\setkomafont{subject}{
	\scshape
}

% \setkomafont{title}{
% 	\normalfont \bfseries \large \raggedright
% }

\setkomafont{subtitle}{
	 \textcolor{teal}}

\setkomafont{paragraph}{
	\noindent \normalfont \bfseries \normalsize }

\setkomafont{descriptionlabel}{
	\textcolor{teal}}

\setkomafont{footnoterule}{
	\textcolor{teal}
}

\SetCiteCommand{\parencite}		% Notwendig für (Zitieren)
\renewcommand{\mkccitation}[1]{ #1}	%

\usepackage{relsize}

\usepackage{fancyhdr}



\pagestyle{fancy}
\lhead{\iffloatpage{}{\leftmark}}
% \chead{}
\rhead{}
% \lfoot{}
% \cfoot{\thepage}
% \rfoot{}
% \renewcommand{\headrulewidth}{0.4pt}
% %\renewcommand{\footrulewidth}{0.4pt}

\renewcommand{\chaptermark}[1]{% 2. Do it now
\markboth{\thechapter.\ #1}{}} 

\makeatletter
\def\headrule{{\color{teal}\if@fancyplain\let\headrulewidth\plainheadrulewidth\fi
\hrule\@height\headrulewidth\@width\headwidth
\vskip-\headrulewidth}}
\makeatother

\renewcommand{\headrulewidth}{\iffloatpage{0pt}{2pt}}











\begin{document}

\maketitle
\singlespacing
\tableofcontents
\listoffigures % Abbildungsverzeichnis
\listoftables % Tabellenverzeichnis 
% \onehalfspacing
%\chapter{Vorwort}
%\Blindtext
%\part{Projekte}
%\chapter{Mädchenbücher--Bubenbücher}
%\begin{refsection}

\newpage 

\recalctypearea
\onehalfspace

\chapter{Einleitung}

\begin{verbatim}
Im Rahmen des gemeinsamen Forschungsprojekts, das sich mit der Enstehung von Geschlechteridentitäten im Kindesalter und Rollenangeboten für Mädchen und Buben beschäftigt, konzentrierten wir uns auf die Bedeutung von Kinderbüchern.

Mögliche Einflussfaktoren auf die Bildung des sozialen Geschlechts zu ermitteln, ist sicher keine neue Idee. Dennoch haben wir in vorangegangenen Studien keine Antwort auf unsere Fragen, was Mädchen und Buben nun wirklich lesen und warum sie das tun, bekommen. Wir nehmen an, dass es  Unterschiede im Leseverhalten gibt, die neben vielen anderen Faktoren auf die Geschlechterrollenbildung von Kindern Einfluss nehmen, gleich wie  sich bereits vorhandene *Rollenspezifika* umgekehrt auf die Lesepräferenzen auswirken können. Oft werden Bücher lediglich auf die Häufigkeit von weiblichen und männlichen Charakteren und die Art, wie diese dargestellt werden, untersucht. Die Erweiterung dieses Ansatzes auf Merkmale, die die Leseentscheidung von Kindern beeinflussen können, bietet für uns einen interessanten Zugang, der auch verfolgt wurde. Relevant wäre auf jeden Fall eine weitere Untersuchung, die herauszufinden versucht, wie das Verhalten von Charakteren in Büchern oder von anderen Vorbildern auf das geschlechtsspezifische Handeln von Kindern Einfluss nehmen kann.
Zu Beginn soll auf den Genderbegriff und unser Verständnis von *Doing Gender* näher eingegangen werden, wobei auch unterschiedliche Strömungen in der Geschlechterforschung kurz angeschnitten werden sollen. 
\end{verbatim}

Wie wir im ersten Kapitel sehen werden, bietet die vorhandene Literatur,
Theorien an, die sich um den Einfluss von Büchern im Allgemeinen drehen
und unsere Interpretation von einem Buch als Akteur darstellen. Außerdem
soll ein kurzer Überblick über die Kinderliteratur, ihre Geschichte und
unterschiedlichen Funktionen, in das Thema einführen. Bevor wir unsere
selbstständige Analyse starteten, war uns ebenfalls wichtig, vorhandene
Forschungen, die sich mit dem Thema Gender in Büchern auseinandergesetzt
hatten, zu sichten, um brauchbare Methoden zu verwenden und Ergebnisse
in unsere Hypothesen miteinfließen zu lassen. Im Forschungsdesign werden
unsere Fragestellungen vorgestellt und mit verschiedenen Analysemethoden
verknüpft. Der zweite Teil dieser Arbeit beschäftigt sich mit der
genauen Vorgehensweise und schließlich den Ergebnissen unserer
Untersuchung, wobei wir hier schrittweise verfahren wollen, um auch den
Forschungsprozess sichtbar zu machen.

% \chapter{Buch und Geschlecht}

\section{Forschung zu Geschlechtsrollen und Geschlechtsidentität}


 Gender ist ein englischer Ausdruck, der das \emph{soziale} Geschlecht bezeichnet. In diesem Sinne ist es ein \hyphenquote{french}{fait social} im klassischen Sinne.\footnote{Leider geht das \emph{fait}, also \emph{gemacht} bei der Übersetzung verloren und im Englischen und Deutschen wird noch immer über konstruiert oder nicht gestritten. \parencite[152--161]{Latour2010}} \parencite[Kap.\,1]{Durkheim1970} Doch in der Genderforschung ist es weniger klar: Sie ist ein heterogenes Feld mit, wie in der Soziologie üblich, vielen, theoretisch gesehen, inkompatiblen Standpunkten. \parencite[67]{Nissen1998} 

Schon die Einteilung der Standpunkte und wie man mit ihnen umgehen soll, stellt ein Problem dar. \inparencite[86]{Nissen1998} teilt die Ansätze in die \enquote{\enquote{drei Räume} des Feminismus} ein: Gleichheit, Differenz und Dekonstruktion. Sie meint, man solle sich in den drei Räumen \enquote{einrichten}. Damit meint sie, man solle sich einem Mix der Theorien bedienen um möglichst viele Aspekte des Problems abzudecken. \inparencite[216]{Gildemeister2000} teilt die Positionen grob in \enquote{Geschlecht als \emph{Strukturkategorie} und Geschlecht als \emph{soziale Konstruktion}} ein. Jedoch ist eine Verbindung der Positionen auch für sie wichtig.
\blockcquote[223]{Gildemeister2000}{Umso wichtiger wird es, solche Verfahren zu entwickeln, in denen die interaktive Herstellung von Geschlecht verbunden wird mit der Analyse von Geschlechterordnungen in modernen Gesellschaften. Bislang steht weitgehend aus, Struktur- und Prozessanalysen miteinander zu verbinden oder, wie es auch heißt: Analyse sozialer Ungleichheit mit dem Fokus auf \enquote{soziale Konstruktion}.}
	
Geschlecht als Strukturkategorie heißt, Geschlecht ist ein messbares Merkmal der Gesellschaft wie Schicht oder Klasse. Der Ansatz verwendet Geschlecht als Analyse-Einheit, wodurch Aussagen über Ungleichheit oder Gleichheit möglich werden.  Die zwei Räume, Gleichheit und Differenz, von \citeauthor{Nissen1998}, fassen Geschlecht als Strukturkategorie auf. Jedoch haben beide Ansätze unterschiedliche Grundannahmen und unterschiedliche Ziele. 
Die Differenzpositionen gehen davon aus, dass es einen Unterschied zwischen Frauen und Männern gibt. Das rechtfertigt jedoch nicht, dass der Mann über der Frau steht. Ziel dieser Ansätze ist eine \emph{Aufwertung} der Weiblichkeit.
Der Gleichheitsansatz geht davon aus, dass von Geburt an alle Menschen gleich sind. 
Die, als Strukturkategorie messbaren, Unterschiede zwischen den Geschlechtern sind Konstruktionen, in die wir Menschen hineingepresst werden. Die Konstruktionen erzeugen eine (reale) Unterscheidung zwischen Frau und Mann, die dem Mann hilft, seine Stellung in der sozialen Hierarchie zu festigen. \parencite[181]{Hertz2007}
\textcquote[181]{Hertz2007}{Und die Männer, die sich heute an den Forderungen der Frau stören, berufen sich auf die \emph{natürliche} Unterlegenheit der Frau.}
Der Gleichheitsansatz verwendet Geschlecht als Strukturkategorie, jedoch sieht er Geschlecht auch als soziale Konstruktion.

Geschlecht als soziale Konstruktion ist problematisch, weil der Begriff \emph{Konstruktion} je nach erkenntnistheoretischer Position etwas anderes bedeutet. \parencite[219]{Gildemeister2000} Allen gemeinsam ist allerdings die Betonung des Werdens von Geschlecht. Um klar zu machen, dass man für das \emph{Werden} soziologische Erklärungen sucht, ist es wichtig, sich von naturwissenschaftlichen zu Distanzieren. Am deutlichsten machen dies \inparencite[126]{West1987}. Sie unterscheiden zwischen dem naturwissenschaftlichen Geschlecht (sex), der Kategorie Geschlecht (sex category) und dem von der Geschlechts-Kategorie abhängigen Verhalten (gender). Gender ist ein Unterschied, den man macht. Anders als bei Geschlecht als Strukturkategorie konzentriert man sich hier nicht auf die Beziehungen von Frauen zu Männern, sondern wie und warum wir in diesen Kategorien überhaupt denken. Gender ist nicht Folge von Struktur sondern Folge von Handlung. Um das zu betonen wird auch von \emph{doing gender} gesprochen. 
\hyphentextcquote{english}[137]{West1987}{Doing gender means creating differences between girls and boys and women and men, differences that are not natural, essential, or biological.}
Somit ist das soziale Geschlecht per Definition immer Ergebnis einer Tätigkeit. Das lenkt das Interesse auf die handelnden Personen und den Raum, der sie so handeln lässt.
Diese Prozesse werden de-, oder wie \inparencite{Gildemeister1992} schreiben, re-konstruiert.

Unser Ziel ist es, sichtbar zu machen, welche Rolle Bücher bei der Konstruktion von Geschlechterunterschieden zwischen Mädchen und Buben spielen. Wir versuchen eine Kette von Akteuren von der Strukturkategorie Geschlecht, also den Unterschieden zwischen Mädchen und Buben, bis zur Konstruktion des Geschlechts durch Kinderbücher zu bauen.

\section{Buch als Akteur}

Bücher verknüpfen eine große Anzahl an Menschen, die Leserschaft, die Autorin oder den Autor, verschiedenste Inhalte, Theorien und Einstellungen.
Das Besondere an Akteur-Netzwerken, die keine Menschen sind, ist, dass sie ihre \emph{Arbeit}, wenn sie einmal da sind, mit viel weniger Aufwand als menschliche Akteur-Netzwerke verrichten.
Ein gutes Beispiel dafür ist der Hirte, der mit viel Aufwand seine Herde hütet und der Weidezaun, der, ist er einmal gebaut, dieselbe Arbeit allein durch seine Existenz verrichtet.
In unserer Welt gibt es viele Akteure, die ihre Arbeit verrichten, ohne dass wir die Arbeit als solche wahrnehmen.
Diese Arbeit, auf die man sich verlassen kann, wie auf das Wasser, das das Mühlrad antreibt, erscheint uns als \emph{Stabilität}.
Diese Stabilität ist für uns schon so gewöhnlich geworden, dass sie natürlich erscheint. Dieser Umstand verdeckt, dass sie das durch ständigen Aufwand Produzierte ist.

Will man die \emph{Mächtigkeit} eines Akteur-Netzwerkes definieren, so könnte man sagen, dass je mehr Akteure durch ein Akteur-Netzwerk miteinander verknüpft werden, es umso mächtiger ist.
Bücher haben die Fähigkeit unzählige Akteure miteinander zu riesigen Akteur-Netzwerken zu verbinden.
Von der Bibel wurden \zB geschätzte 2 bis 3 Milliarden Exemplare unters Volk gebracht. Sie verknüpft seit rund 2000 Jahren verlässlich Menschen und Werte auf der ganzen Welt.
Nicht nur bei der Bibel sehen wir, dass das Buch nicht nur verknüpft, sondern auch differenziert. Wer dieselben Bücher liest, gehört zusammen und grenzt sich so, von denen die es nicht tun, ab. Differenzen wie Kind/Erwachsener oder der Zugehörigkeit zu einer Nation, werden mit differenziertem Leseverhalten in Verbindung gebracht.
\parencites[Kap.\,3 in][]{Postman2011}[50]{McLuhan2012}

Es gibt bestimmte Prinzipien oder  Regeln die, wie \inparencite[10]{McKee2001} schreibt, bestimmen wie Geschichten funktionieren, aber nicht wie eine Geschichte auszusehen hat.  Sie sind die Sprache, die Leserschaft und Autorenschaft sprechen, um sich zu verstehen. \parencite[30]{Daehnke2003} Doch wie jede Sprache ist sie auch eine Eingrenzung. Sie gibt den Rahmen, den Diskursraum vor, in dem sich die Geschichten bewegen werden.

Die Hauptfigur oder die Hauptfiguren\footnote{\blockcquote[155]{McKee2001}{Im Allgemeinen ist der Protagonist eine einzelne Figur. \textelp{} In \film{Panzerkreuzer Potemkin} bildet eine ganze Gesellschaftsklasse, das Proletariat, einen massiven \emph{Plural-Protagonisten}}
Plural-Hauptfiguren unterliegen zwei Bedingungen: sie müssen denselben Wunsch haben und gemeinsam leiden oder profitieren. \parencite[155]{McKee2001}} sind ein großer Teil von dem oben angesprochenen Draht zur Leserschaft. Im Idealfall erkennen wir uns in der Hauptfigur wieder und wollen, dass sie bekommt was sie will. \parencite[161]{McKee2001} \blockcquote[161]{McKee2001}{Ein Publikum\textelp{}vermag zwar , sich in jede Figur einzufühlen, in Ihren Protagonisten aber muß\textins{:sic} es sich einfühlen. Wenn nicht, dann ist das Band zwischen Publikum und Story gerissen}
%\nocite{Eisenstein1925} 
Geht man davon aus, dass ein Band zwischen Leserschaft und Geschichte notwendig ist, dann geht das nicht, ohne dass sich die Leserin oder der Leser in die Hauptfigur einfühlen. Die Hauptfigur ist die Seele der Geschichte.  \blockcquote[407]{McKee2001}{Im Wesentlichen bringt der Protagonist die übrigen Rollen hervor. Alle anderen Figuren sind in einer Story in der Hauptsache deshalb, um zum Protagonisten eine Beziehung einzugehen und dazu beizutragen, allen Dimensionen der komplexen Natur des Protagonisten Gestalt zu verleihen.}  
Wenn sich nun die Leserschaft in die Hauptfigur einfühlt, mit ihr die Geschichte erlebt, dann hat dieses Erleben natürlich einen Einfluss auf die Leserschaft. Wichtig ist also, was die Hauptfigur \emph{erlebt}, wie sie mit ihrer Umwelt inter\emph{agiert}.
	 
Diese Grundannahmen betreffen auch Kinderbücher. Im nächsten Schritt soll geklärt werden, was Kinderbücher sind, vorher wollen wir aber noch auf die Konstruktion von Kindheit eingehen.

\section{Kindheit und Medien}

Bei \inparencite{Postman2011} heißt es, dass dadurch, dass das Wissen, das Kindern durch Bücher zugänglich (und nicht zugänglich) gemacht wird, die Kindheit überhaupt erst erzeugt wird. Erst durch die gezielte Auswahl und Herstellung von Kinderbüchern, die gewisse Aspekte des Lebens zeigen und andere ausblenden entsteht Kindheit. Kindheit ist somit ein geschützter Raum ohne Krankheit, Sexualität und Tod.
Gleichzeitig sind Veränderungen in den Kommunikationsmöglichkeiten und Angebote in  der Lage, Kindheit wieder verschwinden zu lassen. Mit Harold Innis teilt er die Auffassung,  dass Veränderungen innerhalb der Kommunikationstechnik drei Auswirkungen haben: die Veränderung der Interessensstruktur (worüber wird nachgedacht?), den Charakter der Symbole (womit wird gedacht?) und das Wesen der Gemeinschaft (wo entwickeln sich die Gedanken?). \parencite[34]{Postman1985} Wenn er vom \enquote{Verschwinden der Kindheit} spricht, macht er die durch die neuen elektronischen Medien vermittelten Inhalte, die die kindliche Phantasie nicht mehr anregen, verantwortlich: Bilder und andere Darstellungsformen im Fernsehen, also vorrangig visuelle Medien, bieten der eigenen Vorstellungskraft, im Gegensatz zum Text in Büchern, wenig Entfaltungsmöglichkeiten. Gleichzeitig laufen Reflexions- wie Kritikfähigkeit Gefahr zu verkümmern, da nur elementare Fähigkeiten gebraucht würden. Außerdem kritisiert er, dass zunehmend für Erwachsene typische Wünsche transportiert werden, die die Neugier und Andersartigkeit des Kindseins gefährden, auch weil sie keine Geheimnisse mehr hüten. \parencite[93\psq]{Postman1985} Erfahrungsräume, die nur Literatur bietet, können verloren gehen. Lesesozialisation kann als Ausschnitt der Mediensozialisation gesehen werden: durch Lesen wird nämlich nicht nur die Fähigkeit zur Dekodierung von schriftlichen Texten gefördert, sondern es werden auch Kommunikationsinteressen und kulturelle Haltungen erworben.\footnote{Der Literatur wurde nicht immer eine positive Funktion zugeschrieben, gerade der Unterhaltungsliteratur warf man vor, Kinder von sinnvollen Tätigkeiten abzuhalten. Erst durch die Konkurrenz der elektronischen Medien schien der Umgang mit Texten förderungswürdig. } \parencite[22\psqq]{Weinkauff2010}

\section{Kinderliteratur} 

Obwohl sich Kinder- von Jugendliteratur anhand eigener Attribute abgrenzen lässt, bilden sie in theoretischen und empirischen Arbeiten meist eine Einheit, die im Kontrast zur Erwachsenenliteratur steht. Kinderliteratur kann anhand spezifischer Textmerkmale, Inhalte und Funktionen in verschiedene Genres  eingeteilt werden, zu denen etwa Kriminalgeschichten, Abenteuer oder Märchen zählen.  Außerdem werden Kinderbücher  im Allgemeinen mit Altersempfehlungen versehen.\parencite[10]{Ewers2011}
Als Mädchen- oder Bubenliteratur werden die Kommunikationen bezeichnet, die vorwiegend von  weiblichem oder männlichem Lesepublikum angenommen werden, gleichzeitig  scheinen manche Genres, ebenso wie Inhalte oder Gestaltungsstile von Büchern, explizit unterschiedliche Vorlieben von Buben und Mädchen anzusprechen und zu betonen.
Kinderliteratur wird von den gesellschaftlichen, wirtschaftlichen und politischen Verhältnissen der jeweiligen Zeit geprägt: Inhalte, der (ästhetische) Gebrauch von Sprache, erzieherische Absichten und pädagogische Konzepte wie Ansichten der AutorInnen haben sich seit der Entstehung dieses Literaturkonzepts stark verändert.  Die zeitgenössische Auffassung von Kindheit, die ein individualistisches, postmodernes Menschenbild und das Ideal eines autoritativ-partizipativen\footnote{Der autoritativ-partizipative Erziehungsstil  zeichnet sich durch Wärme, Wertschätzung, dem Vereinbaren von Regeln und begründeter Sanktionierung aus. Das Kind kann die Eltern- Kindbeziehung mitgestalten, es wird zwar geleitet, lernt aber selbständig Verantwortung zu übernehmen. \parencite[35]{Kuttler2009}} Erziehungsstils verfolgt, kann mit ziemlicher Sicherheit nicht mit den Normen- und Wertvorstellungen anderer Epochen oder Kulturkreisen verglichen werden. 
Zur Veranschaulichung kann eine Literaturform, die Ende des achtzehnten Jahrhunderts in England entstanden ist und speziell an Mädchen gerichtet war – die sogenannte  \emph{Backfischliteratur}- dienen. Ihr Hauptziel war, Mädchen auf die spätere Rolle als Hausfrau, Mutter und Ehefrau vorzubereiten. Frauen sollten vor allem demütige und religiöse Eigenschaften besitzen, außerdem war das Finden eines geeigneten Ehemannes von entscheidender Bedeutung. Allerdings hat sich die Mädchenliteratur inzwischen stark verändert: Während im traditionellen Mädchenbuch vorherrschende Rollenstereotype und traditionelle Wertmaßstäbe verinnerlicht werden sollen, wird im nächsten Entwicklungsschritt gegen diese protestiert, um dann im emanzipierten Mädchenbuch vor allem die Identitätsfindung zu betonen und sämtliche Rollenerwartungen abzulehnen. Selbst in das eigene Handeln eingreifen zu können und eine aktive Lebensgestaltung stehen, soweit dies für das Kind möglich ist, im Vordergrund. Mädchen und Jungen müssen heute ähnliche Anforderungen bewältigen, wenn es darum geht ein konsistentes Selbstbild zu entwickeln. 

\section{Rollenbilder und -erwartungen}

Ekkehard Kloehn untersuchte geschlechtstypische Verhaltensweisen bei Männern und Frauen anhand von Ergebnissen der differentiellen Psychologie, weist aber deutlich daraufhin, dass Unterschiede im Verhalten von Männern und Frauen nur zu einem sehr geringen Bestandteil auf biologische Dispositionen zurückgeführt werden können. Außerdem sind Unterschiede innerhalb eines Geschlechts oft deutlicher, als wenn man Männer und Frauen einander gegenüberstellt. \parencite[53]{Kloehn1978}
Bei Untersuchungen von Babys kam man etwa zu den Ergebissen, dass Mädchen eher auf autitive und Buben eher auf visuelle Reize reagieren. Bei Kleinkindern zeigte sich, dass Jungen bei der Wahl des Spielzeugs und dem Umgang mit diesem eher grobmotorische Fähigkeiten einsetzten, während bei Mädchen eher ein feinmotorischer oder künstlerischer Umgang mit dem Spielzeug beobachtet werden konnte. Je älter Kinder werden, desto eher konnten Unterschiede festgestellt werden: Danach verhalten sich Mädchen tendenziell ruhiger im Spiel, legen Wert auf Harmonie und Beziehungen. Jungen spielen hingegen eher aggressiver und sind wettbewerbsorienterter. Im Erwachsenenalter drückt sich das dann in einer tendenziellen Beziehungs- und Kommunikationsorientiertheit der Frauen und einer (wieder tendenziellen) Objekt- und Sachbezogenheit der Männer aus. \parencite[74]{Kloehn1978} Dass Männern andere Eigenschaften, Interessen und Fähigkeiten als Frauen zugordnet werden, wird maßgeblich von der Umwelt und Rollenerwartungen bestimmt. So gibt es noch immer Eigenschaften, die als feminin oder maskulin, angesehen werden: Während Aggressivität, Konkurrenzorientiertheit, Aktivität, Logisches Denken oder Abenteuerlust dem typisch männlichen Gender entsprechen, stehen auf der weiblichen Seite Eigenschaften wie Sicherheitsbedürftigkeit, Harmonieorientertheit, träumerisches Denken, Passivität, Fürsorglichkeit, Schwäche oder Abhängigkeit. \parencite[175]{feldmann2006} Diese Eigenschaften und Fähigkeiten sollen Extrempole darstellen und eignen sich gut für Untersuchungen von Charakteren, sollen aber kein Abbild der Wirklichkeit darstellen.

 Bei der Analyse ausgewählter Kinderliteratur der 1990er Jahre legte Anita Schilcher besonderen Fokus auf das Verhalten, der in den Texten vorkommenden Hauptfiguren, das Familiensetting und Bewertungen, die in den Texten vorkamen. Sie kam auf folgende Ergebnisse: Traditionelle Mädcheneigenschaften, wie Passivität, Empfindlichkeit, körperliche Schwäche oder mädchentypische, unpraktische Kleidungsvorlieben werden durchgehend negativ bewertet, während eine selbstbewusste, aktive, durchsetzungsstarke Mädchenfigur als Leitbild wirkt. Auch Jungen, die ein moderneres Rollenbild und Eigenschaften wie Sensibilität, Kreativität und Kommunikationsfähigkeit, vereinen, werden bevorzugt. Auffallend ist, dass berufstätige Mütter gleichzeitig Familien- und Hausarbeit leisten und eine nahezu perfekte, alles vereinende und deswegen vielleicht sogar unrealistische Frauenrolle inne haben. Väter kommen in den meisten Texten seltener vor, da karrierebedingte Entscheidungen, die meist zu längeren Arbeitszeiten führen, öfter im Vordergrund stehen. Dadurch sind sie auch deutlich weniger ins alltägliche Familienleben eingebunden. Weiters gehen Männer kaum in Karenz und sind viel seltener geringfügig beschäftigt, was den tatsächlichen gesellschaftlichen Verhältnissen noch immer entspricht. Frauen spielen zwar durch ihre Berufstätigkeit in ehemalig reinen Männerdomänen mit, fallen aber nach der Ankunft ihres ersten Kindes in traditionelle Rollenmodelle zurück und widmen ihre Zeit in viel höherem Ausmaß als Väter (unbezahlter) Familien- und Hausarbeit, weshalb sie auch Teilzeitarbeitsmodelle  erheblich häufiger in Anspruch nehmen. Die Vermutung, dass Frauen vielfältigere, traditionelle wie moderne Eigenschaften vereinen (müssen) und Männer sich in einem weniger breiten Spektrum bewegen, wird, in der bereits analysierten modernen Kinderliteratur, bestätigt.
        

Allerdings bedeuten die geschlechtsspezifischen Rollenentwürfe der in der Literatur vorkommenden Figuren nicht, dass der/die junge LeserIn diese unvermittelt verinnerlichen. Sie werden natürlich (vorwiegend unbewusst) wahrgenommen, aber vor dem jeweiligen kindlichen Erfahrungshintergrund  in der Gedankenwelt konstruiert. Prädispositionen von Mädchen und Buben beeinflussen folglich auch die Akzeptanz oder Ablehnung eines Lesestoffs. Wenn Kinder also nicht gezwungen sind, sich mit einem bestimmten Lektüreangebot zu beschäftigen, hängt die Leseentscheidung von Belohnungen ab, die erwartet werden. Diese sind intrinsischer Natur und können auf emotionaler, sozialer oder kognitiver Ebene erfolgen: Der Wunsch, bei Themen, die gerade \emph{in} sind, mitreden zu können, kann die Motivation ein Buch zu lesen ebenso beeinflussen wie das Bedürfnis dabei die eigene Fantasie anzuregen und in andere Rollen zu schlüpfen, den persönlichen Wissensdurst zu stillen oder einfach Spaß bei dieser Form der Unterhaltung zu haben. \parencite[547\psq]{Kuhn2010}

% \subsection{Darstellungen der Hauptfiguren}

% Bei der Analyse ausgewählter Kinderliteratur der 1990er Jahre legte Anita Schilcher besonderen Fokus auf das Verhalten, der in den Texten vorkommenden Hauptfiguren, das Familiensetting und Bewertungen, die in den Texten vorkamen. Sie kam auf folgende Ergebnisse: Traditionelle Mädcheneigenschaften, wie Passivität, Empfindlichkeit, körperliche Schwäche oder mädchentypische, unpraktische Kleidungsvorlieben werden durchgehend negativ bewertet, während eine selbstbewusste, aktive, durchsetzungsstarke Mädchenfigur als Leitbild wirkt. Auch Jungen, die ein moderneres Rollenbild und Eigenschaften wie Sensibilität, Kreativität und Kommunikationsfähigkeit, vereinen, werden bevorzugt. Auffallend ist, dass berufstätige Mütter gleichzeitig Familien- und Hausarbeit leisten und eine nahezu perfekte, alles vereinende und deswegen vielleicht sogar unrealistische Frauenrolle inne haben. Väter kommen in den meisten Texten seltener vor, da karrierebedingte Entscheidungen, die meist zu längeren Arbeitszeiten führen, öfter im Vordergrund stehen. Dadurch sind sie auch deutlich weniger ins alltägliche Familienleben eingebunden. Weiters gehen Männer kaum in Karenz und sind viel seltener geringfügig beschäftigt, was den tatsächlichen gesellschaftlichen Verhältnissen noch immer entspricht. Frauen spielen zwar durch ihre Berufstätigkeit in ehemalig reinen Männerdomänen mit\footnote{Gerade in höheren Positionen, sowie in naturwissenschaftlich- technischen Gebieten, sind wenig Frauen zu finden. Diese Tätigkeitsbereiche sind im Allgemeinen von sehr gutem Verdienst gekennzeichnet, während soziale (eher weiblich dominierte) Berufe vergleichsweise unterbezahlt sind. Dass die unterschiedliche Verteilung von Männern und Frauen auf die einzelnen Berufsgruppen, nicht der einzige Grund, für geschlechtsabhängige Lohndifferenzen sind, sei hier nur erwähnt.}, fallen aber nach der Ankunft ihres ersten Kindes in traditionelle Rollenmodelle zurück und widmen ihre Zeit in viel höherem Ausmaß als Väter (unbezahlter) Familien- und Hausarbeit, weshalb sie auch Teilzeitarbeitsmodelle  erheblich häufiger in Anspruch nehmen. Die Vermutung, dass Frauen vielfältigere, traditionelle wie moderne Eigenschaften vereinen (müssen) und Männer sich in einem weniger breiten Spektrum bewegen, wird, in der bereits analysierten modernen Kinderliteratur, bestätigt.
	

% Allerdings bedeuten die geschlechtsspezifischen Rollenentwürfe der in der Literatur vorkommenden Figuren nicht, dass der/die junge LeserIn diese unvermittelt verinnerlichen. Sie werden natürlich (vorwiegend unbewusst) wahrgenommen, aber vor dem jeweiligen kindlichen Erfahrungshintergrund  in der Gedankenwelt konstruiert. Prädispositionen von Mädchen und Buben beeinflussen folglich auch die Akzeptanz oder Ablehnung eines Lesestoffs. Wenn Kinder also nicht gezwungen sind, sich mit einem bestimmten Lektüreangebot zu beschäftigen, hängt die Leseentscheidung von Belohnungen ab, die erwartet werden. Diese sind intrinsischer Natur und können auf emotionaler, sozialer oder kognitiver Ebene erfolgen: Der Wunsch, bei Themen, die gerade \emph{in} sind, mitreden zu können, kann die Motivation ein Buch zu lesen ebenso beeinflussen wie das Bedürfnis dabei die eigene Fantasie anzuregen und in andere Rollen zu schlüpfen, den persönlichen Wissensdurst zu stillen oder einfach Spaß bei dieser Form der Unterhaltung zu haben. \parencite[547\psq]{Kuhn2010}



% Nach den bisherigen theoretischen Ausführungen ist es Aufgabe dieses Teilbereiches die methodischen Herangehensweisen rund um das Thema Kinderbuch vorzustellen. Schlussendlich beschäftigen wir uns hier mit den grundsätzlichen Fragen: Gibt es Mädchen- und Bubenbücher denn überhaupt und wenn ja, wie unterscheiden sie sich. Regalreihen mit den Hinweistafeln \enquote{Mädchen 8--12} sind ein praktisches Beispiel, wie bestimmte Kinderbücher einer Altersgruppe und vor allem einem Geschlecht zugeordnet werden.  Die dabei wohl ergiebigste und spannendste Forschungsmethode stellt die Inhaltsanalyse dar, die besonders geeignet ist um schriftliche und bildliche Darstellungen vergleichen zu können. Es liegt daher nahe die Vor- und Hinterseiten -- das Buchcover -- gesondert von den Inhalten der zahlreichen Geschichten, Abenteuer und Erzählungen zu untersuchen.

% Konzentriert man sich auf den zuerst genannten Punkt einer unterschiedlichen \emph{Aufmachung} sind die Arbeiten von Erving Goffman, der bereits früh mit Untersuchungen von unterschiedlichen bildlichen Darstellungen der Geschlechter begonnen hat, eine fruchtbar Anregung. Goffman verwendete Werbegraphiken um aufzuzeigen, in welchen Rollen Männr und Frauen dargestellt werden. Seiner Interpretation nach machen immer wiederkehrende Konstellationen und dargestellte Situationen es möglich, Aussagen über Rollenerwartungen zu tätigen. Reklamebilder sind für ihn eine \textcquote[104]{Goffman1981}{verallgemeinernde Darstellung einer heimlichen Thematik der Geschlechter, vor allem des weiblichen Geschlechts.} Auch wenn es sich bei den Abbildungen um keine reale Situation handelt, sondern diese Bilder rein zum Zweck der Werbung inszeniert werden, kann festgehalten werden, dass die Reklame \textcquote[111]{Goffman1981}{von den Betrachtern als gar nicht ungewöhnlich, als etwas \enquote{ganz Natürliches} aufgefaßt wird.} Verknüpfen wir diese Erkenntnis mit unseren Buchcovern, so kann davon ausgegangen werden, dass Kinder die bildlich dargestellten Geschlechterrollen -- egal ob Geschlechterklischee oder nicht -- als gängig und konventionell interpretieren. Goffman legte zur Analyse der Reklame einige Indikatoren fest, wie beispielsweise die \emph{relative Größe}. Wollen Fotografen eine Person auf einem Bild besonders mächtig bzw. autoritär erscheinen lassen, so wird die Körpergröße gerne verwendet um den erwünschten Eindruck zu hinterlassen. Untergebene werden oftmals sitzend, hockend oder schlichtweg kleiner dargestellt. Dies könnte auch als Indikator für Kinderbuchcover übernommen werden, aber auch die restlichen von Goffman festgehaltenen Merkmale \enquote{weibliche Berührung}, \enquote{Rangordnung nach Funktion}, \enquote{Familie}, \enquote{Rituale der Unterordnung} und \enquote{zulässiges Ausweichen} können brauchbare Elemente für Buchdeckel-Analysen enthalten. 

% Eine etwas andere Herangehensweise verfolgt eine Studie aus dem Jahre 1988 von \citeauthor{Schmerl1988}. Die Studie behandelt inhaltsanalytisch eine repräsentative Stichprobe von in Kindergärten und Vorschulen häufig gelesenen Bilderbüchern. Da Bilderbücher oftmals eine Kombination von Bild und Text sind, die in Bezug zu einander stehen, kann ein Vergleich zu einem Buchcover gezogen werden. Auch der Umstand, dass der bildlichen Darstellung ein besonderer Stellenwert zugesprochen wird, ähnelt einem Cover eines Kinderbuches sehr stark. Die Bücher wurden mittels einer Bestandsaufnahme von 29 Kindergärten der Umgebung ausgewählt, wobei nach einigen Ausschlussverfahren 52 Werke übrig blieben, die in späterer Folge analytisch auf 15 Inhaltskategorien untersucht wurden. Passenderweise wurden Texte und Bilder separat analysiert und ausgewertet. \parencite[133\psq]{Schmerl1988}

% Beispielsweise wurden Geschlechterproportionen der handlungstragenden Figuren analysiert, was anhand des Buchcovers meistens kein Problem darstellt: Es konnten in dem Kinderbilderbücher-Sampling der Studie von \citeauthor{Schmerl1988} insgesamt 62 Hauptfiguren identifiziert werden, wovon jedoch nur 55 einem Geschlecht zugeordnet wurden. Davon wurden 17 (30,9\%) dem weiblichen und 38 (69,1\%) dem männlichen Geschlecht zugeschrieben. Diese Proportionen ähneln dem allgemeinen Geschlechterverhältnis aller Charaktere sehr stark. Auch bei den weiteren subkategorischen Vergleichen sind ähnliche Tendenzen festzuhalten. So ist der Unterschied bei Kindern um einiges geringer als bei erwachsenen Figuren (Frauen zu Männern = 1:3,8; Mädchen zu Buben = 1:1,4) und auch bei dem Verhältnis Bild zu Text zeigen die verschriftlichten Darstellungen eine geringere Differenz auf. 
% Außerdem wurde der Frage nachgegangn, ob Autorinnen ein ähnliches Geschlechterverhältnis wie ihre männlichen Kollegen in ihren Werken wiedergeben. 
% Auch wenn die Unterschiede zwischen bildlichen und schriftlichen Darstellungen etwas variieren, konnte einheitlich festgehalten werden, dass Autorinnen um ein ausgewogenes Geschlechterverhältnis bemüht waren, während ihre männlichen Kollegen in ihren Charakteren ein extremes Ungleichgewicht bei den Geschlechtern vorwiesen. Diese Erkenntnis macht es notwendig den Autor bzw. die Autorin bei einer Analyse eines Kinderbuches festzuhalten und die Inhalte ihrer Werke auf den Gebrauch von Geschlechterrollen zu untersuchen. 

% Als klassisches Beispiel einer Inhaltsanalyse könnte die bereits oben beschriebene Studie von \citeauthor{Schmerl1988} genannt werden. So erheben \citeauthor{Schmerl1988} beispielsweise die unterschiedliche Darstellung von Männern und Frauen bei Beruf, Familienarbeit und Kommunikation, aber auch Differenzen bei Emotionalität, Bedürfnissen und Verhalten (aggressiv, passiv und/oder altruistisch). \parencite{Schmerl1988} 

% Eine Möglichkeit wäre, die aktive Beschäftigung mit der Umwelt ins Blickfeld zu rücken. Dabei wurde in der vorligenden Studie zwischen „körperlichen“ (suchen, sammeln, essen, ernten, angeln, tanzen, sich verstecken, fotografieren) und „geistigen“ (denken, überlegen, träumen, sich erinnern, Ideen haben, etwas wissen, beschließen) Aktivitäten unterschieden. Die Proportionen waren besonders bei den körperlichen Betätigungen -- klar verteilt. Männliche Figuren wurden im Durchschnitt dreimal so oft wie weibliche Figuren bei derartigen Aktivitäten dargestellt.  Bei geistigen Tätigkeiten fallen diese Zahlen etwas ausgeglichener aus. 
% Eigenschaften lassen sich am besten mittels Personenbeschreibungen ermitteln.  Unter äußeren Eigenschaften wurden körperliche Beschreibungen verstanden, so wurden Mädchen als klein, zierlich, anmutig und/oder leichtfüßig beschrieben und Frauen als schön, reizend, jung und/oder alt. Das männliche Geschlecht wurde äußerlich nur sehr wenig beschrieben, etwa als alt oder stark. Bei Jungen fehlte die Beschreibung äußerer Eigenschaften sogar gänzlich. Bei den inneren Eigenschaften, also Beschreibungen von Charakterzügen, wurde zwischen positiven und negativen Eigenschaften unterschieden.  \textcquote[145]{Schmerl1988}{Das absolute wie relative Überwiegen negativer Eigenschaftsbeschreibungen weiblicher Figuren ist vor dem Hintergrund der sonstigen durchgehenden Unterrepräsentierung von Mädchen und Frauen als besonders diskriminierende Konstellation zu bewerten.} %\parencite[144\psq]{Schmerl1988}

% Als eine Beispielarbeit für modernere Analysen dient die Diplomarbeit von Carl Pick von 2009, die sich mit der seriellen Narration beschäftigt und dabei die Bedeutung von unterschiedlichen Handlungssträngen beschreibt. Es gibt zwei Arten von Handlungssträngen, die am besten mittels ihrer Verwendung in seriellen Narrationen erklärt werden. Unter seriellen Narrationen gibt es fünf Typen: Serie, Reihe, Fortsetzungsroman, Zyklen und Mehrteiler. \parencite[12--18]{Pick2009} Oftmals synonym verwendet sind sie medienwissenschaftlich von einander zu unterscheiden. Eines der Hauptmerkmale um sie voneinander trennen zu können, ist die unterschiedliche Verwendung der Handlungsstränge. Es gibt zwei Arten von Handlungssträngen. \parencite[23\psq]{Pick2009}

% \pagebreak % um Aufzählung nicht zu zerreißen.
% 				\begin{itemize}
% 					\item Übergeordnete Handlungsstränge
% 					\item Untergeordnete Handlungsstränge
% 				\end{itemize}

% Übergeordnete Handlungsstränge werden oftmals auch als Hauptkonflikte oder Hauptziel bezeichnet.Untergeordnete Handlungsstränge wären somit der Überbegriff für sämtliche Abenteuer, Erlebnisse die ein Protagonist durchleben muss um sein Hauptziel zu erreichen oder den Konflikt zu lösen. Untergeordnete Handlungsstränge werden oft auch folgenimmanente Handlungsstränge genannt. Sie müssen aber nicht immer auf genau ein Buch oder Kapitel einer seriellen Erzählung bezogen sein. Es kommt auch vor, dass untergeordnete Handlungsstränge über mehrere Teile beschrieben werden oder auch mehrere solcher Handlungsstränge in einem Teilwerk vorkommen. \parencite[23\psq]{Pick2009}
				
% Diese Unterteilung in übergeordnete und folgenimmanente (bzw. untergeordnete) Handlungsstränge, kann vor allem bei seriellen Erzählungen verwendet werden und birgt daher für die inhaltliche Analyse von Kinderbüchern eine große Chance, da viele von Kindern gern gelesene Bücher Serien, Reihen, etc. sind. Versucht und damit geklärt, gehört vor allem auch, ob diese Unterteilung auch für Einzelwerke verwendet werden kann, da auch hier vermutet wird, dass es Haupt- und Nebenstränge gibt.
				
% Im Hinblick auf unsre Forschungsfrage ist diese Herangehensweise besonders interessant und kann zeigen ob ein Geschlecht verstärkt in inem der beiden Handlungsstränge vorzufinden ist. Um jedoch eine Aussage über die Tragweite einer unterschiedlichen Ausprägung von weiblichen oder männlichen Charakteren in Haupt- und Nebensträngen erzielen zu können, muss erklärt werden, ob Leser sich mit dem Hauptprotagonisten oder mit Charakteren des eigenen Geschlechtes identifizieren. Um die Annahme zu berücksichtigen, dass ein Leser sich vorranging mit dem Hauptprotagonisten identifiziert, auch wenn dieser womöglich nicht dem eigenen Geschlecht angehört, sollen zusätzlich die Handlungsstränge zwischen Mädchen- und Bubenbüchern verglichen werden. Etwaige Unterschiede beim Aufbau, Ablauf, Inhalt und/oder Ausgang der Handlungsstränge könnten somit Aufschluss über die Bevorteilung eines Geschlechts zeigen.

% Die Inhaltsanalyse bietet zahlreiche Möglichkeiten im Feld der Kinderbücher zu aussagekräftigen Erkenntnissen zu kommen. Oft wurde diese Möglichkeit jedoch nicht genutzt und nur an der Oberfläche gekratzt. Fest steht dennoch, dass sich Kinderbücher besonders gut dazu eignen, Geschlechterrollen aufzuzeigen und weiter zu vermitteln. Wie wir festgestellt haben wurden weibliche, im Vergleich zu männlichen Beschreibungen oft benachteiligt und unwahr präsentiert. Auch konnte gezeigt werden, dass Autorinnen sich viel mehr um eine ausgewogene Präsenz von weiblichen zu männlichen Figuren bemühten als ihre männlichen Berufskollegen. Zusammenfassend darf gesagt werden, dass Unterschiede festgestellt wurden und in der moderneren Kinderliteratur auffallend oft versucht wird, diese Klischees nicht länger zu verwenden. 

% Der Schlüssel zu einer aussagekräftigen Analyse liegt wohl auch in der Offenheit der Forscher, neue Wege zu gehen. Einer dieser möglichen Ansätze wäre eine stärkere Gewichtung der HauptprotagonistenInnen und die reine Konzentration auf Geschlechter  zu erweitern. 

\chapter{Forschungsdesign}

Im Rahmen unseres Forschungspraktikums mit dem Titel
Geschlechtsidentität und Geschlechterrollen bei Kindern, möchten wir
untersuchen, wie sich Kinderbücher, je nachdem ob sie eher von Mädchen
oder Buben gelesen werden, unterscheiden und welche Merkmale für
unterschiedliche Lesepräferenzen entscheidend sein können.

\section{Definitionen}

Mädchebuch - Buch das von einer Mehrheit an weiblichen Lesern konsumiert
wird.

Bubenbuch - Buch das von einer Mehrheit an männlichen Lesern konsumiert
wird.

\section{Forschungsfragen}

\begin{itemize}
\item
  Lesen Mädchen andere Bücher als Buben?
\item
  Weisen von Mädchen favorisierte Bücher im Vergleich zu denen von Buben
  inhaltliche Unterschiede auf?
\item
  Kann man ohne über den Inhalt eines Buchs Bescheid zu wissen, auf das
  Verhältnis von Leserinnen zu Lesern schließen?
\end{itemize}

Die Beantwortung dieser drei Fragen ist das genannte Ziel dieser
Untersuchung, erst in späterer Folge sollen diese mit Theorien in
Verbindung gebracht werden, die mögliche Konsequenzen auf die
Rekonstruktion von Geschlechteridentitäten und Geschlechterrollen
aufzeigen können, um schließlich dem Titel und Anspruch dieses
Forschungsprojektes gerecht zu werden.

\section{Hypothesen}

\paragraph{Lesen Mädchen andere Bücher als Buben?}

\begin{itemize}
\item
  Mädchen lesen andere Bücher als Buben.
\item
  Mädchen lesen mehr Bücher als Buben.
\item
  Mädchen lesen lieber über andere Themen als Buben.
\end{itemize}

\paragraph{Weisen von Mädchen favorisierte Bücher im Vergleich zu denen
von Buben inhaltliche Unterschiede auf?}

\begin{itemize}
\item
  Von Mädchen favorisierte Bücher unterscheiden sich anhand der
  Darstellung des sozialen Geschlechts von jenen der Buben.
\item
  Von Mädchen favorisierte Bücher unterscheiden sich anhand inhaltlicher
  Merkmale von jenen der Buben.
\item
  Mädchen neigen dazu eher Bücher zu lesen, in denen der
  Beziehungsaspekt angesprochen wird, während Buben Literatur
  bevorzugen, die eher von Konkurrenz geprägt sind oder es ein großes
  \emph{Ziel} zu erreichen gibt.
\end{itemize}

\paragraph{Kann man ohne über den Inhalt eines Buchs Bescheid zu wissen,
auf das Verhältnis von Leserinnen zu Lesern schließen?}

\begin{itemize}
\item
  Mädchenbücher können auch ohne Kenntnisse über den Inhalt als solche
  erkannt werden.
\end{itemize}

Um all diese Hypothesen beantworten zu können, ist ein geplantes
Vorgehen und der Einsatz der jeweils passenden Erhebungsmethode
notwendig.

\section{Erhebungsmethoden}

\subsection{Fragebogenerhebung}

\paragraph{Mädchen lesen andere Bücher als Buben.}

Mit Hilfe einer Bücherliste soll erhoben werden, welche Bücher von den
Schülern und Schülerinnen der dritten und vierten Klassen von
außgewählten Grazer Volksschulen gelesen werden. Diese Bücherliste wird
aus Bestsellerlisten (Amazon), Bibliotheksleihverzeichnissen und
Expertisen von Büchereimitarbeiter\_innen zusammengestellt. Als
Kontrolle wird eine offene Frage nach den Lieblingsbüchern hinzugefügt.
Diese wird im Fragebogen vor die Liste platziert um den Einfluss der
Bücherliste auf die Antwortgebung zu vermeiden.

\paragraph{Mädchen lesen mehr Bücher als Buben.}

Hier soll dieselbe Bücherliste dazu dienen eine Aussage über die
Lesefreudigkeit des jeweiligen Geschlechts zu liefern. Dafür werden die
Nennungen an gelesenen Büchern ausgezählt und verglichen.

\paragraph{Mädchen lesen über andere Themen als Buben.}

Eine weitere kurze Liste mit Aufzählungen beliebter Thematiken,soll
Aufschluss darüber geben, welche Thematiken von dem jeweiligen
Geschlecht bevorzugt werden.

\subsection{Inhaltsanalyse}

\paragraph{Von Mädchen favorisierte Bücher unterscheiden sich anhand der
Darstellung des sozialen Geschlechts von jenen der Buben.}

Den Hauptprotagonisten der jeweiligen Werke sollen mithilfe einer Liste
von 13 geschlechterstereotypen Eigenschaftspaaren ein Genderwert
zugeteilt werden um eine Aussage über die unterschiedliche Darstellung
des sozialen Geschlechts in den jeweils von einem Geschlecht bevorzugten
Kinderbüchern zu treffen.



      \ctable[
      %  cap    = ,
        caption = {Geschlechterstereotype},
        label   = stereo ,
        % pos   = htp,
      %  width    = \textwidth
      ]{ll}{
        \tnote{Quelle: \inparencite[175]{feldmann2006}}
      }{                  
      \FL {\small weibliche Stereotype} &  {\small männliche Stereotype}
      \ML unterwürfig           & dominant
      \NN abhängig              & unabhängig
      \NN harmonieorientiert/kooperativ & konkurenzorientiert
      \NN passiv                & aktiv/tatkräftig
      \NN sicherheitsbedürftig  & abeteuerlustig/unternehmenslustig
      \NN sanft                 & aggresiv
      \NN furchtsam             & kühn/mutig
      \NN schwach               & stark/kräftig
      \NN träumerisch           & rational/realistisch
      \NN weichherzig/milde     & grausam/hartherzig/streng
      \NN fürsorglich/mütterlich  & egoistisch
      \NN einfühlsam/emotional/gefühlvoll & emotionslos
      \NN unlogisch             & logisch denkend \LL
      }

\paragraph{Von Mädchen favorisierte Bücher unterscheiden sich anhand
inhaltlicher Merkmale von jenen der Buben.}

Die Kinderbücher sollen daraufhin untersucht werden, ob sie Alltags-
oder Abenteuergeschichten darstellen. Es wird vermutet, dass Buben
Abenteuergeschichten bevorzugen, Mädchen hingegen zu Alltagsgeschichten
tendieren. Sollte sich diese Annahme bewahrheiten, sollen weitere
inhaltliche Merkmale auf ihr Vorkommen in Alltags- und
Abenteuergeschichten untersucht werden.

Folgende inhaltliche Merkmale werden hierbei auf Vorkommen in und ihren
Zusammenhang mit Alltags- und Abenteurgeschichten erhoben und
analysiert:

\begin{itemize}
\item
  Quests - Präsenz eines Hauptzieles das durch das Lösen von Rätseln
  erreicht wird.
\item
  Phantastische Elemente - Präsenz von irrealen Wesen, Orten oder
  Handlungen.
\item
  Innerer Monolog - Wiedergabe der Gedankenwelt des Protagonisten.
\item
  Growing Up - Präsenz eines Reifeprozesses.
\end{itemize}

Unterstützend sollen qualitative Inhaltsanalysen spezifischer
Textbeispiele zugunsten einer besseren Verständlichkeit Verwendung
finden.

\subsection{Statistische Verfahren}

\paragraph{Mädchenbücher können ohne Kenntnisse über den Inhalt als
solche erkannt werden.}

Mit Hilfe statistischer Verfahren soll hinterfragt werden, durch welche
vom Inhalt der Bücher völlig unabhängigen Merkmale eine Favorisierung
von spezifischen \emph{Geschlechterbüchern} erklärt werden kann. Ziel
ist es ein Modell zu entwerfen, das den Einfluss dieser
inhaltsunabhängigen Merkmale auf die Wahl eines Buches durch ein
Geschlecht bestmöglich erklärt.

Folgende inhaltsunabhängige Variablen sollen zu diesem Zweck erhoben
werden:

\begin{itemize}
\item
  Helligkeit des Deckblattes
\item
  Geschlecht der Autorin/des Autors
\item
  Figurenanzahl am Cover
\item
  Dicke des Buches
\item
  Geschlecht der Titelfigur
\item
  Altersempfehlung
\end{itemize}


    \ctable[
      caption = {Variablen},
      label   = var,
      pos   = htp,
      width = \textwidth,
    ]{l>{\raggedright}X}{}{
      \FL   \small Variable             & \small Erhebungsmerkmal
      \ML   Geschlecht der Hauptfigur   & Klappentext, Buchtitel
      \NN   Geschlecht von Autorin/Autor& Name
      \NN   Helligkeit                  & Mittelwert des Histogramms des Covers
      \NN   Länge des Titels            &   Buchstabenanzahl
      \NN   Dicke                       &   Anzahl der Seiten
      \LL 
    }

\subsection{Überblick der Erhebungsmethoden}

%\singlespacing

\ctable[       
    caption = Zuordnung von Inhalten zu Methoden, % Tabellenüberschrift       
    label   = zuo,       
    cap   = Zuordnung: Inhalte--Methoden, % Kurztitel f. Tabellenverz.  
    pos   = tbp, % Positon d. Tabelle       
    width   = \textwidth, % Tab.br. \textwidth, \columnwidth     
    ]
    {>{\raggedright}Xcccc}{  % Aufteilung d. Spalten
  % Fußnoten     
}{              % Hier beginnt die Tabelle
\FL \small Fragestellung  & \begin{sideways}\small Fragebogen\end{sideways}&
\begin{sideways}\small Inhaltsanalyse\end{sideways}& \begin{sideways}\small
quant. Inhaltsa.\end{sideways}& \begin{sideways}\small Qual.
Inhaltsa.\end{sideways}        
\ML Gibt es Unterschiede bei den Lesepräferenzen von Mädchen und Buben? (Bubenbücher, Mädchenbücher) & x &   &   &        
\NN[0.5em] Gibt es Unterschiede zwischen Mädchen- und Bubenbücher? (Erscheinung, Inhalt, Aufgeben, \ldots)                         &
& x & x & x       
\NN[0.5em] Kann man Bubenbücher anhand von \emph{oberflächlichen} Merkmalen von Mädchenbüchern unterscheiden? (Farben, Themen, Umfang, Autorengeschlecht, \ldots)
&   & x &   &
\NN[0.5em] Kann man Bubenbücher anhand von inhaltlichen Merkmalen von
Mädchenbüchern unterscheiden? (Schreibweise, Stereotype, \ldots)
&   &   & x &        
\NN[0.5em] Sind Unterschiede tatsächlich in den bevorzugten Büchern auffindbar? (Rollensettings, Lösungen von Aufgaben, \ldots)
&   &   &   & x       
\NN[0.5em] Gibt es Bücher die die Einteilungen (Mädchen- und Bubenbuch) besonders gut repräsentieren? (Welche)                         
& & x & x & x       
\LL }     

%\onehalfspacing

  \subsection{Zeitplan}
    Die Studie wird innerhalb der Lehrveranstaltung durchgeführt. Die Lehrveranstaltung dauert vom Sommersemester 2012 bis zum Wintersemester 2012/13. Für eine Überblick siehe Tabelle~\ref{zeit}.
    \singlespacing
    \ctable[
      caption = Zeitplan,
      label   = zeit,
      cap   = Zeitplan,
      pos   = htp,
      width   = \textwidth,
    ]{l>{\raggedright}X}{}{
      \FL     Mai & Literaturstudium, Kontaktaufnahme zu Schulen, Konstruktion der Erhebungsbögen, Leitfäden für Fokusgruppengespräche, Kontaktaufnahme zu Experten.
      \NN[0.5em] Juni & Durchführung der Erhebungen und Fokusgruppengespräche, Zusammenfassung von Literatur.
      \NN[0.5em]Juli--Spt.  & Dateneingabe und Analyse der Erhebungsfragebögen, Expertengespräche, Umschlaganalyse, Tiefenanalyse, Analyse der Interviews.
      \NN[0.5em] Oktober  & Abschluss der Analyse, Ergänzende Schritte.

      \NN[0.5em] November & Anfertigung der Erstfassung des Forschungsbericht.
      \NN[0.5em] Dez.--Jän.& Endfassung anfertigen, Präsentation erstellen.
      \NN[0.5em] Februar  & Ergebnisse den Beteiligten zukommen lasse, Projektabschluss.
      \LL
    }
    \onehalfspacing


\chapter{Forschungsdesign}

\section{Definitionen}

\begin{description}
\item[Geschlecht]
Unter Geschlecht verstehen wir, ob jemand als Frau oder Mann, als
Mädchen oder Bub wahrgenommen wird. Wenn wir von Geschlecht reden,
verwenden wir die Begriffe \emph{weiblich}, \emph{männlich},
\emph{Mädchen}, \emph{Bub}, \emph{Frau} oder \emph{Mann}.
\item[Gender]
Unter Gender verstehen wir das vom Geschlecht abhängige Verhalten. Wenn
ein Mann einer Frau den Vortritt lässt dann macht er \emph{Gender}.
\item[Geschlechter Stereotype]
Unter Geschlechter Stereotypen verstehen wir kulturelle Vorstellungen
über das, was typisch Frau bzw. typisch Mann ist. Wenn wie von
Geschlechter Sterotypen reden verwenden wir folgende Begriffe:
\emph{feminin}, \emph{maskulin}. (TabelleXY)
\item[Geschlechterverhältnis]
Unter Geschlechterverhältnis verstehen wir das Verhältnis von weiblichen
zu männlichen Personen. In unserem Fall hauptsächlich das Verhältnis von
Leserinnen zu Lesern. Ist der Anteil der weiblichen Personen höher,
handelt es sich um ein weibliches Geschlechterverhältnis. Ist der Anteil
der Buben oder Männer größer, dann ist das Geschlechterverhältnis
männlich. Sonst ist es weder weiblich noch männlich ist es neutral.
(siehe: w/m-Faktor)
\item[Leserinnen und Leser]
Wenn wir von Leserinnen und Lesern sprechen, meinen wir in dieser Arbeit
nur die Untersuchte Grundgesamtheit, das heißt Leserinnen und Leser in
Österreich, die die 3. oder 4. Schulstufe besuchen. Leserinnen
\emph{oder} Leser \emph{ohne} die jeweils andere Form verwendet, handelt
es sich nur um Mädchen oder Buben.
\end{description}

\section{Fragestellungen}

Welche Unterschiede in Kinderbüchern hängen mit dem
Geschlechterverhältnis zusammen?

\subsection{Ebenen}

\begin{itemize}
\item
  Unterschiede bei den Hauptfiguren.
\item
  Unterschiede im Setting der Geschichten.
\end{itemize}

\section{Hypothesen}

\begin{description}
\item[H1]
Es gibt einen Zusammenhang zwischen dem Geschlecht der Hauptfigur und
dem Geschlecht der Lesenden.

\begin{description}
\item[H1.1]
Je größer der Anteil an weiblichen Hauptfiguren, desto größer ist der
Anteil an Leserinnen.
\item[H1.2]
Je größer der Anteil an männlichen Hauptfiguren, desto größer ist der
Anteil an Lesern.
\item[H1.3]
Der Anteil der Hauptfiguren, die sich nicht eindeutig zuzuordnen lassen,
hat keinen Einfluss auf das Geschlechterverhältnis der Lesenden.
\item[H1.4]
Das Geschlecht der Hauptfiguren hängt mit Anzahl der Leser stärker
zusammen als mit der Anzahl der Leserinnen.
\end{description}
\item[H2]
Es gibt einen Zusammenhang zwischen dem Geschlechterverhältnis der
Lesenden mit dem Verhältnis zwischen femininen und maskulinen Verhalten
der Hauptfiguren.

\begin{description}
\item[H2.1]
Je größer der Anteil an Leserinnen, umso femininer verhalten sich die
Hauptfiguren.
\item[H2.2]
Der Zusammenhang zwischen Geschlechterverhältnis der Lesenden und dem
mit dem Verhältnis zwischen femininen und maskulinen Verhalten der
Hauptfiguren lässt sich nicht durch das Geschlecht der Hauptfiguren
erklären.
\end{description}
\end{description}

\chapter{Unterschiedliche Lesepräferenzen von Mädchen und Buben}

Wie wir bereits im Literaturteil erarbeitet haben, beeinflussen Bücher
neben vielen anderen Sozialisationsfaktoren auch die Entwicklung von
Gender. Was Mädchen oder Buben lesen, wirkt in gewisser Weise auf sie
ein, gleich wie etwa bestimmte Erwartungen der Eltern, der Schule oder
von Freunden. Und wenn Buben und Mädchen tatsächlich Unterschiedliches
lesen, mit anderem Spielzeug spielen oder andere Filme ansehen, dann
kann das unter Umständen auch dazu beitragen, dass traditionelle
Rollenbilder stabilisiert werden. Wir nehmen an, dass es Unterschiede im
Leseverhalten gibt, die neben vielen anderen Faktoren auf die
Geschlechterrollenbildung von Kindern Einfluss nehmen, gleich wie sich
bereits vorhandene \emph{Rollenspezifika} umgekehrt auf die
Lesepräferenzen auswirken können. Natürlich bleiben Bücher, die von
beiden Geschlechtern in gleichem Ausmaß gelesen werden, nicht von
unserer Fragestellung ausgeschlossen: Können solche \emph{neutralen}
Kinderbücher eher dazu beitragen typische Geschlechterverhältnisse
abzubauen oder bieten sie einfach mehr unterschiedliche
Identifikationsmöglichkeiten für den Leser oder die Leserin? Wir nehmen
also an, dass bestimmte Bücher eher Mädchen oder Buben ansprechen,
während andere, vom geschlechtsspezifischen Leseverhalten her, nicht
zuordenbar sind. Wenn sich diese Annahmen bestätigen, wollen wir in den
weiteren Schritten herausfinden, auf welche Faktoren diese Differenzen
zurückgeführt werden können.

Wenn es Bücher gibt, die eindeutig von Mädchen oder eben von Buben
bevorzugt gelesen werden, dann haben wir natürlich bestimmte
Vermutungen, die den Inhalt oder die äußerliche Gestaltung betreffen,
aber auch über die Charakterzüge der Hauptfigur. Hier erwarten wir
typische Kennzeichen: Während Buben eher Bücher lesen, deren Geschichten
sich beispielsweise um Abenteuer oder Fußball drehen, sprechen
Pferdebücher oder Geschichten über Prinzessinnen eher Mädchen an. Ist
das Geschlecht des Protagonisten für die Leseentscheidung eines Buches
ausschlaggebend, oder ist es die Covergestaltung? Ist ein deutlicher
Schwerpunkt auf Beziehungen und Selbstreflexion ein Kennzeichen für ein
Mädchenbuch und greifen Buben eher zu Heldenepen? Kinder entscheiden
meist nicht nur von sich heraus, was sie lesen: was angeboten wird, wird
stark von der Werbeindustrie bestimmt, die natürlich von klassischen
Einteilungen und Käufergruppen lebt. Außerdem spielen Vorlieben von
Freunden und Freundinnen bzw. Schulkolleginnen und -kollegen, Geschenke
von Verwandten oder der Bücherbestand von älteren Geschwistern eine
Rolle.

Wir wollen jedoch nicht analysieren wie ein Kind genau zu diesem oder
jenem Buch kommt und welche Faktoren die Auswahl beeinflussen und haben
auch keine Möglichkeiten Prozesse zu untersuchen, ob und wie Bücher,
deren Handlungsstränge oder das Verhalten der Hauptfiguren Auswirkungen
auf geschlechtsspezifisches oder rollenunkonformes Handeln der Kinder
haben. Ohne den Anspruch zu erheben, mit unseren Mitteln den
Zusammenhang vom Lesen bestimmter Bücher und dem „Doing Gender`` der
Kinder feststellen zu können, soll das Bestehen eines solchen, als
Vermutung, nicht verworfen werden. Das Wesentliche besteht nun darin,
sich selbst ein Bild über die Literatur zu machen, die Kinder
konsumieren und unsere Hypothesen über die Präferenzen eines Geschlechts
zu überprüfen.

\section{Erhebung der Lesepräferenzen anhand einer Fragebogenanalyse}

Um herauszufinden, was Buben und Mädchen lesen, liegt eine
Fragebogenanalyse am nächsten. Unsere Stichprobe bildeten
Volksschulkinder der dritten und vierten Klassen in Graz. Die Schulen,
die sich daran beteiligt haben, waren die ``VS Bertha von Suttner'',die
``VS Afritsch'' (davon 2 weitere Klassen am Standort Rosenberggürtel),
die ``VS Engelsdorf'', die ``VS Leopoldinum'', die ``VS Mariatrost'' und
die Prvatschule der``VS Ursulinen''.

Zur Erstellung des Fragebogens muss hinzugefügt werden, dass wir
zusätzlich zu einer offenen Frage (\emph{Was ist dein Lieblingsbuch?})
eine Liste mit Büchern, von denen wir annahmen, dass sie häufig gelesen
werden, zum Ankreuzen verwendeten und noch eine weitere geschlossene
Frage (\emph{Über welche Themen liest du gerne?}) angeboten haben. Zur
Erstellung unserer Bücherliste verwendeten wir hauptsächlich
Bestsellerlisten, zum Teil von Amazon, Ausleihstatistiken von
Bibliotheken und die Expertise einer Mitarbeiterin einer Buchhandlung.
Obwohl wir uns auf Kinder der dritten und vierten Schulstufe
beschränkten, waren auch Bücher in der Auswahl enthalten, die eher die
Funktion eines Vorlese- oder Erstlesebuchs erfüllen. Das hatte den
einfachen Grund auch Schülern und Schülerinnen, die nicht so viel lesen
oder sich auf einem weniger hohen Leseniveau befinden (wir waren auch in
Klassen mit hohen Migrationsanteil und in einer Integrationsklasse),
etwas anzubieten. Außerdem interessierte uns auch, ob und wie sich
Rollenangebote in den Büchern mit steigendem empfohlenem Lesealter
verändern. Der Vorteil einer Liste bestand für uns darin, eine gewisse
Breite an Büchern abzudecken und einer möglichen Schreibfaulheit der
Schüler und Schülerinnen entgegenzukommen, aber auch um Bücher, die vor
einiger Zeit gelesen und eventuell in Vergessenheit geraten waren, zu
repräsentieren. Bei offenen Fragen ist das Problem größer, die Frage
gemeinsam mit dem Nachbarn oder der Nachbarin zu beantworten, was
unserer Annahme nach insgesamt weniger und dafür mehr gleiche Antworten
produziert. Natürlich ist auch eine vorgefertigte Liste nicht frei von
ungewollten Ergebnissen: die Schüler und Schülerinnen könnten möglichst
viel ankreuzen, damit sie vielleicht besser dastehen, genauso gut
zusammenarbeiten oder auch Bücher, die sie nur von Fernsehserien oder
Filmen kennen, angeben. Außerdem kann ein Bias entstehen, wenn etwa eine
Klasse ein bestimmtes Buch auf der Literaturliste hatte und das jeder
Schüler und jede Schülerin sowieso lesen musste. Nach der Durchführung
eines Pretests wurden noch Einzelheiten im Fragebogen verändert. Danach
war es uns möglich, einzuschätzen, ob die Gestaltung des Bogens
überhaupt verständlich und adäquat ist und wie lange Kinder in diesem
Alter brauchen, um einen Bogen auszufüllen. Die Anzahl der Bücher
erschien uns passend, gleich wie die Auswahl der Titel.

\subsection{Auswertung und Ergebnisse}

Wir führten die Fragebogenerhebung gemeinsam mit einer zweiten Gruppe
unseres Forschungsprojekts, die sich mit Fernsehserien beschäftigte,
durch. Auch die Dateneingabe erfolgte in der Großgruppe: Es war wichtig
für jedes Buch und für jede Serie, das/die in der offenen Fragen genannt
wurde, eine eigene Variable zu bilden. Die Aufteilung, Kompatibilität
und Vollständigkeit stellten kein Problem dar. Insgesamt konnten wir mit
502 ausgefüllten Fragebögen (240 von Mädchen, 258 von Buben und vier
ohne Angabe des Geschlechts) aus zwanzig Klassen unsere ersten
Auswertungen beginnen. In die nähere Auswahl gelangten dann nur Bücher,
die mindestens fünfzig Nennungen aufwiesen, die anderen mussten wir
unberücksichtigt lassen, um die Auswahl zu reduzieren und die am
häufigsten gelesenen herauszuheben. Mithilfe von Häufigkeitsanalysen
konnten wir unsere vorhandene Liste dann erstmals von siebenunddreißig
auf dreißig Titel einschränken. Erst dann sahen wir uns die
Verhältnisse, also Nennungen von Buben und Mädchen separat an. Die
Ergebnisse sind in der Tabelle unten dargestellt, hier wird neben den
absoluten Lesehäufigkeiten ein Faktor errechnet, der ausdrücken soll, ab
wann es sich um ein Buben- oder Mädchenbuch handelt, ob sozusagen Leser
oder Leserinnen deutlich überwiegen oder nicht. Die Werte gehen hier
theoretisch von $1$ (alle Leser sind Buben = Bubenbuch), bis $-1$
(ausschließlich Leserinnen =
Mädchenbuch).\footnote{$w/m=\frac{b-m}{b+m}$}

      \small
      \ctable[
      %  cap    = ,
        caption = {Bücher die über 50 mal genannt wurden},
        label   = top30 ,
        % pos   = htp,
      %  width    = \textwidth
      ]{lD{,}{,}{0}D{,}{,}{0}D{,}{,}{0}D{,}{,}{2}}{
        \tnote{--1: 100\% Leserinnen; 0: gleich viele Leserinnen wie Leser; 1: 100\% Leser}
      % \tnote[.]{< 0,1}
      % \tnote[*]{< 0,05}
      % \tnote[**]{< 0,01}
      % \tnote[***]{< 0,001}
      }{                  
      \FL \footnotesize Bücher &  \multicolumn{1}{c}{\footnotesize Mädchen} & \multicolumn{1}{c}{\footnotesize Buben} & \multicolumn{1}{c}{\footnotesize Gesamt} & \multicolumn{1}{c}{ \footnotesize w/m-Faktor\tmark}
      \ML Die wilden Fußballkerle & 43 & 110 & 153 & 0,44\tmark[**]% 18  $ 0,00 **
      \NN Tiger-Team & 49 & 69 & 118 & 0,17
      \NN Knickerbocker-Bande & 48 & 67 & 115 & 0,17
      \NN Gregs Tagebuch & 86 & 117 & 203 & 0,15\tmark[*]% 28  $ 0,03 *
      \NN Harry Potter & 95 & 125 & 220 & 0,14\tmark[$\circ$]% 29  $ 0,05 .
      \NN Die drei ??? & 93 & 122 & 215 & 0,14\tmark[$\circ$]% 27  $ 0,05 .
      \NN Das magische Baumhaus & 84 & 105 & 189 & 0,11
      \NN Der kleine Ritter Trenk & 42 & 52 & 94 & 0,11
      \NN Tom Turbo & 92 & 113 & 205 & 0,10
      \NN Der kleine Drache Kokosnuss & 46 & 52 & 98 & 0,06
      \NN Der Räuber Hotzenplotz & 92 & 101 & 193 & 0,05
      \NN Sams & 63 & 67 & 130 & 0,03
      \NN Fünf Freunde & 114 & 118 & 232 & 0,02
      \NN Die Olchis & 47 & 48 & 95 & 0,01
      \NN Der Grüffelo & 58 & 54 & 112 & -0,04
      \NN Die Geggis & 36 & 31 & 67 & -0,08
      \NN Peter Pan & 90 & 73 & 163 & -0,10\tmark[*]% 15  $ 0,03 *
      \NN Der Regenbogenfisch & 122 & 95 & 217 & -0,12\tmark[**]% 1   $ 0,00 **
      \NN Baumhausgeschichten & 29 & 22 & 51 & -0,14
      \NN Geschichten von Franz & 83 & 60 & 143 & -0,16\tmark[*]% 16  $ 0,01 *
      \NN Pinocchio & 96 & 68 & 164 & -0,17\tmark[**]% 14  $ 0,00 **
      \NN Das kleine Wutmonster & 34 & 23 & 57 & -0,19% 7   $ 0,07 .
      \NN Der kleine Eisbär & 91 & 56 & 147 & -0,24\tmark[**]% 10  $ 0,00 **
      \NN Pipi Langstrumpf & 141 & 75 & 216 & -0,31\tmark[**]% 32  $ 0,00 **
      \NN Die kleine Hexe & 109 & 52 & 161 & -0,35\tmark[**]% 11  $ 0,00 **
      \NN Hexe Lilli & 162 & 53 & 215 & -0,51\tmark[**]% 35  $ 0,00 **
      \NN Die wilden Hühner & 77 & 25 & 10 & -0,51\tmark[**]% 24  $ 0,00 **
      \NN Mini & 59 & 16 & 75 & -0,57\tmark[**]% 36  $ 0,00 **
      \NN Conni & 94 & 22 & 116 & -0,62\tmark[**]% 37  $ 0,00 **
      \NN Prinzessin Lillifee & 109 & 14 & 123& -0,77\tmark[**]% 17  $ 0,00 **  
      \LL \multicolumn{5}{l}{\footnotesize $^\circ p<0{,}1, ^*p<0{,}05, ^{**}p<0{,}01$ $\chi^2$-Test (Mädchen/Buben) $N=498$}
      }
      

\normalsize

% 1   $ 0,00 **
% 3   $ 0,85
% 4   $ 0,33
% 5   $ 0,39
% 7   $ 0,07 .
% 10  $ 0,00 **
% 11  $ 0,00 **
% 13  $ 0,45
% 14  $ 0,00 **
% 15  $ 0,03 *
% 16  $ 0,01 *
% 17  $ 0,00 **
% 18  $ 0,00 **
% 19  $ 0,22
% 20  $ 0,97
% 21  $ 0,19
% 22  $ 0,10
% 23  $ 0,11
% 24  $ 0,00 **
% 25  $ 0,78
% 26  $ 0,69
% 27  $ 0,05 .
% 28  $ 0,03 *
% 29  $ 0,05 .
% 32  $ 0,00 **
% 33  $ 0,78
% 34  $ 0,19
% 35  $ 0,00 **
% 36  $ 0,00 **
% 37  $ 0,00 **

% N=498

Durch den Anspruch, dass Titel fünfzig Mal genannt worden sein mussten,
fielen alle Bücher, die nicht schon in der Liste enthalten waren,
heraus. Die offene \emph{Lieblingsbücher}-Frage erwies sich dennoch als
sinnvoll um die Bücherauswahl zu kontrollieren. Die höchste Anzahl an
Nennungen bei der offenen Frage, bekamen die \emph{Lustigen
Taschenbücher} mit vierzehn, was leider nicht als repräsentativ
angesehen werden kann. Die anderen \emph{Lieblingsbücher} dümpelten
meistens bei bis zu fünf Nennungen.

Schon auf den ersten Blick auf die Tabelle ist leicht zu erkennen, dass
Mädchenbücher einen eindeutigeren Faktor aufweisen als Bubenbücher:
Buben präferieren eindeutig \emph{Die wilden Fußballkerle} mit einem
Wert von über 0,4. Dann kommt erst mit einem Wert von 0,17 das
\emph{Tiger- Team}, dann \emph{Die Knickerbockerbande} und \emph{Gregs
Tagebuch}. Bei den Mädchen können wir die Zahlen viel eindeutiger
interpretieren, da ihre Werte näher am Extremwert angesiedelt sind.
\emph{Prinzessin Lillifee} führt die Liste mit einem Wert von -0,77 an,
es folgen \emph{Conni}, \emph{Geschichten von Mini}, \emph{Die wilden
Hühner} und \emph{Hexe Lilli}. Dieser w/m- Faktor ist aber immer noch
höher, als der von dem eindeutigsten Bubenbuch.

Kann angenommen werden, dass die beliebtesten Bücher klare Mädchen- oder
Bubenbücher sind oder handelt es sich bei ihnen um ausgewogene
Verhältnisse? Am beliebtesten beziehungsweise insgesamt am häufigsten
gelesen wurden die \emph{5 Freunde}. Dabei handelt es sich um Bände, die
keine klare Präferenz von Seiten der Buben oder Mädchen aufweisen. Dann
folgt \emph{Harry Potter}, wo Unterschiede nur tendenziell zugunsten der
Buben interpretiert werden könnten, wir es aufgrund der hohen
Leserinnenanzahl aber nicht als Bubenbuch definieren wollen. Der
Regenbogenfisch liegt klar in der Mitte, hier handelt es sich aber um
ein Buch, das bereits in früheren Altersgruppen gekannt wird und eher
nicht von acht- bis zehnjährigen Mädchen oder Buben bewusst ausgewählt
wird. Die \emph{Hexe Lilli} kann aufgrund der hohen Gesamtanzahl und der
klaren weiblichen Bevorzugung als eindeutiges Mädchenbuch definiert
werden.

Weitere interessante Ergebnisse lieferte die Auswertung der Frage zu
Themen, die Mädchen und Buben interessieren könnten. Dabei sind wir nach
dem gleichen Schema wie bei den einzelnen Titeln vorgegangen, weshalb
die folgende Tabelle gleich gelesen werden kann wie die obere.
Interessant ist, dass kein einziger Bub angegeben hat, gerne etwas über
Prinzessinnen zu lesen. Literatur, die sich um dieses Thema dreht, ist
eindeutig weiblich konnotiert, wahrscheinlich würde sich ein Junge
schämen, wenn man irgendetwas \emph{prinzessinnenhaftes} bei ihm
entdecken würde. Das erweckt den Eindruck, dass Buben sehr
mädchentypische Dinge stark ablehnen. Die Vermutung, dass Mädchen sich
eher für Freundschaft und Liebe interessieren (und sich das auch angeben
trauen), hat sich mit diesem Ergebnis bestätigt: Dabei kann eine
stärkere ``Beziehungsorientierung'' des weiblichen Geschlechts schon im
frühen Alter bestätigt werden. Auch Tiere, die eher im niedlicheren
Bereich eingeschätzt werden, finden bei den Mädchen eine klare
Bevorzugung. Buben favorisieren hingegen den technischen Bereich, auch
Drachen und Ritter sind für sie interessant. Wir vermuteten ursprünglich
hingegen eine stärkere Ablehnung von Mädchen im Bereich Fußball und
Sport.

     \small
      \ctable[
      %  cap    = ,
        caption = {Welche Themen liest du gerne?},
        label   = themen ,
        star,
        % pos   = htp,
      %  width    = \textwidth
      ]{lD{,}{,}{0}D{,}{,}{0}D{,}{,}{0}D{,}{,}{3}}{
        \tnote{--1: 100\% Leserinnen; 0: gleich viele Leserinnen wie Leser; 1: 100\% Leser}
      }{                  
      \FL \footnotesize Themen &  \multicolumn{1}{c}{\footnotesize Mädchen} & \multicolumn{1}{c}{\footnotesize Buben} & \multicolumn{1}{c}{\footnotesize Gesamt} & \multicolumn{1}{c}{ \footnotesize w/m-Faktor\tmark}
      \ML Autos/Technik & 16 & 130 & 146 & 0,78\tmark[**]
      \NN Drachen/Ritter & 44 & 107 & 151 & 0,68\tmark[**]
      \NN Dinosaurier & 32 & 87 & 119 & 0,46\tmark[**]
      \NN Fußball/Sport & 67 & 173 & 240 & 0,44\tmark[**]
      \NN Abenteuer/Indianer/Piraten & 77 & 116 & 193 & 0,20\tmark[**]
      \NN Geister/Monster & 97 & 122 & 219 & 0,11%
      \NN Meerestiere & 92 & 77 & 169 & -0,09\tmark[*]
      \NN Hexen/Zauberer & 114 & 52 & 166 & -0,37\tmark[**]
      \NN Pferde, Hunde, Katzen & 145 & 52 & 166 & -0,37\tmark[**]
      \NN Freunde/Liebe & 108 & 23 & 132 & -0,64\tmark[**]
      \NN Prinzessinen & 53 & 0 & 53& -1,00\tmark[**]  
      \LL \multicolumn{5}{l}{\footnotesize $^*p<0{,}05, ^{**}p<0{,}01$ $\chi^2$-Test (Mädchen/Buben) $N=498$}
      }
      \normalsize

Problematisch beim Fragebogen im Nachhinein war, dass die Bücherauswahl
doch eine weite Range abdeckt und Titel deshalb auch teilweise schwer
miteinander zu vergleichen sind. Bei Harry Potter und dem
Regenbogenfisch würde der Versuch ad absurdum führen. Außerdem stehen
den Kindern neben Klassikern, die seit Jahrzehnten gelesen werden (
\emph{Fünf Freunde}, \emph{Pipi Langstrumpf}) auch eine große Auswahl an
neuen Büchern zur Verfügung ( \emph{Gregs Tagebuch}, \emph{Die wilden
Fußballkerle}). Informationstechnisch hätte man aus dem Fragebogen nach
einer gründlicheren Recherche und Literaturanalyse etwas mehr
herausholen können. Außerdem wurden Sachbücher aufgrund eines fehlenden
Hauptcharkters nicht berücksichtigt, die in diesem Alter gerade von
Buben gerne gelesen werden und die auch Themenvorlieben gut
repräsentieren könnten. Auch die Auswahl kann, trotz Änderungen und
Verbesserungen unsererseits, Verzerrungen aufweisen, da bei diesem
Umfang der Liste von vornherein Vieles ausgeschlossen werden musste.

\subsection{Interpretation der Ergebnisse}

Bei den vorherigen Recherchen stießen wir immer wieder auf Ergebnisse
von PISA oder ähnlichen Studen, die darauf hinwiesen, dass Burschen
deutlich weniger lesen würden als Mädchen und auch eher mit
Leseschwächen zu kämpfen hätten, was aber allein anhand unseres
Fragebogens nicht nachgewiesen werden kann. Die leicht höhere Anzahl an
Gesamtnennungen bei den Mädchen kann das allein nicht bestätigen.
Unserer Vermutung nach könnte sich die geringere Anzahl an
Gesamtnennungen eventuell mit einem etwas geringere Leseinteresse der
Buben, wie auch einem fehlenden Angebot an Comics oder Sachbüchern (
z.B. \emph{Was ist was?}) erklären lassen. Warum aber gibt es mehr
eindeutige Mädchen- als Bubenbücher? Hier liegt die Erklärung nahe, die
sich auch mit Aussagen in der Literatur deckt, dass Mädchen einen
größeren Spielraum haben, wenn es um Interessengebiete oder
Handlungsmöglichkeiten geht. Mädchen sollen sogar weibliche und
männliche Elemente verbinden: starke, wie auch technikversierte Frauen
sind gern gesehen. Diese Entwicklung ist äußerst positiv zu bewerten,
wobei kritisiert werden kann, dass bei Buben diese Flexibilität (noch)
nicht den gleichen Stellenwert erreicht hat. Sie bewegen sich viel
seltener in ``weiblichen Domänen'', als dies Mädchen und Frauen
inzwischen umgekehrt tun.

Die beliebtesten stellen nicht gleichzeitig die eindeutigsten Bücher
dar. Allerdings sind einige unter den viel gelesenen dabei, die in eine
klare Richtung weisen und es deshalb wert sind, auf ihre Besonderheiten
hin, untersucht zu werden.

\chapter{Handeln Hauptfiguren in Mädchenbüchern anders als in
Bubenbüchern?}

\section{Inhaltliche Unterschiede}

Da wir in den vorangegangenen Kapiteln zeigen konnten, dass Mädchen und
Buben unterschiedliche Bücher lesen, was zwar klarer bei den Jungen als
bei den Mädchen festgstellt werden konnte, stellt sich dennoch die
Frage, ob es denn auch Unterschiede in den, von den Gruppen
favorisierten, Werken gibt. Dieses Kapitel hat sich zur Aufgabe gemacht,
sich auf die Suche nach inhaltlichen Merkmalen zu machen, die die beiden
Büchergruppen unterscheiden. Untermauert mit inhaltlichen
Interpretationen und Auszügen aus Beispielbüchern soll so ein
Verständnis über diese spezifisch verwendeten Merkmale erzeugt werden um
schließlich Annahmen über mögliche Folgen von derartigen Unterschieden
in den Lesepräferenzen von jungen Menschen zu formulieren. Der Inhalt
der Kinderbücher wurde hierbei auf zwei Ebenen untersucht:

\begin{itemize}
\item
  Unterschiede in der Darstellung des sozialen Geschlechts (Gender) der
  Hauptprotagonisten
\item
  Unterschiede im Aufbau und Verwendung stilistischer Mittel
\end{itemize}

\section{Darstellung Gender}

Auch wenn die Frage nach der Intensität des Einflusses von Kinderbüchern
auf die Sozialisierung und die Ausprägung von Geschlechterrollen bei
Kindern nicht restlos beantwortet werden kann, muss hier dennoch von
einem Einfluss ausgegangen werden. Dieser Einfluss geht vor allem von
den Hauptprotagonisten und deren Verhalten, Handeln und Denken aus. Sie
stellen jene Identifizierungsinstanz dar mit denen im Verlauf der
Geschichte des jeweiligen Buches am meisten mitgelitten, gefeiert und
gebangt wird. Daher ist es von hoher Bedeutung wie diese Charaktere
dargestellt werden. Jene Form der Darstellung, die uns hier im
Besonderen interessiert, ist die des sozialen Geschlechtes, welche mit
Hilfe einer Liste von 13 Eigenschaftspaaren (siehe Tabelle \ref{stereo})
erhoben wurde. Jedes dieser Eigenschaftspaare weist einen stereotyp
maskulinen und femininen Pol auf. Jeder Hauptcharakter der 30 Bücher
unserer Erhebung wurde auf diese 13 Eigenschaftspaare untersucht und ein
Gender-Faktor erstellt, der uns zeigen kann, wie maskulin oder feminin
die Protagonisten dargestellt werden.

      
      \ctable[
      %  cap    = ,
        caption = {w/m-Faktor -- Gender-Faktor},
        label   = top30gender ,
        pos   = htp,
        star,
      %  width    = \textwidth
      ]{lD{,}{,}{2}D{,}{,}{2}}{
        \tnote{--1: 100\% Leserinnen; 0: gleich viele Leserinnen wie Leser; 1: 100\% Leser}
      }{                  
      \FL \small Bücher &  \multicolumn{1}{c}{\small w/m-Faktor} & \multicolumn{1}{c}{ \small Gender-Faktor}
      \ML Die wilden Fußballkerle   & 0,44      & 0,54
      \NN Tiger-Team                & 0,17      &0,15
      \NN Die Knickerbocker-Bande        & 0,16      & 0,31
      \NN Gregs Tagebuch            & 0,15      &0,23
      \NN Harry Potter              & 0,13      &0,23
      \NN Die drei ???              & 0,14      &0,54
      \NN Das magische Baumhaus     & 0,11      &0,50
      \NN Der kleine Ritter Trenk   & 0,11      &0,23
      \NN Tom Turbo                 & 0,10      &0,69
      \NN Der kleine Drache Kokosnuss  & 0,06   &0,08
      \NN Der Räuber Hotzenplotz    & 0,05      &-0,08
      \NN Sams                      & 0,03      &-0,23
      \NN Fünf Freunde              & 0,02      &0,15
      \NN Die Olchis                & 0,01      &-0,15
      % \NN Der Grüffelo              & -0,04     &0,69
      % \NN Die Geggis                & -0,08     &-0,08
      \NN Peter Pan                 & -0,10     &-0,38
      % \NN Der Regenbogenfisch       & -0,12     &0,45
      % \NN Baumhausgeschichten       & -0,14     &-0,15
      \NN Geschichten von Franz     & -0,16     & -0,69
      \NN Pinocchio                 & -0,17     & -0,38
      % \NN Das kleine Wutmonster     & -0,19     & -0,23
      % \NN Der kleine Eisbär         & -0,24     & 0,17
      \NN Pipi Langstrumpf          & -0,31     & 0,08
      \NN Die kleine Hexe           & -0,35     & 0,54
      \NN Hexe Lilli                & -0,51     & 0,08
      \NN Die wilden Hühner         & -0,51     & 0,31
      \NN Mini                      & -0,57     & - 0,31
      \NN Conni                     & -0,62     & -0,62
      % \NN Prinzessin Lillifee       &-0,77      & -0,33  
      \LL
      }
      

Das Lesegeschlecht (w/m-Faktor) korreliert mit dem Gender-Faktor
hochsignifikant mit einem Wert von R = 0,471. Somit kann die bereits in
der Tabelle erkennbare Tendenz, dass Mädchen vor allem mit femininen
Charakteren sowie Buben vor allem mit maskulinen Charakteren
konfrontiert werden, auch statistisch festgehalten werden. Besonders
interessant ist jedoch dieses Ergebnis erst, wenn wir uns die
Geschlechter im Einzelnen ansehen. In der Abbildung \ref{wm-gender}
sehen wir um wie viel klarer diese Konfrontation bei den Jungen ausfällt
als bei den Mädchen. Dies ist wie folgt zu interpretieren: Beide
Geschlechter lesen vermehrt von sozialen Geschlechtern, die ihrem
eigenen biologischen Geschlecht entsprechen. Während jedoch die Buben
hier besonders stark mit maskulinen Protagonisten konfrontiert werden,
lesen Mädchen von Charakteren beider Genderausprägungen. Dies ist
wahrscheinlich einerseits in dem Tabu für Jungen in Mädchendomänen
einzudringen verbunden, anderseits auch mit einem emanzipatorischen
Anspruch vieler Autorinnen verknüpft die ihre Charaktere bewusst
untypisch darstellen, um bestehende Geschlechterverhältnisse
aufzubrechen.

\begin{figure}
\center
  \caption[w/m-Faktor--Gender-Faktor]{w/m-Faktor zu Gender-Faktor}
  \label{wm-gender}
% Created by tikzDevice version 0.6.2-92-0ad2792 on 2013-02-07 01:52:29
% !TEX encoding = UTF-8 Unicode
\begin{tikzpicture}[x=1pt,y=1pt]
\definecolor[named]{fillColor}{rgb}{1.00,1.00,1.00}
\path[use as bounding box,fill=fillColor,fill opacity=0.00] (0,0) rectangle (216.81,180.67);
\begin{scope}
\path[clip] (  0.00,  0.00) rectangle (216.81,180.67);
\definecolor[named]{drawColor}{rgb}{1.00,1.00,1.00}
\definecolor[named]{fillColor}{rgb}{1.00,1.00,1.00}

\path[draw=drawColor,line width= 0.6pt,line join=round,line cap=round,fill=fillColor] (  0.00,  0.00) rectangle (216.81,180.68);
\end{scope}
\begin{scope}
\path[clip] ( 42.89, 34.03) rectangle (204.77,168.63);
\definecolor[named]{fillColor}{rgb}{0.90,0.90,0.90}

\path[fill=fillColor] ( 42.89, 34.03) rectangle (204.77,168.63);
\definecolor[named]{drawColor}{rgb}{0.95,0.95,0.95}

\path[draw=drawColor,line width= 0.3pt,line join=round] ( 42.89, 55.45) --
	(204.77, 55.45);

\path[draw=drawColor,line width= 0.3pt,line join=round] ( 42.89, 86.04) --
	(204.77, 86.04);

\path[draw=drawColor,line width= 0.3pt,line join=round] ( 42.89,116.63) --
	(204.77,116.63);

\path[draw=drawColor,line width= 0.3pt,line join=round] ( 42.89,147.22) --
	(204.77,147.22);

\path[draw=drawColor,line width= 0.3pt,line join=round] ( 68.64, 34.03) --
	( 68.64,168.63);

\path[draw=drawColor,line width= 0.3pt,line join=round] (105.43, 34.03) --
	(105.43,168.63);

\path[draw=drawColor,line width= 0.3pt,line join=round] (142.22, 34.03) --
	(142.22,168.63);

\path[draw=drawColor,line width= 0.3pt,line join=round] (179.01, 34.03) --
	(179.01,168.63);
\definecolor[named]{drawColor}{rgb}{1.00,1.00,1.00}

\path[draw=drawColor,line width= 0.6pt,line join=round] ( 42.89, 40.15) --
	(204.77, 40.15);

\path[draw=drawColor,line width= 0.6pt,line join=round] ( 42.89, 70.74) --
	(204.77, 70.74);

\path[draw=drawColor,line width= 0.6pt,line join=round] ( 42.89,101.33) --
	(204.77,101.33);

\path[draw=drawColor,line width= 0.6pt,line join=round] ( 42.89,131.92) --
	(204.77,131.92);

\path[draw=drawColor,line width= 0.6pt,line join=round] ( 42.89,162.51) --
	(204.77,162.51);

\path[draw=drawColor,line width= 0.6pt,line join=round] ( 50.24, 34.03) --
	( 50.24,168.63);

\path[draw=drawColor,line width= 0.6pt,line join=round] ( 87.03, 34.03) --
	( 87.03,168.63);

\path[draw=drawColor,line width= 0.6pt,line join=round] (123.83, 34.03) --
	(123.83,168.63);

\path[draw=drawColor,line width= 0.6pt,line join=round] (160.62, 34.03) --
	(160.62,168.63);

\path[draw=drawColor,line width= 0.6pt,line join=round] (197.41, 34.03) --
	(197.41,168.63);
\definecolor[named]{fillColor}{rgb}{0.00,0.00,0.00}

\path[fill=fillColor] (112.51,101.98) circle (  2.13);

\path[fill=fillColor] (118.17,104.19) circle (  2.13);

\path[fill=fillColor] (129.49,105.08) circle (  2.13);

\path[fill=fillColor] (129.49, 70.32) circle (  2.13);

\path[fill=fillColor] (140.81,107.84) circle (  2.13);

\path[fill=fillColor] (163.45, 79.67) circle (  2.13);

\path[fill=fillColor] ( 78.54, 63.36) circle (  2.13);

\path[fill=fillColor] ( 95.52, 90.89) circle (  2.13);

\path[fill=fillColor] ( 95.52, 94.95) circle (  2.13);

\path[fill=fillColor] (174.77,107.60) circle (  2.13);

\path[fill=fillColor] ( 72.88, 91.49) circle (  2.13);

\path[fill=fillColor] (101.18, 66.26) circle (  2.13);

\path[fill=fillColor] (106.85,103.21) circle (  2.13);

\path[fill=fillColor] (129.49, 82.64) circle (  2.13);

\path[fill=fillColor] (135.15,102.39) circle (  2.13);

\path[fill=fillColor] (135.15,111.70) circle (  2.13);

\path[fill=fillColor] (163.45,128.12) circle (  2.13);

\path[fill=fillColor] (146.47,111.44) circle (  2.13);

\path[fill=fillColor] (140.81,109.67) circle (  2.13);

\path[fill=fillColor] (140.81,110.68) circle (  2.13);

\path[fill=fillColor] (146.47, 70.14) circle (  2.13);

\path[fill=fillColor] (160.62,108.13) circle (  2.13);

\path[fill=fillColor] (163.45,109.58) circle (  2.13);
\end{scope}
\begin{scope}
\path[clip] (  0.00,  0.00) rectangle (216.81,180.67);
\definecolor[named]{drawColor}{rgb}{0.50,0.50,0.50}

\node[text=drawColor,anchor=base east,inner sep=0pt, outer sep=0pt, scale=  0.96] at ( 35.77, 36.85) {-1.0};

\node[text=drawColor,anchor=base east,inner sep=0pt, outer sep=0pt, scale=  0.96] at ( 35.77, 67.44) {-0.5};

\node[text=drawColor,anchor=base east,inner sep=0pt, outer sep=0pt, scale=  0.96] at ( 35.77, 98.03) {0.0};

\node[text=drawColor,anchor=base east,inner sep=0pt, outer sep=0pt, scale=  0.96] at ( 35.77,128.62) {0.5};

\node[text=drawColor,anchor=base east,inner sep=0pt, outer sep=0pt, scale=  0.96] at ( 35.77,159.21) {1.0};
\end{scope}
\begin{scope}
\path[clip] (  0.00,  0.00) rectangle (216.81,180.67);
\definecolor[named]{drawColor}{rgb}{0.50,0.50,0.50}

\path[draw=drawColor,line width= 0.6pt,line join=round] ( 38.62, 40.15) --
	( 42.89, 40.15);

\path[draw=drawColor,line width= 0.6pt,line join=round] ( 38.62, 70.74) --
	( 42.89, 70.74);

\path[draw=drawColor,line width= 0.6pt,line join=round] ( 38.62,101.33) --
	( 42.89,101.33);

\path[draw=drawColor,line width= 0.6pt,line join=round] ( 38.62,131.92) --
	( 42.89,131.92);

\path[draw=drawColor,line width= 0.6pt,line join=round] ( 38.62,162.51) --
	( 42.89,162.51);
\end{scope}
\begin{scope}
\path[clip] (  0.00,  0.00) rectangle (216.81,180.67);
\definecolor[named]{drawColor}{rgb}{0.50,0.50,0.50}

\path[draw=drawColor,line width= 0.6pt,line join=round] ( 50.24, 29.77) --
	( 50.24, 34.03);

\path[draw=drawColor,line width= 0.6pt,line join=round] ( 87.03, 29.77) --
	( 87.03, 34.03);

\path[draw=drawColor,line width= 0.6pt,line join=round] (123.83, 29.77) --
	(123.83, 34.03);

\path[draw=drawColor,line width= 0.6pt,line join=round] (160.62, 29.77) --
	(160.62, 34.03);

\path[draw=drawColor,line width= 0.6pt,line join=round] (197.41, 29.77) --
	(197.41, 34.03);
\end{scope}
\begin{scope}
\path[clip] (  0.00,  0.00) rectangle (216.81,180.67);
\definecolor[named]{drawColor}{rgb}{0.50,0.50,0.50}

\node[text=drawColor,anchor=base,inner sep=0pt, outer sep=0pt, scale=  0.96] at ( 50.24, 20.31) {-1.0};

\node[text=drawColor,anchor=base,inner sep=0pt, outer sep=0pt, scale=  0.96] at ( 87.03, 20.31) {-0.5};

\node[text=drawColor,anchor=base,inner sep=0pt, outer sep=0pt, scale=  0.96] at (123.83, 20.31) {0.0};

\node[text=drawColor,anchor=base,inner sep=0pt, outer sep=0pt, scale=  0.96] at (160.62, 20.31) {0.5};

\node[text=drawColor,anchor=base,inner sep=0pt, outer sep=0pt, scale=  0.96] at (197.41, 20.31) {1.0};
\end{scope}
\begin{scope}
\path[clip] (  0.00,  0.00) rectangle (216.81,180.67);
\definecolor[named]{drawColor}{rgb}{0.00,0.00,0.00}

\node[text=drawColor,anchor=base,inner sep=0pt, outer sep=0pt, scale=  1.20] at (123.83,  9.03) {Gender-Faktor};
\end{scope}
\begin{scope}
\path[clip] (  0.00,  0.00) rectangle (216.81,180.67);
\definecolor[named]{drawColor}{rgb}{0.00,0.00,0.00}

\node[text=drawColor,rotate= 90.00,anchor=base,inner sep=0pt, outer sep=0pt, scale=  1.20] at ( 17.30,101.33) {w/m-Faktor};
\end{scope}
\end{tikzpicture}


\end{figure}

\begin{figure}
\center
  \caption[Mädchen--Gender-Faktor]{Mädchen zu Gender-Faktor}
  \label{w-gender}
% Created by tikzDevice version 0.6.2-92-0ad2792 on 2013-02-07 02:00:58
% !TEX encoding = UTF-8 Unicode
\begin{tikzpicture}[x=1pt,y=1pt]
\definecolor[named]{fillColor}{rgb}{1.00,1.00,1.00}
\path[use as bounding box,fill=fillColor,fill opacity=0.00] (0,0) rectangle (216.81,180.67);
\begin{scope}
\path[clip] (  0.00,  0.00) rectangle (216.81,180.67);
\definecolor[named]{drawColor}{rgb}{1.00,1.00,1.00}
\definecolor[named]{fillColor}{rgb}{1.00,1.00,1.00}

\path[draw=drawColor,line width= 0.6pt,line join=round,line cap=round,fill=fillColor] (  0.00,  0.00) rectangle (216.81,180.68);
\end{scope}
\begin{scope}
\path[clip] ( 41.82, 34.03) rectangle (204.77,168.63);
\definecolor[named]{fillColor}{rgb}{0.90,0.90,0.90}

\path[fill=fillColor] ( 41.82, 34.03) rectangle (204.77,168.63);
\definecolor[named]{drawColor}{rgb}{0.95,0.95,0.95}

\path[draw=drawColor,line width= 0.3pt,line join=round] ( 41.82, 55.45) --
	(204.77, 55.45);

\path[draw=drawColor,line width= 0.3pt,line join=round] ( 41.82, 86.04) --
	(204.77, 86.04);

\path[draw=drawColor,line width= 0.3pt,line join=round] ( 41.82,116.63) --
	(204.77,116.63);

\path[draw=drawColor,line width= 0.3pt,line join=round] ( 41.82,147.22) --
	(204.77,147.22);

\path[draw=drawColor,line width= 0.3pt,line join=round] ( 67.74, 34.03) --
	( 67.74,168.63);

\path[draw=drawColor,line width= 0.3pt,line join=round] (104.78, 34.03) --
	(104.78,168.63);

\path[draw=drawColor,line width= 0.3pt,line join=round] (141.81, 34.03) --
	(141.81,168.63);

\path[draw=drawColor,line width= 0.3pt,line join=round] (178.84, 34.03) --
	(178.84,168.63);
\definecolor[named]{drawColor}{rgb}{1.00,1.00,1.00}

\path[draw=drawColor,line width= 0.6pt,line join=round] ( 41.82, 40.15) --
	(204.77, 40.15);

\path[draw=drawColor,line width= 0.6pt,line join=round] ( 41.82, 70.74) --
	(204.77, 70.74);

\path[draw=drawColor,line width= 0.6pt,line join=round] ( 41.82,101.33) --
	(204.77,101.33);

\path[draw=drawColor,line width= 0.6pt,line join=round] ( 41.82,131.92) --
	(204.77,131.92);

\path[draw=drawColor,line width= 0.6pt,line join=round] ( 41.82,162.51) --
	(204.77,162.51);

\path[draw=drawColor,line width= 0.6pt,line join=round] ( 49.23, 34.03) --
	( 49.23,168.63);

\path[draw=drawColor,line width= 0.6pt,line join=round] ( 86.26, 34.03) --
	( 86.26,168.63);

\path[draw=drawColor,line width= 0.6pt,line join=round] (123.29, 34.03) --
	(123.29,168.63);

\path[draw=drawColor,line width= 0.6pt,line join=round] (160.33, 34.03) --
	(160.33,168.63);

\path[draw=drawColor,line width= 0.6pt,line join=round] (197.36, 34.03) --
	(197.36,168.63);
\definecolor[named]{fillColor}{rgb}{0.00,0.00,0.00}

\path[fill=fillColor] (111.90, 68.91) circle (  2.13);

\path[fill=fillColor] (117.59, 96.44) circle (  2.13);

\path[fill=fillColor] (128.99, 68.30) circle (  2.13);

\path[fill=fillColor] (128.99,139.26) circle (  2.13);

\path[fill=fillColor] (140.38, 65.85) circle (  2.13);

\path[fill=fillColor] (163.17,106.84) circle (  2.13);

\path[fill=fillColor] ( 77.71, 97.66) circle (  2.13);

\path[fill=fillColor] ( 94.81, 98.89) circle (  2.13);

\path[fill=fillColor] ( 94.81, 95.21) circle (  2.13);

\path[fill=fillColor] (174.57, 96.44) circle (  2.13);

\path[fill=fillColor] ( 72.02, 90.93) circle (  2.13);

\path[fill=fillColor] (100.50, 76.25) circle (  2.13);

\path[fill=fillColor] (106.20, 78.70) circle (  2.13);

\path[fill=fillColor] (128.99,126.42) circle (  2.13);

\path[fill=fillColor] (134.69,109.90) circle (  2.13);

\path[fill=fillColor] (134.69, 70.13) circle (  2.13);

\path[fill=fillColor] (163.17, 66.46) circle (  2.13);

\path[fill=fillColor] (146.08, 69.52) circle (  2.13);

\path[fill=fillColor] (140.38, 98.27) circle (  2.13);

\path[fill=fillColor] (140.38, 92.77) circle (  2.13);

\path[fill=fillColor] (146.08, 87.26) circle (  2.13);

\path[fill=fillColor] (160.33, 91.54) circle (  2.13);

\path[fill=fillColor] (163.17, 97.05) circle (  2.13);
\end{scope}
\begin{scope}
\path[clip] (  0.00,  0.00) rectangle (216.81,180.67);
\definecolor[named]{drawColor}{rgb}{0.50,0.50,0.50}

\node[text=drawColor,anchor=base east,inner sep=0pt, outer sep=0pt, scale=  0.96] at ( 34.71, 36.85) {0};

\node[text=drawColor,anchor=base east,inner sep=0pt, outer sep=0pt, scale=  0.96] at ( 34.71, 67.44) {50};

\node[text=drawColor,anchor=base east,inner sep=0pt, outer sep=0pt, scale=  0.96] at ( 34.71, 98.03) {100};

\node[text=drawColor,anchor=base east,inner sep=0pt, outer sep=0pt, scale=  0.96] at ( 34.71,128.62) {150};

\node[text=drawColor,anchor=base east,inner sep=0pt, outer sep=0pt, scale=  0.96] at ( 34.71,159.21) {200};
\end{scope}
\begin{scope}
\path[clip] (  0.00,  0.00) rectangle (216.81,180.67);
\definecolor[named]{drawColor}{rgb}{0.50,0.50,0.50}

\path[draw=drawColor,line width= 0.6pt,line join=round] ( 37.55, 40.15) --
	( 41.82, 40.15);

\path[draw=drawColor,line width= 0.6pt,line join=round] ( 37.55, 70.74) --
	( 41.82, 70.74);

\path[draw=drawColor,line width= 0.6pt,line join=round] ( 37.55,101.33) --
	( 41.82,101.33);

\path[draw=drawColor,line width= 0.6pt,line join=round] ( 37.55,131.92) --
	( 41.82,131.92);

\path[draw=drawColor,line width= 0.6pt,line join=round] ( 37.55,162.51) --
	( 41.82,162.51);
\end{scope}
\begin{scope}
\path[clip] (  0.00,  0.00) rectangle (216.81,180.67);
\definecolor[named]{drawColor}{rgb}{0.50,0.50,0.50}

\path[draw=drawColor,line width= 0.6pt,line join=round] ( 49.23, 29.77) --
	( 49.23, 34.03);

\path[draw=drawColor,line width= 0.6pt,line join=round] ( 86.26, 29.77) --
	( 86.26, 34.03);

\path[draw=drawColor,line width= 0.6pt,line join=round] (123.29, 29.77) --
	(123.29, 34.03);

\path[draw=drawColor,line width= 0.6pt,line join=round] (160.33, 29.77) --
	(160.33, 34.03);

\path[draw=drawColor,line width= 0.6pt,line join=round] (197.36, 29.77) --
	(197.36, 34.03);
\end{scope}
\begin{scope}
\path[clip] (  0.00,  0.00) rectangle (216.81,180.67);
\definecolor[named]{drawColor}{rgb}{0.50,0.50,0.50}

\node[text=drawColor,anchor=base,inner sep=0pt, outer sep=0pt, scale=  0.96] at ( 49.23, 20.31) {-1.0};

\node[text=drawColor,anchor=base,inner sep=0pt, outer sep=0pt, scale=  0.96] at ( 86.26, 20.31) {-0.5};

\node[text=drawColor,anchor=base,inner sep=0pt, outer sep=0pt, scale=  0.96] at (123.29, 20.31) {0.0};

\node[text=drawColor,anchor=base,inner sep=0pt, outer sep=0pt, scale=  0.96] at (160.33, 20.31) {0.5};

\node[text=drawColor,anchor=base,inner sep=0pt, outer sep=0pt, scale=  0.96] at (197.36, 20.31) {1.0};
\end{scope}
\begin{scope}
\path[clip] (  0.00,  0.00) rectangle (216.81,180.67);
\definecolor[named]{drawColor}{rgb}{0.00,0.00,0.00}

\node[text=drawColor,anchor=base,inner sep=0pt, outer sep=0pt, scale=  1.20] at (123.29,  9.03) {Gender-Faktor};
\end{scope}
\begin{scope}
\path[clip] (  0.00,  0.00) rectangle (216.81,180.67);
\definecolor[named]{drawColor}{rgb}{0.00,0.00,0.00}

\node[text=drawColor,rotate= 90.00,anchor=base,inner sep=0pt, outer sep=0pt, scale=  1.20] at ( 17.30,101.33) {Mädchen};
\end{scope}
\end{tikzpicture}


\end{figure}

\begin{figure}
\center
  \caption[Buben--Gender-Faktor]{Buben zu Gender-Faktor}
  \label{m-gender}
% Created by tikzDevice version 0.6.2-92-0ad2792 on 2013-02-06 22:44:19
% !TEX encoding = UTF-8 Unicode
\begin{tikzpicture}[x=1pt,y=1pt]
\definecolor[named]{fillColor}{rgb}{1.00,1.00,1.00}
\path[use as bounding box,fill=fillColor,fill opacity=0.00] (0,0) rectangle (216.81,216.81);
\begin{scope}
\path[clip] (  0.00,  0.00) rectangle (216.81,216.81);
\definecolor[named]{drawColor}{rgb}{0.00,0.00,0.00}

\path[draw=drawColor,line width= 0.4pt,line join=round,line cap=round] ( 54.47, 61.20) -- (186.34, 61.20);

\path[draw=drawColor,line width= 0.4pt,line join=round,line cap=round] ( 54.47, 61.20) -- ( 54.47, 55.20);

\path[draw=drawColor,line width= 0.4pt,line join=round,line cap=round] ( 87.44, 61.20) -- ( 87.44, 55.20);

\path[draw=drawColor,line width= 0.4pt,line join=round,line cap=round] (120.41, 61.20) -- (120.41, 55.20);

\path[draw=drawColor,line width= 0.4pt,line join=round,line cap=round] (153.37, 61.20) -- (153.37, 55.20);

\path[draw=drawColor,line width= 0.4pt,line join=round,line cap=round] (186.34, 61.20) -- (186.34, 55.20);

\node[text=drawColor,anchor=base,inner sep=0pt, outer sep=0pt, scale=  1.00] at ( 54.47, 39.60) {-1.0};

\node[text=drawColor,anchor=base,inner sep=0pt, outer sep=0pt, scale=  1.00] at ( 87.44, 39.60) {-0.5};

\node[text=drawColor,anchor=base,inner sep=0pt, outer sep=0pt, scale=  1.00] at (120.41, 39.60) {0.0};

\node[text=drawColor,anchor=base,inner sep=0pt, outer sep=0pt, scale=  1.00] at (153.37, 39.60) {0.5};

\node[text=drawColor,anchor=base,inner sep=0pt, outer sep=0pt, scale=  1.00] at (186.34, 39.60) {1.0};

\path[draw=drawColor,line width= 0.4pt,line join=round,line cap=round] ( 49.20, 65.14) -- ( 49.20,163.67);

\path[draw=drawColor,line width= 0.4pt,line join=round,line cap=round] ( 49.20, 65.14) -- ( 43.20, 65.14);

\path[draw=drawColor,line width= 0.4pt,line join=round,line cap=round] ( 49.20, 84.85) -- ( 43.20, 84.85);

\path[draw=drawColor,line width= 0.4pt,line join=round,line cap=round] ( 49.20,104.55) -- ( 43.20,104.55);

\path[draw=drawColor,line width= 0.4pt,line join=round,line cap=round] ( 49.20,124.26) -- ( 43.20,124.26);

\path[draw=drawColor,line width= 0.4pt,line join=round,line cap=round] ( 49.20,143.96) -- ( 43.20,143.96);

\path[draw=drawColor,line width= 0.4pt,line join=round,line cap=round] ( 49.20,163.67) -- ( 43.20,163.67);

\node[text=drawColor,rotate= 90.00,anchor=base,inner sep=0pt, outer sep=0pt, scale=  1.00] at ( 34.80, 65.14) {0};

\node[text=drawColor,rotate= 90.00,anchor=base,inner sep=0pt, outer sep=0pt, scale=  1.00] at ( 34.80, 84.85) {50};

\node[text=drawColor,rotate= 90.00,anchor=base,inner sep=0pt, outer sep=0pt, scale=  1.00] at ( 34.80,124.26) {150};

\node[text=drawColor,rotate= 90.00,anchor=base,inner sep=0pt, outer sep=0pt, scale=  1.00] at ( 34.80,163.67) {250};

\path[draw=drawColor,line width= 0.4pt,line join=round,line cap=round] ( 49.20, 61.20) --
	(191.61, 61.20) --
	(191.61,167.61) --
	( 49.20,167.61) --
	( 49.20, 61.20);
\end{scope}
\begin{scope}
\path[clip] (  0.00,  0.00) rectangle (216.81,216.81);
\definecolor[named]{drawColor}{rgb}{0.00,0.00,0.00}

\node[text=drawColor,anchor=base,inner sep=0pt, outer sep=0pt, scale=  1.00] at (120.41, 15.60) {Gender-Faktor};

\node[text=drawColor,rotate= 90.00,anchor=base,inner sep=0pt, outer sep=0pt, scale=  1.00] at ( 10.80,114.41) {Häufigkeit Buben};
\end{scope}
\begin{scope}
\path[clip] ( 49.20, 61.20) rectangle (191.61,167.61);
\definecolor[named]{drawColor}{rgb}{0.83,0.83,0.83}

\path[draw=drawColor,line width= 0.4pt,line join=round,line cap=round] ( 54.47, 61.20) -- ( 54.47,167.61);

\path[draw=drawColor,line width= 0.4pt,line join=round,line cap=round] ( 87.44, 61.20) -- ( 87.44,167.61);

\path[draw=drawColor,line width= 0.4pt,line join=round,line cap=round] (120.41, 61.20) -- (120.41,167.61);

\path[draw=drawColor,line width= 0.4pt,line join=round,line cap=round] (153.37, 61.20) -- (153.37,167.61);

\path[draw=drawColor,line width= 0.4pt,line join=round,line cap=round] (186.34, 61.20) -- (186.34,167.61);

\path[draw=drawColor,line width= 0.4pt,line join=round,line cap=round] ( 49.20, 65.14) -- (191.61, 65.14);

\path[draw=drawColor,line width= 0.4pt,line join=round,line cap=round] ( 49.20, 84.85) -- (191.61, 84.85);

\path[draw=drawColor,line width= 0.4pt,line join=round,line cap=round] ( 49.20,104.55) -- (191.61,104.55);

\path[draw=drawColor,line width= 0.4pt,line join=round,line cap=round] ( 49.20,124.26) -- (191.61,124.26);

\path[draw=drawColor,line width= 0.4pt,line join=round,line cap=round] ( 49.20,143.96) -- (191.61,143.96);

\path[draw=drawColor,line width= 0.4pt,line join=round,line cap=round] ( 49.20,163.67) -- (191.61,163.67);
\end{scope}
\begin{scope}
\path[clip] (  0.00,  0.00) rectangle (216.81,216.81);
\definecolor[named]{drawColor}{rgb}{0.00,0.00,0.00}

\path[draw=drawColor,line width= 0.4pt,line join=round,line cap=round] ( 49.20, 61.20) --
	(191.61, 61.20) --
	(191.61,167.61) --
	( 49.20,167.61) --
	( 49.20, 61.20);
\end{scope}
\begin{scope}
\path[clip] ( 49.20, 61.20) rectangle (191.61,167.61);
\definecolor[named]{drawColor}{rgb}{0.00,0.00,0.00}

\path[draw=drawColor,line width= 0.4pt,line join=round,line cap=round] (155.91,108.49) circle (  2.25);

\path[draw=drawColor,line width= 0.4pt,line join=round,line cap=round] (130.55, 92.33) circle (  2.25);

\path[draw=drawColor,line width= 0.4pt,line join=round,line cap=round] (140.69, 91.55) circle (  2.25);

\path[draw=drawColor,line width= 0.4pt,line join=round,line cap=round] (135.62,111.25) circle (  2.25);

\path[draw=drawColor,line width= 0.4pt,line join=round,line cap=round] (135.62,114.41) circle (  2.25);

\path[draw=drawColor,line width= 0.4pt,line join=round,line cap=round] (155.91,113.22) circle (  2.25);

\path[draw=drawColor,line width= 0.4pt,line join=round,line cap=round] (153.37,106.52) circle (  2.25);

\path[draw=drawColor,line width= 0.4pt,line join=round,line cap=round] (135.62, 85.63) circle (  2.25);

\path[draw=drawColor,line width= 0.4pt,line join=round,line cap=round] (166.05,109.68) circle (  2.25);

\path[draw=drawColor,line width= 0.4pt,line join=round,line cap=round] (125.48, 85.63) circle (  2.25);

\path[draw=drawColor,line width= 0.4pt,line join=round,line cap=round] (115.33,104.95) circle (  2.25);

\path[draw=drawColor,line width= 0.4pt,line join=round,line cap=round] (105.19, 91.55) circle (  2.25);

\path[draw=drawColor,line width= 0.4pt,line join=round,line cap=round] (130.55,111.65) circle (  2.25);

\path[draw=drawColor,line width= 0.4pt,line join=round,line cap=round] (110.26, 84.06) circle (  2.25);

\path[draw=drawColor,line width= 0.4pt,line join=round,line cap=round] ( 95.05, 93.91) circle (  2.25);

\path[draw=drawColor,line width= 0.4pt,line join=round,line cap=round] ( 74.76, 88.79) circle (  2.25);

\path[draw=drawColor,line width= 0.4pt,line join=round,line cap=round] ( 95.05, 91.94) circle (  2.25);

\path[draw=drawColor,line width= 0.4pt,line join=round,line cap=round] (125.48, 94.70) circle (  2.25);

\path[draw=drawColor,line width= 0.4pt,line join=round,line cap=round] (155.91, 85.63) circle (  2.25);

\path[draw=drawColor,line width= 0.4pt,line join=round,line cap=round] (125.48, 86.03) circle (  2.25);

\path[draw=drawColor,line width= 0.4pt,line join=round,line cap=round] (140.69, 74.99) circle (  2.25);

\path[draw=drawColor,line width= 0.4pt,line join=round,line cap=round] (100.12, 71.45) circle (  2.25);

\path[draw=drawColor,line width= 0.4pt,line join=round,line cap=round] ( 79.83, 73.81) circle (  2.25);
\definecolor[named]{drawColor}{rgb}{0.00,0.76,0.75}

\path[draw=drawColor,line width= 0.4pt,line join=round,line cap=round] ( 74.76, 80.86) --
	(166.05,105.26);
\end{scope}
\end{tikzpicture}


\end{figure}

\subsection{Eigenschaftspaare}

Zur besseren Nachvollziehbarkeit des Genderfaktors sollen im Folgenden
drei der 13 Eigenschaftspaare etwas genauer vorgestellt werden, die
besonders aufgrund ihrer Signifikanz besonders aufgefallen sind.

\subsubsection{Unterwürfig/Dominant}

Unterwürfig ist eine Person besonders dann, wenn sie sich befehligen
lässt. Gehorsam zu sein und ohne viel Wiederstand seine eigene Position
aufzugeben sind weitere Beschreibungen dieser Eigenschaft. Dominante
Personen können hingegen ihren Willen gegenüber anderen durchsetzen.
Andere zu befehligen ist ein guter Indikator um dominante Charaktere zu
erkennen. Die Unterwerfung ist aus Sicht des Doing-Gender eine weibliche
Eigenschaft, während den Männern die Dominanz zugeschrieben wird.

Der w/m-Faktor korreliert dabei mit dieser Variabel sehr stark (0,379;
Sig. 0,043). Wir können daraus lesen, dass diese Eigenschaften besonders
klischeehaft in den gelesenen Kinderbüchern bei den Hauptcharakteren
verwendet wurden. Auffällig ist hier ebenfalls, dass männliche Autoren
dazu tendieren dominante Hauptprotagonisten zu entwerfen (R = 0,330;
Sig. 0,086)

\subsubsection{Sicherheitsbedürftig/Abenteuerlustig}

Sicherheitsbedürftig zu sein äußert sich zumeist daran, dass eine
Persönlichkeit sehr zurückgezogen lebt und sehr überlegt handelt.
Zumeist umgeben sich sicherheitsbedürftige Menschen mit anderen Menschen
ihres persönlichen Vertrauens. Abenteuerlustige Personen gehen Risiken
ein und werfen sich der Gefahr entgegen. Dies tun sie meist ohne viel
darüber nachzudenken. Auch hier wird jede der Eigenschaften einem
Geschlecht stereotyp zugeteilt. Sicherheitsbedürftigkeit ist somit eine
weibliche Eigenschaft, während Männer abenteuerlustig sind.

Der w/m-Faktor korreliert dabei mit dieser Variabel sehr hoch (0,384;
Sig. 0,036). Wir können daraus erkennen, dass dieses Eigenschaftspaar
besonders klischeehaft bei der Konstruktion von Protagonisten in
Kinderbüchern verwendet wird.

\subsubsection{Träumerisch/Realistisch}

Verträumte Entscheidungen sind oftmals optimistisch motiviert während
realistisches Denken starke rationale Gedankengänge verlangt. Oftmals
aus einer Laune heraus getroffen sind verträumte Entscheidungen spontan
aber auch mit Risiken verknüpft. Träumerisch wird aus Sichtweise des
Doing-Gender mit Femininität assozierte. Realistisch wird hier als
maskulines Attribut geführt.

Der w/m-Faktor korreliert mit dieser Variable am höchsten von allen
Eigenschaftspaaren (0,479; Sig. 0,01). Auch hier kann davon ausgegangen
werden, dass diese Eigenschaften besonders stereotyp verwendet werden.
Dabei sollte noch erklärt werden, dass alle anderen Eigenschaftspaare
gleich gepolt sind und zwar ebenfalls positiv korrelieren, das
Signifikanzniveau jedoch zu niedrig ist um eine klare Aussage zu
tätigen. Daher kann hier nur soweit interpretiert werden, dass keine
einzige Gender-Eigenschaft eine Tendenz zu einer nicht-klischeehaft
Verwendung vorweist.

\subsubsection{Beispiel Franz}

Wie man aus der Tabelle \ref{top30gender} entnehmen kann, werden die
Geschichten vom Franz bevorzugterweise von Mädchen gelesen und das
obwohl die Hauptfigur ein Bub ist. Der Gender-Faktor des lieben Franz
sieht jedoch ganz anders aus. Mit dem niedrigsten Wert aller 30
Hauptcharaktere stellt er den feministen Protagonisten dar und bietet
damit eine spannende Basis für eine inhaltsanalytische Untersuchung.

Situationsbeschreibung:

Franz spielt mit Sandra und Gabi Prinz und Prinzessin, wobei Sandra den
Prinzen spielt und Gabi - in die er sich verliebt hat - die Prinzessin.
Die beiden verlangen von ihm den Hofzwerg zu spielen und das obwohl er
sogern der Prinz wäre:

\begin{quote}
``Als sie dann eines Tages wollte, dass der Franz den königlichen
Hofzwerg spielte, da reichte es ihm! Und als sie dann noch erklärte, der
Franz sollte sich deswegen nicht aufregen, denn für einen Prinzen sei er
viel zu klein, da sah der Franz nur noch rot. Er warf der Sandra die
Zipfelmütze, die er als Hofzwerg aufsetzen sollte, an den Kopf und lief
nach Hause. Schluchzend warf er sich auf sein Bett und trommelte mit den
Fäusten in sein Kissen.'' \parencite[][30]{Noestlinger2010}
\end{quote}

Die Geschichten vom Franz thematisieren auf humorvolle Weise die
Bewältigung des Alltags: Schulprobleme, die erste Liebe, Beziehungen,
Peinlichkeiten, Gefühle und Vieles mehr. Franz zeigt Emotionen und wirkt
oft so, als hätte er nicht viel Selbstbewusstsein. Anhand der drei
Eigenschaftspaare die vorgestellt wurden, kann man ihn als
unterwürfigen, sicherheitsbedürftigen und träumerischen Protagonisten
einordnen.

\subsection{Multiprotagonisten}

Leser und Leserinnen dieser Arbeit, die einige der hier untersuchten
Kinderbücher kennen oder gar selbst gelesen haben, wird aufgefallen
sein, dass nicht jedem der 30 Bücher ein klar definierter einzelner
Hauptcharakter zugeordnet werden kann. Durch eine starke Selektion von
ebenfalls wichtigen aber dennoch untergeordneten Charakteren, konnte
sich die Forschungsgruppe in den meisten Fällen auf einen einzelnen bzw.
den prägendsten Charakter einigen. Das dabei am heftigsten diskutierte
Opfer dieser Selektion ist Peter Pan, da die literarische Aufbereitung
des Textes von den meisten Verfilmungen abweicht und nicht Peter sondern
vielmehr Wendy im Mittelpunkt der Erzählungen steht. Besonders in den
Detektivgeschichten ist es zumeist nicht möglich einen Charakter als den
Hauptprotagonisten zu deklarieren, da diese fast ausschließlich aus
einem Team junger Detektive und Detektivinnen bestehen, die gleichwertig
nebeneinander agieren. Hier wurde das Prinzip des Multiprotagonisten
verwendet, der das Team als einen einzelnen Charakter erhebt. Um jedoch
einen Multiprotagonisten erstellen zu können, muss ein Kriterium erfüllt
werden: Die Charaktere müssen dasselbe Ziel haben. Zusammengefasst
handelt es sich bei Multiprotagonisten um eine Gruppe von Akteuren, die
jedoch in ihrer Gesamtheit ebenso als ein einziger Charakter verstanden
werden können, dessen komplexe Attribute - aufgrund der leichteren
Verständlichkeit für Kinder - in verschiedene Persönlichkeiten
aufgeteilt wurden. Nur wenn eine Gruppe als solches verstanden werden
kann, kann ein Multiprotagonist erstellt werden. Bei den zuvor genannten
Geschichten des Peter Pan wäre die Konstruktion eines solchen
Multiprotagonisten beispielsweise nicht möglich gewesen, da sowohl Wendy
wie auch Peter Pan als eigenständige Charaktere begriffen werden müssen
und über kein gemeinsames Ziel verfügen. Zum leichteren Verständnis wird
hier ein Beispiel eines solchen Multiprotagonisten genannt und erklärt:

\subsubsection{Beispiel: Tom Turbo}

Das Dreiergespann Tom Turbo, Karo und Klaro können als ideales Beispiel
für einen Multiprotagonisten fungieren. Tom Turbo ist das tollste
Fahrrad der Welt mit zahlreichen Tricks, die auf der Verbrecherjagt von
Nutzen sein können. Seine Detektivkollegen Karo und Klaro sind ein
Geschwisterpaar. Karo ist ein taffes kleines Mädchen, dass sich ohne
viel scheu in ein Abenteuer wirft, genauso wie ihr Bruder Klaro der
oftmals sogar etwas nachdenklicher wirkt als seine Schwester. Sie
trennen sich während der Bewältiung ihrer Abenteur nie, wenn nicht einer
der drei das Opfer der Geschichte ist (Beispiel Entführung). Alle drei
gemeinsam haben, dasselbe Ziel und sind zumeist der gleichen Meinung.
Entsteht einmal ein Disput zwischen den beiden Geschwistern ähneln diese
einer Abwägung von Pros und Contras, die auch eine einzelne Person
gedanklich abarbeiten würde, stecke sie in einer ähnlichen Situation.
\parencite{brezina2000}

\section{Merkmale des inhaltlichen Aufbaus}

Aus dem Wissen, dass Buben und Mädchen unterschiedliche Lesepräferenzen
aufweisen, ergibt sich die Frage, worin sie sich unterscheiden. Es ist
bekannt, dass Buben verstärkt auf Sachbücher -- die in unserer Erhebung
bewusst auf Grund der fehlenden Darstellungen von Protagonisten nicht
erhoben wurden - ansprechen, Mädchen hingegen tendieren zu Büchern die
eine Geschichte erzählen. In unserer Erhebung haben wir ausschließlich
Bücher erhoben, die aus dieser Unterscheidung von Mädchen favorisiert
werden sollten. Dennoch konnte hier kein nennenswerter Unterschied in
der Menge des Gelesenen festgestellt werden, was es uns ermöglicht, die
unterschiedlichen Lesepräferenzen auf eine andere Form hin zu
untersuchen. Hierbei wurde analysiert, ob es sich bei den Büchern um
Abenteuer- oder Alltagsgeschichten handelt (siehe Tabelle \ref{genre}).

      
      \ctable[
      %  cap    = ,
        caption = {Inhaltliche Merkmale von Geschichten},
        label   = genre ,
        pos   = htp,
      %  width    = \textwidth
      ]{lccccc}{}{                  
      \FL \small Bücher &  
      \multicolumn{1}{c}{\small Phant} & 
      \multicolumn{1}{c}{\small Grow} & 
      \multicolumn{1}{c}{\small Mono} & 
      \multicolumn{1}{c}{ \small Quest} & 
      \multicolumn{1}{c}{ \small Abenteuer}
      \ML Die wilden Fußballkerle &  &    &     &     & Alltag
      \NN Tiger-Team          &     &     &     & x   & Abenteuer
      \NN Knickerbockerbande  &     &     &     & x   & Abenteuer
      \NN Gregs Tagebuch      &     &     & x   &     & Alltag
      \NN Harry Potter        & x   & x   &     & x   & Abenteuer
      \NN Die drei ???        &     &     &     & x   & Abenteuer
      \NN Das magische Baumhaus & x &     &     &     & Abenteuer
      \NN Der kleine Ritter Trenk & x &   &     & x   & Abenteuer
      \NN Tom Turbo           & x   &     &     & x   & Abenteuer
      \NN Der kleine Drache Kokosnuss & x & &   & x   & Abenteuer
      \NN Der Räuber Hotzenplotz & x &    &     & x   & Abenteuer
      \NN Sams                & x   &     &     &     & Alltag
      \NN Fünf Freunde        &     &     &     & x   & Abenteuer
      \NN Die Olchis          & x   &     &     &     & Alltag
      \NN Der Grüffelo        & x   &     &     &     & Alltag
      \NN Die Geggis          & x   & x   & x   &     & Abenteuer
      \NN Peter Pan           & x   & x   & x   &     & Abenteuer
      \NN Der Regenbogenfisch & x   & x   & x   &     & Alltag
      \NN Baumhausgeschichten &     &     &     &     & Alltag
      \NN Geschichten von Franz &   & x   & x   &     & Alltag
      \NN Pinocchio           & x   & x   &     &     & Abenteuer
      \NN Das kleine Wutmonster & x & x   &     &     & Alltag
      \NN Der kleine Eisbär   & x   & x   &     &     & Abenteuer
      \NN Pipi Langstrumpf    & x   &     &     &     & Alltag
      \NN Die kleine Hexe     & x   &     &     &     & Alltag
      \NN Hexe Lilli          & x   &     &     & x   & Alltag
      \NN Die wilden Hühner   &     &     &     & x   & Alltag
      \NN Mini                &     & x   & x   &     & Alltag
      \NN Conni               &     & x   &     &     & Alltag
      \NN Prinzessin Lillifee & x   &     &     & x   & Abenteuer 
      \LL
      }
      

\subsection{Alltagsgeschichten}

Alltagsgeschichten spielen in einem dem Hauptprotagonisten vertrauten
Umfeld. Bei kindlichen Protagonisten handelt es sich zumeist um die
familiäre und/oder schulische Umgebung. Es werden Themen und
Problematiken angesprochen, die im realen Leben der Leser und Leserinnen
mit großer Wahrscheinlichkeit vorkommen können. Beispiele dafür sind
Beziehungsprobleme mit Freunden, Eltern oder Lehrern, aber auch
Leistungsdruck in der Schule, Erlebnisse auf Klassenfahrten, Urlaube
oder der Tod von Haustieren.

\subsubsection{Beispiel: Hexe Lilli}

``Das ist Lilli, die Hauptperson unserer Geschichte. Sie ist ungefähr so
alt wie du und sieht aus wie ein gewöhnliches Kind.''
\parencite[][6]{KNISTER1999} Bereits dieser Satz, mit dem die Erzählung
beginnt, verrät viel darüber wie versucht wird, den Leser/ die Leserin
in die Geschichte zu integrieren, was in späterer Folge nicht schwer
fällt, da Lilli Situationen durchlebt, die wohl keinem gänzlich
unbekannt sind. Zankerein mit dem kleinen Bruder, sowie Unverständnis
über die Einstellungen der Eltern gehören wohl zum vielfältigen
Kinderalltag.

Situationsbeschreibung: Der Schulrat besucht an diesem Tag die Klasse
von Lilli und möchte den Unterricht von Frau Grach der Klassenlehrerin
inspizieren. Lilli möchte der Frau Lehrerin gerne helfen einen guten
Eindruck zu hinterlasssen, doch der Herr Schulrat taucht natürlich genau
im falschen Moment auf als das totale Chaos in der Klasse herrscht.

\begin{quote}
``Auweia'', flüstert Lilli. So war das nicht gedacht! Hier muss sie
schnell eingreifen bevor der Schulrat gleich zu Anfang einen schlechten
Eindruck bekommt. \parencite[][47
]{KNISTER1999}
\end{quote}

Lilli ist tatkräftig und dominant aber zugleich auch hilfsbereit und
großherzig. Sie bietet aus Sicht des Doing-Gender einen Mix an
Eigenschaften, der sich auch im Wert der Gendertabelle (siehe
Tabelle\ldots{}) widerspiegelt. Lilli wird sehr klar bevorzugt von
Mädchen gelesen und zeigt, wie Mädchen ebenfalls mit maskulinen
Genderatributen dargestellt werden. Sie ist ein Paradebeispiel dafür,
dass Mädchen mit beiden Genderrollen konfrontiert werden, was Buben
gemeinhin noch verwehrt wird.

\subsection{Abenteuergeschichten}

Abenteuergeschichten sind das Gegenstück zu Alltagsgeschichten. Dabei
durchlebt der Hauptprotagonist ein wahrscheinlich einzigartiges
Erlebnis, das zumeist mit großen Risiken und Gefahren verbunden ist. Der
Protagonist ist dabei zumeist gezwungen sein gewohntes Umfeld zu
verlassen und sich in völlig fremden oft auch unrealistischen
Situationen zurechtzufinden. Beispiele hierfür wären die Suche nach
einem verschollenen Schatz, das Tätigen einer gefährlichen und
ungewissen Reise, das Kämpfen mit bösen Mächten wie Ganoven oder Drachen
usw.

\subsubsection{Beispiel: Harry Potter}

Harry ist ein schmächtiger Junge, der bei der Familie seiner Tante lebt,
da seine Eltern gestorben sind. Das allerdings nur so lange bis er
erfährt, das er ein Zauberer ist und auf die Zauberschule kommt. Dort
angekommen erlebt er ein Abenteuer nach dem anderen. Diese Gipfeln in
einem großen und brutalen Show-Down im Kampf gegen den Mörder seiner
Eltern. \parencite{Rowling1998}

Harry Potter hat viele Attribute die feminin deklariert sind, so ist er
beispielsweise großherzig und emotional, manchmal sogar etwas
träumerisch, aber auch mutig und aktiv. Er ist teilweise sehr dominant
und hält sich nicht an Regeln. Diese Attribute lassen den Genderwert
leicht ins maskuline wandern. Sein Genderwert ist beispielsweise jenem
von Hexe Lilli nicht unähnlich, die Bücher werden schließlich auch von
vielen Mädchen gerne gelesen, jedoch tendenziel eher von Jungen. Harry
Potter ist ein passendes Beispiel dafür, dass Jungen männliche
Protagonisten, vor allem aber auch Abenteuergeschichten favorisieren.

Wie wir auf Tabelle \ref{genre}. erkennen lesen Buben verstärkt
Abenteuergeschichten während Mädchen einen viel höhere Anzahl an
Alltagsgeschichten in ihrer Lesepräferenz vorweisen. Dies kann auch mit
einer Korrelation von 0,314 (Sig. 0,091) statistisch festgehalten
werden. Auch hier finden wir dasselbe Bild, dass Mädchen in beiden
Ausprägungen zu finden sind, Buben hingegen nur sehr wenige
Alltagsgeschichten lesen.

Doch hat diese unterschiedliche Präferenz eine Auswirkung auf die
unterschiedliche Entwicklung von Geschlechterausprägungen? Diese Frage
kann anhand der erhobenen Daten nicht beantwortet werden und dennoch
kann sie dazu nutzen eine weitere Frage aufzuwerfen, die einen
Anhaltspunkt für die Beantwortung liefern kann. Aus unserer Erhebung zur
Darstellung von Gendermerkmalen (siehe Genderfaktor) wissen wir, dass
gewisse Eigenschaften als besonders maskulin oder feminin empfunden
werden. Wir haben uns daher gefragt, ob es denn Merkmale im Inhalt und
Aufbau von Kinderbüchern geben könnte, die die Ausprägung solcher
Eigenschaften unterstützen. Da die Ergebnisse für Jungen viel
einseitiger ausgefallen sind, kann die Frage auch wie folgt formuliert
werden. Haben beispielsweise Abenteuergeschichten bestimmte Merkmale,
mit denen Buben verstärkt konfrontiert werden und können sie als ein
möglicher Faktor zur Entwicklung \emph{maskuliner} Eigenschaften
beitragen? Oder: Fehlen durch das sparsame Lesen von Alltagsgeschichten
bestimmte Merkmale, die eine femininere Entwicklung verhindern?

Folgende vier Kriterien wurden erhoben, bei denen von einem Einfluss auf
die Geschlechterrollenentwicklung ausgegangen wurde:

\subsubsection{Quest}

Verläuft die Geschichte des Buches auf ein bestimmtes Ziel hin, das
erreicht werden soll? Erfordert das Erreichen des Zieles das Lösen von
Aufgaben bzw. Rätseln? Besonders Kriminal- und Detektivgeschichten sind
mit einem obersten Ziel verknüpft, dass erreicht werden soll. Auf dem
Weg bis zur Lösung stellen sich dem Protagonisten Stolpersteine in den
Weg, die zuerst entfernt werden müssen. Dies braucht oft rationales
Denken, Mut, aktives Handeln oftmals auch körperliche Stärke und
Aggression. All diese Attribute werden im Sinne des Doing-Gender als
maskuline Eigenschaften wahrgenommen und könnten daher eine spezifische
Geschlechterrollenentwicklung miterklären. Es ist daher wenig
überraschend, dass die Präsenz von Quests in enger Verbindung mit
Abenteuergeschichten steht. Bei einer hochsignifikanten Korrelation von
0,517 kann daher auch eine Verknüpfung mit einem verstärkten Vorkommen
in den Büchern mit männlicher Lesepräferenz ausgegangen werden.

\paragraph{Beispiel \emph{Die Knickerbockerbande}}

Die Knickerbockerbande besteht aus Lilo, Axel, Dominik und Poppi, die in
jedem Band neue \emph{Rätsel} lösen. Dabei kann es vorkommen, dass sie
etwa im Urlaub auf mysteriöse Fälle stoßen, die sie dann meist zu viert
aufklären. Die Geschichten sind spannend, die vier geraten öfter in
Gefahr oder in die Hände von Verbrechern, aus denen sie sich aber mit
List und Geschick wieder befreien. Dies ist ein eindeutiger Indikator
für das Merkmal des Quests. Die Hauptfiguren haben unterschiedliche
Qualitäten, können aber als ein Multiprotagonist verstanden werden. Auch
das Verhalten untereinander ist sehr hilfsbereit, sie sind verlässlich
und sie alle vereint dasselbe Hobby, nennen wir es ``Dedektiv spielen'',
worauf sie sich auch in ihrer Freizeit vorbereiten und trainieren, wie
man sich zum Beispiel anschleicht oder besonders schnell ist. Die
Geschichten wirken anfangs mysteriös, was auch die Titel wiedergeben,
die manchmal gruselige und surreale Situationen zu versprechen scheinen,
die sich dann aber immer als menschengemacht herausstellen.

\begin{quote}
Die Männer in den roten Mänteln lagen kraftlos am Boden. \textelp{} Axel
waren sofort die kleinen roten Federbüschel aufgefallen, die ihnen
seitlich aus dem Hals ragten. Sie dienten einer kleinen Nadel als
Stabilisator. Solche Nadeln wurden aus Blasrohren abgefeuert. Axel
erinnerte sich, etwas im Fernsehen darüber gesehen zu haben.
\parencite[][117]{Brezina2010}
\end{quote}

Die Protagonisten agieren sehr rational. Sie können Situationen gut
einschätzen und verknüpfen das Wissen aus anderen Informationsquellen
mit dem Erlebten. Diese Eigenschaft hilft ihnen dabei die Rätsel zu
lösen und ihr Ziel zu erreichen.

\subsubsection{Phantastische Elemente}

Kommen in den Büchern Figuren, Orte oder Handlungen vor, die in der
Realität nicht vorkommen? Beispiele: Einhörner, sprechende Tiere,
fliegende Menschen, Zauberer, fremde Welten, uvm. Phantastische Elemente
könnten einen Hang zum träumerischen, irrationalen Denken fördern,
welches aus der Sicht des Doing-Genders feminine Attribute wären. Doch
Abenteuergeschichten sind natürlich gespickt mit unmöglichen
Situationen. Oftmals bekämpfen Charaktere Monster und Gespenster. Daher
tendieren die Zahlen dazu, das Vorkommen von phantastischen Elementen
den Abenteuerbüchern zuzuschreiben. Das niedrige Signifikanzniveau lässt
hier jedoch keine genaue Aussage zu. Es kann auch kein Zusammenhang mit
dem w/m-Faktor gefunden werden. Dieses inhaltliche Merkmal scheint in
beiden Geschlechtsgruppen annähernd gleich oft verwendet zu werden und
kann daher keine Annäherung zur Erklärung von Geschlechtsrollenbildung
liefern.

\subsubsection{Innerer Monolog}

Welche Rolle spielt die Gedankenwelt des Hauptprotagonisten? Wie stark
reflektiert er seine Entscheidungen vor und/oder nach dem Handeln? Wie
intensiv wird sie dem Leser/der Leserin vermittelt? Mädchen gelten als
passiver und introvertierter als ihre männlichen Altersgenossen und
haben aus Sicht des Doing-Gender ein größeres Einfühlungsvermögen als
Jungen. All dies wären Indizien den Inneren Monolog als ein Merkmal zu
deklarieren, das Jungen fehlen könnte eine feminine Seite zu entwickeln.
Und tatsächlich tendieren die Zahlen unserer Ergebnisse dazu einen
Zusammenhang von Alltagsgeschichten und Innerem Monolog zu bescheinigen.
Auch hier ist jedoch das Signifikanzniveau zu niedrig um fixe Aussagen
zu tätigen. Auffällig ist, dass das Merkmal des Inneren Monologs negativ
mit dem Merkmal Quests korreliert (R= -0,333; Sig. 0,083). Das bedeutet,
dass das kombinierte Vorkommen dieser beiden Merkmale äußerst selten
anzutreffen ist.

\paragraph{Beispiel Mini}

In den Mini-Büchern geht es darum den frühen Alltag eines Kindes zu
bewältigen und persönliche Konflikte auf sehr humorvolle Art aus Minis
Sicht wiederzugeben.

Mini ist schon sehr groß für ihr Alter und gleichzeitig sehr dünn,
weshalb ihr auch alle möglichen Spitznamen gegeben werden, was sie
kränkt. Der Schule blickt sie mit gemischten Gefühlen entgegen,
gleichzeitig freut sie sich schon drauf, hat aber auch Angst in die
falsche Schule zu kommen, von der die falschen Lehrerin unterrichtet zu
werden oder vor den fremden Kindern, die sie wieder hänseln könnten.
Dies zeigt wie viel sie reflektiert und über mögliche Situationen und
Folgen nachdenkt. Als Beispiel kann hier der Gedankengang genannt
werden, der zeigt wie erleichtert sie darüber ist, dass sie nicht die
größte in ihrer Klasse ist:

\begin{quote}
Und die Mini fing vor lauter Staunen zu schielen an. \textelp{} Warum
die Mini so erstaunt und verblüfft war? Weil sie garantiert nicht das
größte Kind ihrer Klasse war! Ein Bub und ein Mädchen waren noch ein
bisschen größer als die Mini, 2 Buben und 2 Mädchen waren genauso groß
wie die Mini. Die Mini
dachte:\emph{Wenn es unter zwanzig Kindern sieben \emph{lange Latten} gibt, dann ist ja die Überlänge direkt normal!}\parencite[][61]{Noestlinger2011}
\end{quote}

\subsubsection{Growing-Up:}

Verändert sich im Verlauf der Geschichte die Persönlichkeit des
Hauptprotagonisten? Durchläuft er einen Reifeprozess? Erhebt das Buch
den Anspruch eine pädagogische Nachricht zu vermitteln, hingerichtet auf
eine positive Sozialisierung? Mädchen gelter oftmals im Vergleich zu den
gleichaltrigen Jungen als sozial weiter entwickelt. Dieser Vorsprung in
der Entwicklung könnte zum Teil durch eine vermehrte Konfrontation mit
pädagogisch motivierter Literatur mitbegründet sein. Es kann hier jedoch
kein Zusammenhang festgestellt werden. Lediglich eine starke
signifikante Korrelation mit dem Merkmal des Inneren Monologs kann hier
festgestellt werden, sowie eine stark signifikante negative Korrelation
mit dem Merkmal Quest, wie auch schon beim Merkmal des Inneren Monologs.

Die Untersuchung der vier inhaltlichen Merkmale von Alltags- und
Abenteuergeschichten konnte vor allem zeigen, dass eine Analyse anhand
von 30 Büchern keine wirklichen aussagekräftigen Ergebnisse liefern kann
und hier eine große eigenständige Untersuchung notwendig wäre um etwaige
Zusammenhänge zwischen inhaltlichen Merkmalen und der Ausprägungen von
geschlechterspezifischen Eigenschaftsmerkmalen notwendig wäre. Somit
bleibt hier als einziges aussagekräftiges Ergebnis nur die Erkenntnis,
dass sich die Buben bei ihrer Lesepräferenz vor allem auf
Abenteuergeschichten konzentrieren, Mädchen hingegen auch
Alltagsgeschichten lesen. Als kleiner Hoffnungsschimmer am Firmament ist
die Untersuchung des Merkmals Quests zu sehen, die gezeigt hat, dass das
Vorkommen eines Merkmals mit bestimmten Eigenschaften die als maskulin
deklariert sind in Verbindung gebracht werden und eine größere
Untersuchung womöglich aussagekräftigere Ergebnisse liefern könnte.

\section{Fazit und Verknüpfung mit der Theorie}

Als großer Triumpf dieses Kapitels ist die Untersuchung der
Gendermerkmale von Hauptprotagonisten in Kinderbüchern zu sehen. Sie hat
uns gezeigt, dass Buben vor allem über maskuline Charaktere lesen,
Mädchen zu weiblichen tendieren aber mit beiden konfrontiert werden.
Weiterhin werden manche stereotype Eigenschaften bestimmten biologischen
Geschlechtern zugeschrieben und verhindern damit den Prozess des
Gender-Mainstreaming. Das soziale Geschlecht kann hier ergänzend zum
biologischen Geschlecht gesehen werden um das Doing-Gender von
Kinderbuchcharakteren zu erklären. Die Regression zeigt uns, dass die
Wahrscheinlichkeit einer Erklärung steigt, wenn beide Faktoren
miteinander kombiniert werden.(R bei biologischem Geschlecht: 0.58; bei
Gender: 0.19; Kombiniert: 0.66).

Der Versuch dieses Ergebnis mit Theorie zu verknüpfen kann zu
provokanten Aussagen führen, die hier nur genannt werden um etwaige
Untersuchungen in der Zukunft motivieren. So kann etwa unser Wissen,
dass Mädchen und Buben unterschiedliche Bücher lesen mit der Theorie
verknüpft werden, dass das Verhalten von Protagonisten in Büchern auf
deren Leser abfärbt und somit Verhalten reproduziert. Diese Verknüpfung
würde aus Sicht der Ergebnisse dieser Untersuchung wie folgt zu
interpretieren sein: Wenn Buben verstärkt mit maskulinen Protagonisten
konfrontiert werden, könnte diese Einseitigkeit zu einer Stabilisierung
von stereotypen Geschlechterrollen führen. Bei Mädchen hingegen sagen
uns die Ergebnisse, dass sie inzwischen mit vielfältigeren Eigenschaften
und Handlungsalternativen konfrontiert werden und daher im Sinne des
Gender-Mainstreamings eine größere Chance haben, bestehende stereotype
Rollenbilder zu brechen und neu zu gestalten.

\chapter{Merkmale die das Leseverhalten erklären}

Drei Merkmale eines Kinderbuchs reichen aus, um das Verhältnis von
Leserinnen zu Lesern bei einem Kinderbuch bestimmen zu können: das
\emph{Geschlecht der Titelfigur}, die \emph{Helligkeit} und die
\emph{Anzahl der Seiten}. Die Genauigkeit eines linearen Modells mit
diesen drei Merkmalen ist mit einem korrigierten Bestimmtheitsmaß von
$0{,}82$ sehr genau. Wobei das Vorhandensein einer weiblichen Namens im
Titel am meisten zu dem Modell beiträgt ($\beta=-0{,}77$). Danach kommt
die Helligkeit des Covers ($\beta=0{,}29$). Die Anzahl der Seiten dient
dann nur noch zu Verfeinerung ($\beta=0{,}19$). Stellte man eigene
Modelle für das Geschlecht der Titelfigur sowie für die Coverhelligkeit
auf sieht man, dass Beide auch alleine noch einen beachtlichen Teil
erklären. (Siehe Tabelle~\ref{wmmodel})

All diese Merkmale können von Kindern ohne Probleme und ohne dass sie
das Buch aufmachen müssen wahrgenommen werden. Unsere beiden Fragen, ob
Merkmale des Buchs das Verhältnis von Leserinnen zu Lesern erklären und
ob sie das ohne das Buch zu öffnen können, können wir eindeutig mit
\emph{ja} beantworten. Steht im Titel ein weiblicher Name, ist das Buch
noch dazu sehr hell und obendrein auch noch dünn. Dann ist die
Wahrscheinlichkeit sehr hoch, dass das Buch viel mehr Mädchen als Buben
gelesen haben. Ist das Buch dunkel, dick und kommt auch noch ein
männlicher Name im Titel vor, ist es wahrscheinlicher, dass mehr Buben
als Mädchen das Buch gelesen haben.

      
      \ctable[
      %  cap    = ,
        caption = {Lineare Modelle die den w/m-Faktor erklären},
        label   = wmmodel ,
        pos   = htp,
      %  width    = \textwidth
      ]{lD{,}{,}{2}D{,}{,}{2}D{,}{,}{2}}{
        % \tnote[.]{< 0,1}
        % \tnote[*]{< 0,05}
        % \tnote[**]{< 0,01}
        % \tnote[***]{< 0,001}
      }{                  
      \FL 
      \small   &  
      \multicolumn{1}{c}{\small Modell 1} & 
      \multicolumn{1}{c}{\small Modell 2} & 
      \multicolumn{1}{c}{\small Modell 3}
      \ML Geschlecht Titelfigur (unbestimmt)    & 0,06      & 0,02      &
      \NN Geschlecht Titelfigur (weiblich)      & -0,77\tmark[***]     & -0,82\tmark[***]     & 
      \NN Geschlecht Titelfigur (männlich)      & \multicolumn{1}{c}{ref.}      & \multicolumn{1}{c}{ref.}      & 
      \NN Coverhelligkeit                       & -0,29\tmark[**]     &           & -0,46\tmark[**]
      \NN Seitenanzahl                          & 0,19\tmark[*]      &           & 
      \ML Korrigiertes $R^2$                    & 0,82\tmark[***]      & 0,69\tmark[***]      & 0,18\tmark[**]  
      \LL \multicolumn{4}{l}{\footnotesize  $^*p<0{,}05, ^{**}p<0{,}01, ^{***}p<0{,}001$}
      }
      

\section{Für Mädchen und Buben sind unterschiedliche Merkmale
ausschlaggebend}

Dies heißt jedoch nicht, dass die drei Merkmale auf Mädchen und Buben
denselben Einfluss haben. Die Wahrscheinlichkeit, dass Mädchen oder
Buben ein Buch lesen, hängt mit unterschiedlichen Merkmalen von Büchern
zusammen. Dafür, dass ein Buch hauptsächlich von Mädchen gelesen wird,
ist es wichtig, dass das Buch von einer Frau geschrieben wurde
($R^2=0{,}19; p=0{,}04$), wiederum, dass die Figur im Titel weiblich ist
($R^2=0{,}18; p=0{,}03$) und dass wenige Figuren am Cover
($R=-0{,}37; p=0{,}4$) sichtbar sind. Insgesamt hat das Modell mit
diesen drei Merkmalen ein korrigiertes Bestimmtheitsmaß von $0{,}33$
($p=0{,}02$). Die Helligkeit und die Anzahl der Seiten ist für die
Anzahl der Mädchen die ein Buch lesen irrelevant.

Diese Merkmale sind für die Häufigkeit bei den Buben natürlich um so
wichtiger. (Helligkeit: $R^2=0{,}25$; Seiten: $R^2=0{,}16$; $p=0{,}01$)
Das lässt auch darauf schließen, dass grundsätzlich das Leseverhalten
von Buben für das Verhältnis zwischen Mädchen und Buben relevanter ist.
Und tatsächlich ist die Korrelation zwischen der Häufigkeit der
Nennungen pro Buch bei den Buben und dem Verhältnis der Nennungen
zwischen Mädchen und Buben mit $0{,}70$ größer als zwischen den Mädchen
und dem Verhältnis, dass nur eine Korrelation von $-0{,}41$ aufweist. Da
die Nennungen der Buben für unser Verhältnis so wichtig sind, fangen wir
hier mit einer detaillierteren Analyse der Merkmale an.

\section{Das Geschlecht der Titelfigur}

Der erste Einflussfaktor ist das Geschlecht der Figur, die im Titel
genannt wird. Das ist in den meisten Fällen auch die Hauptfigur, also
die Figur mit der sich die Leserin oder der Leser am wahrscheinlichsten
identifiziert. Nur bei wenigen Geschichten ist die Figur, die am Titel
erwähnt wird, nicht die eigentliche Protagonistin bzw. der eigentliche
Protagonist. Auch wenn die Hauptfigur eine andere ist, heißt das noch
immer nicht, dass sich auch das Geschlecht unterscheidet. Zum Beispiel
ist in \emph{der Räuber Hotzenplotz} die Hauptfigur der Kasperl, aber
beide sind männlich. In \emph{Grüffelo} ist die Hauptfigur eine Maus und
beide sind \emph{neutral}. In unseren 30 meist genannten Büchern bleibt
nur ein Buch übrig, bei denen sich das Geschlecht der Titelfigur und der
Hauptfigur unterscheiden und hier handelt es sich um einen Streitfall.
Gemeint ist \emph{Peter Pan}, bei dem, im Original, Wendy die
Protagonistin ist. Jedoch ist bei vielen Adaptionen der Fokus ganz zu
Peter gewandert. Eine andere Möglichkeit einer Differenz zwischen den
beiden Merkmalen ist, dass das Geschlecht der Hauptfigur nicht vorkommt
oder nicht eindeutig bestimmbar ist.

Das Geschlecht der Hauptfigur ist ein Merkmal, über das die Autorin oder
der Autor die völlige Kontrolle haben. Das Geschlecht der Hauptfigur
entsteht meist ganz am Anfang und hat insgesamt gesehen den größten
Erklärungswert für das Gesamt-Modell und ist für Mädchen und Buben
relevant.

\section{Buben lesen keine hellen Bücher}

Das nächste wichtige Merkmal ist die Cover-Helligkeit eines Buchs.
Dieses Merkmal hat bei Buben immerhin einen gleich großen Erklärungswert
wie das Geschlecht der Titelfigur. Die Entstehung dieses Merkmals ist
jedoch schon nicht mehr direkt mit der Autorin oder dem Autor zu
verbinden. Das Cover wird zu einem Zeitpunkt, an dem die Geschichte
schon längst an einen Verlag verkauft worden ist, gestaltet. Es kann
auch vorkommen, dass das Cover bei neueren Fassungen komplett anders
gestaltet wurde. Der Verlag hat die Aufgabe die Geschichte an den
Endkunden zu verkaufen. Das heißt, es ist seine Aufgabe, Kindern, deren
Eltern und weiteren potenziellen Käufern die Entscheidung zu
erleichtern.

Wir vermuten, dass die Verlage herausgefunden haben, dass dunkle
\emph{coole} Bücher Buben eher ansprechen als lieblich helle, rosa
Bücher. Zusätlich muss der Verlag eine Entscheidung treffen, für wen die
Geschichte gedacht ist. Der Verlag hat für diese Zeit mehr Ressourcen
als der Endkunde. Hier werden Inhalte eines Buches von den dafür
zuständigen Personen im Cover ausgedrückt und gewissermaßen
\emph{übersetzt}. Dabei wirkt es nicht überraschend, dass sie sich an,
in der Gesellschaft verfestigten Geschlechterrollenbildern orientieren.
Tatsächlich hat der \emph{Gender-Faktor} auf die Helligkeit den größten
Einfluss ($R=-0{,}51$). Gemeinsam mit dem Geschlecht der Hauptfigur
lässt sich die Helligkeit schon recht gut voraussagen
($R^2=0{,}24; p=0{,}02$). So ist die Helligkeit ein gutes
\emph{Transportmittel} um den Gender-Faktor ankommen zu
lassen.\footnote{Wir gehen davon aus, dass weitere Merkmale des Covers, die wir nicht operationalisiert haben, wie die Form der Darstellung oder die Komplexität des Bildes noch einen wesentlichen Anteil zur Übersetzung des Genderfaktor beitragen.}

Nicht übersehen darf man, dass nur das Leseverhalten von Buben von der
Helligkeit beeinflusst wird. Bei den Mädchen kann kein Zusammenhang mit
der Helligkeit nachgewiesen werden. Das heißt Mädchen lesen genauso
helle wie dunkle Bücher. Buben meiden jedoch helle Bücher. Das zeigt,
dass Buben es eher vermeiden mädchenhafte Literatur zu konsumieren,
während der Spielraum der Mädchen hier weniger eingeschränkt wird.

\section{Buben bevorzugen Bücher für Ältere}

Ein weiterer Einfluss auf das Leseverhalten, speziell von Buben, ist die
Dicke eines Buchs beziehungsweise das eng damit zusammenhängende
empfohlene Alter. Und zwar steigt mit der Dicke der Bücher auch die
Anzahl der männlichen Leser. Auf den ersten Blick widerspricht dieser
Fakt den Ergebnissen der Lesesozialisationsforschung, in der Buben meist
als \emph{Lesemuffel} dargestellt werden. Vor allem weil das
Leseverhalten von Mädchen dadurch wiederum nicht nachweisbar beeinflusst
wird. Weiters kann man hier auch nicht klar zu sagen welches Merkmal,
Alter oder Dicke, eigentlich wirksam ist.

Um das Wirken des Merkmalpaares haben wir zwei Vermutungen. Die erste
bezieht sich darauf, dass Mädchen früher zu lesen beginnen. Wir haben
die Kinder gefragt, welche Bücher sie gelesen haben. Die befragten
Kinder waren zwischen 8 und 10 Jahren und es ist durchaus vorstellbar,
dass die Mädchen früher zum Lesen von \emph{Geschichten-Büchern}
anfangen. Das heißt, dass sie davor weniger oder andere von uns nicht
untersuchte Bücher, wie die bei den Buben sehr beliebten Sachbücher,
lesen. Die zweite Vermutung bezieht sich auf den \emph{Coolness-Faktor}.
Das heißt, das es für Buben wichtiger ist \emph{cool} zu sein. So kann
sich von unserer Forschungsgruppe ein männliches Mitglied noch sehr gut
erinnern, dass das empfohlene Alter hinten auf den Büchern, für ihn,
gerade im Alter der Untersuchten, sehr wichtig war.

\section{Der Einfluss des Geschlechts der Autorin/des Autors ist zu
vernachlässigen}

Wenden wir uns wieder dem Modell, dass die Häufigkeiten der Mädchen
erklären soll, zu. Davon haben wir das für die Mädchen zweitwichtigste
Merkmal, das Geschlecht der Titelfigur, schon analysiert. Jedoch kommt
bei den Mädchen ein weiteres \emph{Geschlechts-Merkmal} hinzu. Das
Geschlecht der Autorin/des Autors. Bei diesem Oberflächenmerkmal ist für
die Buben kein Zusammenhang nachweisbar.

Aber auch die Erklärungskraft bei den Mädchen ist nicht überzubewerten,
da sie zu einem sehr großen Teil aus einem sehr gewichtigen
\emph{Ausreißer} besteht. Der/die Autor\_in von \emph{Der Hexe Lilli},
dem Buch, das bei den Mädchen das Ranking anführt nennt sich
\emph{Knister}. Hinter dem Pseudonym steckt ein Mann, jedoch entschieden
wir uns, für die Cover- Analyse nur eindeutig feststellbare Geschlechter
anzuführen. Da es sich hier um einen Ausnahmefall handelt und
\emph{Knister} das einzige neutrale Autorengeschlecht auf den ersten
Blick darstellt und dieses Buch von den Mädchen am häufigsten gelesen
wurde, erklärt warum dieser Wert, wenn überhaupt, nur mit besonderer
Vorsicht interpretiert werden kann. Vor allem da sich die Werte zwischen
weiblich und männlich nicht signifikant unterscheiden. (Siehe
Abbildung~\ref{maedchen-geschlecht})

\begin{figure}
\center
  \caption[Leserinnen--Geschlecht]{Anzahl der Leserinnen zu Geschlecht der AutorIn}
  \label{maedchen-geschlecht}
% Created by tikzDevice version 0.6.2-92-0ad2792 on 2013-01-10 17:41:47
% !TEX encoding = UTF-8 Unicode
\begin{tikzpicture}[x=1pt,y=1pt]
\definecolor[named]{fillColor}{rgb}{1.00,1.00,1.00}
\path[use as bounding box,fill=fillColor,fill opacity=0.00] (0,0) rectangle (361.35,361.35);
\begin{scope}
\path[clip] (  0.00, 68.86) rectangle ( 18.07,320.51);
\definecolor[named]{drawColor}{rgb}{0.00,0.00,0.00}

\path[draw=drawColor,line width= 0.4pt,line join=round,line cap=round] (  0.67,111.47) --
	( 17.40,111.47) --
	( 17.40,193.81) --
	(  0.67,193.81) --
	(  0.67,111.47);

\path[draw=drawColor,line width= 0.4pt,line join=round,line cap=round] (  0.67,176.29) --
	( 17.40,176.29);

\path[draw=drawColor,line width= 0.4pt,line join=round,line cap=round] (  9.03, 78.18) --
	(  9.03,111.47);

\path[draw=drawColor,line width= 0.4pt,line join=round,line cap=round] (  9.03,193.81) --
	(  9.03,311.19);
\end{scope}
\begin{scope}
\path[clip] ( 58.90,  0.00) rectangle (340.43, 18.07);
\definecolor[named]{drawColor}{rgb}{0.00,0.00,0.00}

\path[draw=drawColor,line width= 0.4pt,line join=round,line cap=round] (156.22,  0.67) --
	(156.22, 17.40) --
	(257.60, 17.40) --
	(257.60,  0.67) --
	(156.22,  0.67);

\path[draw=drawColor,line width= 0.4pt,line join=round,line cap=round] (221.39,  0.67) --
	(221.39, 17.40);

\path[draw=drawColor,line width= 0.4pt,line join=round,line cap=round] ( 69.33,  9.03) --
	(156.22,  9.03);

\path[draw=drawColor,line width= 0.4pt,line join=round,line cap=round] (257.60,  9.03) --
	(330.01,  9.03);
\end{scope}
\begin{scope}
\path[clip] (  0.00,  0.00) rectangle (361.35,361.35);
\definecolor[named]{drawColor}{rgb}{0.00,0.00,0.00}

\path[draw=drawColor,line width= 0.4pt,line join=round,line cap=round] ( 86.71, 68.86) -- (312.63, 68.86);

\path[draw=drawColor,line width= 0.4pt,line join=round,line cap=round] ( 86.71, 68.86) -- ( 86.71, 63.88);

\path[draw=drawColor,line width= 0.4pt,line join=round,line cap=round] (124.36, 68.86) -- (124.36, 63.88);

\path[draw=drawColor,line width= 0.4pt,line join=round,line cap=round] (162.02, 68.86) -- (162.02, 63.88);

\path[draw=drawColor,line width= 0.4pt,line join=round,line cap=round] (199.67, 68.86) -- (199.67, 63.88);

\path[draw=drawColor,line width= 0.4pt,line join=round,line cap=round] (237.32, 68.86) -- (237.32, 63.88);

\path[draw=drawColor,line width= 0.4pt,line join=round,line cap=round] (274.98, 68.86) -- (274.98, 63.88);

\path[draw=drawColor,line width= 0.4pt,line join=round,line cap=round] (312.63, 68.86) -- (312.63, 63.88);

\node[text=drawColor,anchor=base,inner sep=0pt, outer sep=0pt, scale=  0.83] at ( 86.71, 50.94) {-0.6};

\node[text=drawColor,anchor=base,inner sep=0pt, outer sep=0pt, scale=  0.83] at (124.36, 50.94) {-0.4};

\node[text=drawColor,anchor=base,inner sep=0pt, outer sep=0pt, scale=  0.83] at (162.02, 50.94) {-0.2};

\node[text=drawColor,anchor=base,inner sep=0pt, outer sep=0pt, scale=  0.83] at (199.67, 50.94) {0.0};

\node[text=drawColor,anchor=base,inner sep=0pt, outer sep=0pt, scale=  0.83] at (237.32, 50.94) {0.2};

\node[text=drawColor,anchor=base,inner sep=0pt, outer sep=0pt, scale=  0.83] at (274.98, 50.94) {0.4};

\node[text=drawColor,anchor=base,inner sep=0pt, outer sep=0pt, scale=  0.83] at (312.63, 50.94) {0.6};

\path[draw=drawColor,line width= 0.4pt,line join=round,line cap=round] ( 58.90, 97.46) -- ( 58.90,307.69);

\path[draw=drawColor,line width= 0.4pt,line join=round,line cap=round] ( 58.90, 97.46) -- ( 53.92, 97.46);

\path[draw=drawColor,line width= 0.4pt,line join=round,line cap=round] ( 58.90,132.49) -- ( 53.92,132.49);

\path[draw=drawColor,line width= 0.4pt,line join=round,line cap=round] ( 58.90,167.53) -- ( 53.92,167.53);

\path[draw=drawColor,line width= 0.4pt,line join=round,line cap=round] ( 58.90,202.57) -- ( 53.92,202.57);

\path[draw=drawColor,line width= 0.4pt,line join=round,line cap=round] ( 58.90,237.61) -- ( 53.92,237.61);

\path[draw=drawColor,line width= 0.4pt,line join=round,line cap=round] ( 58.90,272.65) -- ( 53.92,272.65);

\path[draw=drawColor,line width= 0.4pt,line join=round,line cap=round] ( 58.90,307.69) -- ( 53.92,307.69);

\node[text=drawColor,rotate= 90.00,anchor=base,inner sep=0pt, outer sep=0pt, scale=  0.83] at ( 46.95, 97.46) {40};

\node[text=drawColor,rotate= 90.00,anchor=base,inner sep=0pt, outer sep=0pt, scale=  0.83] at ( 46.95,132.49) {60};

\node[text=drawColor,rotate= 90.00,anchor=base,inner sep=0pt, outer sep=0pt, scale=  0.83] at ( 46.95,167.53) {80};

\node[text=drawColor,rotate= 90.00,anchor=base,inner sep=0pt, outer sep=0pt, scale=  0.83] at ( 46.95,202.57) {100};

\node[text=drawColor,rotate= 90.00,anchor=base,inner sep=0pt, outer sep=0pt, scale=  0.83] at ( 46.95,237.61) {120};

\node[text=drawColor,rotate= 90.00,anchor=base,inner sep=0pt, outer sep=0pt, scale=  0.83] at ( 46.95,272.65) {140};

\node[text=drawColor,rotate= 90.00,anchor=base,inner sep=0pt, outer sep=0pt, scale=  0.83] at ( 46.95,307.69) {160};

\path[draw=drawColor,line width= 0.4pt,line join=round,line cap=round] ( 58.90, 68.86) --
	(340.43, 68.86) --
	(340.43,320.51) --
	( 58.90,320.51) --
	( 58.90, 68.86);
\end{scope}
\begin{scope}
\path[clip] ( 18.07, 18.07) rectangle (361.35,361.35);
\definecolor[named]{drawColor}{rgb}{0.00,0.00,0.00}

\node[text=drawColor,anchor=base,inner sep=0pt, outer sep=0pt, scale=  0.83] at (199.67, 31.02) {Gender-Faktor};

\node[text=drawColor,rotate= 90.00,anchor=base,inner sep=0pt, outer sep=0pt, scale=  0.83] at ( 27.03,194.69) {Häufigkeit Mädchen};
\end{scope}
\begin{scope}
\path[clip] ( 58.90, 68.86) rectangle (340.43,320.51);
\definecolor[named]{drawColor}{rgb}{0.83,0.83,0.83}

\path[draw=drawColor,line width= 0.4pt,line join=round,line cap=round] ( 86.71, 68.86) -- ( 86.71,320.51);

\path[draw=drawColor,line width= 0.4pt,line join=round,line cap=round] (124.36, 68.86) -- (124.36,320.51);

\path[draw=drawColor,line width= 0.4pt,line join=round,line cap=round] (162.02, 68.86) -- (162.02,320.51);

\path[draw=drawColor,line width= 0.4pt,line join=round,line cap=round] (199.67, 68.86) -- (199.67,320.51);

\path[draw=drawColor,line width= 0.4pt,line join=round,line cap=round] (237.32, 68.86) -- (237.32,320.51);

\path[draw=drawColor,line width= 0.4pt,line join=round,line cap=round] (274.98, 68.86) -- (274.98,320.51);

\path[draw=drawColor,line width= 0.4pt,line join=round,line cap=round] (312.63, 68.86) -- (312.63,320.51);

\path[draw=drawColor,line width= 0.4pt,line join=round,line cap=round] ( 58.90, 97.46) -- (340.43, 97.46);

\path[draw=drawColor,line width= 0.4pt,line join=round,line cap=round] ( 58.90,132.49) -- (340.43,132.49);

\path[draw=drawColor,line width= 0.4pt,line join=round,line cap=round] ( 58.90,167.53) -- (340.43,167.53);

\path[draw=drawColor,line width= 0.4pt,line join=round,line cap=round] ( 58.90,202.57) -- (340.43,202.57);

\path[draw=drawColor,line width= 0.4pt,line join=round,line cap=round] ( 58.90,237.61) -- (340.43,237.61);

\path[draw=drawColor,line width= 0.4pt,line join=round,line cap=round] ( 58.90,272.65) -- (340.43,272.65);

\path[draw=drawColor,line width= 0.4pt,line join=round,line cap=round] ( 58.90,307.69) -- (340.43,307.69);
\end{scope}
\begin{scope}
\path[clip] (  0.00,  0.00) rectangle (361.35,361.35);
\definecolor[named]{drawColor}{rgb}{0.00,0.00,0.00}

\path[draw=drawColor,line width= 0.4pt,line join=round,line cap=round] ( 58.90, 68.86) --
	(340.43, 68.86) --
	(340.43,320.51) --
	( 58.90,320.51) --
	( 58.90, 68.86);
\end{scope}
\begin{scope}
\path[clip] ( 58.90, 68.86) rectangle (340.43,320.51);
\definecolor[named]{drawColor}{rgb}{0.00,0.00,0.00}

\path[draw=drawColor,line width= 0.4pt,line join=round,line cap=round] (214.15,107.97) circle (  1.55);

\path[draw=drawColor,line width= 0.4pt,line join=round,line cap=round] (170.70, 78.18) circle (  1.55);

\path[draw=drawColor,line width= 0.4pt,line join=round,line cap=round] (170.70,109.72) circle (  1.55);

\path[draw=drawColor,line width= 0.4pt,line join=round,line cap=round] (228.63,227.10) circle (  1.55);

\path[draw=drawColor,line width= 0.4pt,line join=round,line cap=round] (228.63,113.22) circle (  1.55);

\path[draw=drawColor,line width= 0.4pt,line join=round,line cap=round] (257.60,111.47) circle (  1.55);

\path[draw=drawColor,line width= 0.4pt,line join=round,line cap=round] (293.80,174.54) circle (  1.55);

\path[draw=drawColor,line width= 0.4pt,line join=round,line cap=round] (136.91,218.34) circle (  1.55);

\path[draw=drawColor,line width= 0.4pt,line join=round,line cap=round] (214.15,311.19) circle (  1.55);

\path[draw=drawColor,line width= 0.4pt,line join=round,line cap=round] ( 83.81,192.06) circle (  1.55);

\path[draw=drawColor,line width= 0.4pt,line join=round,line cap=round] (214.15,274.40) circle (  1.55);

\path[draw=drawColor,line width= 0.4pt,line join=round,line cap=round] (141.74,130.74) circle (  1.55);

\path[draw=drawColor,line width= 0.4pt,line join=round,line cap=round] (257.60,162.28) circle (  1.55);

\path[draw=drawColor,line width= 0.4pt,line join=round,line cap=round] (301.04,218.34) circle (  1.55);

\path[draw=drawColor,line width= 0.4pt,line join=round,line cap=round] (231.05,186.80) circle (  1.55);

\path[draw=drawColor,line width= 0.4pt,line join=round,line cap=round] (156.22, 86.94) circle (  1.55);

\path[draw=drawColor,line width= 0.4pt,line join=round,line cap=round] (214.15, 90.45) circle (  1.55);

\path[draw=drawColor,line width= 0.4pt,line join=round,line cap=round] (185.19,188.56) circle (  1.55);

\path[draw=drawColor,line width= 0.4pt,line join=round,line cap=round] (243.11,100.96) circle (  1.55);

\path[draw=drawColor,line width= 0.4pt,line join=round,line cap=round] (127.26,195.56) circle (  1.55);

\path[draw=drawColor,line width= 0.4pt,line join=round,line cap=round] (330.01,188.56) circle (  1.55);

\path[draw=drawColor,line width= 0.4pt,line join=round,line cap=round] (127.26,185.05) circle (  1.55);

\path[draw=drawColor,line width= 0.4pt,line join=round,line cap=round] (301.04,102.71) circle (  1.55);

\path[draw=drawColor,line width= 0.4pt,line join=round,line cap=round] ( 69.33,172.79) circle (  1.55);

\path[draw=drawColor,line width= 0.4pt,line join=round,line cap=round] (243.11,193.81) circle (  1.55);

\path[draw=drawColor,line width= 0.4pt,line join=round,line cap=round] (301.04,190.31) circle (  1.55);

\path[draw=drawColor,line width= 0.4pt,line join=round,line cap=round] (243.11,178.05) circle (  1.55);

\path[draw=drawColor,line width= 0.4pt,line join=round,line cap=round] (330.01,128.99) circle (  1.55);

\path[draw=drawColor,line width= 0.4pt,line join=round,line cap=round] (285.24,241.12) circle (  1.55);

\path[draw=drawColor,line width= 0.4pt,line join=round,line cap=round] (156.22,137.75) circle (  1.55);
\definecolor[named]{drawColor}{rgb}{0.00,0.80,0.00}

\path[draw=drawColor,line width= 0.4pt,line join=round,line cap=round] ( 69.33,162.09) --
	(330.01,170.14);
\end{scope}
\end{tikzpicture}
\begin{tikzpicture}[x=1pt,y=1pt]
\definecolor[named]{fillColor}{rgb}{1.00,1.00,1.00}
\path[use as bounding box,fill=fillColor,fill opacity=0.00] (0,0) rectangle (361.35,361.35);
\begin{scope}
\path[clip] (  0.00, 68.86) rectangle ( 18.07,320.51);
\definecolor[named]{drawColor}{rgb}{0.00,0.00,0.00}

\path[draw=drawColor,line width= 0.4pt,line join=round,line cap=round] (  0.67,149.56) --
	( 17.40,149.56) --
	( 17.40,260.81) --
	(  0.67,260.81) --
	(  0.67,149.56);

\path[draw=drawColor,line width= 0.4pt,line join=round,line cap=round] (  0.67,182.09) --
	( 17.40,182.09);

\path[draw=drawColor,line width= 0.4pt,line join=round,line cap=round] (  9.03, 78.18) --
	(  9.03,149.56);

\path[draw=drawColor,line width= 0.4pt,line join=round,line cap=round] (  9.03,260.81) --
	(  9.03,311.19);
\end{scope}
\begin{scope}
\path[clip] ( 58.90,  0.00) rectangle (340.43, 18.07);
\definecolor[named]{drawColor}{rgb}{0.00,0.00,0.00}

\path[draw=drawColor,line width= 0.4pt,line join=round,line cap=round] (156.22,  0.67) --
	(156.22, 17.40) --
	(257.60, 17.40) --
	(257.60,  0.67) --
	(156.22,  0.67);

\path[draw=drawColor,line width= 0.4pt,line join=round,line cap=round] (221.39,  0.67) --
	(221.39, 17.40);

\path[draw=drawColor,line width= 0.4pt,line join=round,line cap=round] ( 69.33,  9.03) --
	(156.22,  9.03);

\path[draw=drawColor,line width= 0.4pt,line join=round,line cap=round] (257.60,  9.03) --
	(330.01,  9.03);
\end{scope}
\begin{scope}
\path[clip] (  0.00,  0.00) rectangle (361.35,361.35);
\definecolor[named]{drawColor}{rgb}{0.00,0.00,0.00}

\path[draw=drawColor,line width= 0.4pt,line join=round,line cap=round] ( 86.71, 68.86) -- (312.63, 68.86);

\path[draw=drawColor,line width= 0.4pt,line join=round,line cap=round] ( 86.71, 68.86) -- ( 86.71, 63.88);

\path[draw=drawColor,line width= 0.4pt,line join=round,line cap=round] (124.36, 68.86) -- (124.36, 63.88);

\path[draw=drawColor,line width= 0.4pt,line join=round,line cap=round] (162.02, 68.86) -- (162.02, 63.88);

\path[draw=drawColor,line width= 0.4pt,line join=round,line cap=round] (199.67, 68.86) -- (199.67, 63.88);

\path[draw=drawColor,line width= 0.4pt,line join=round,line cap=round] (237.32, 68.86) -- (237.32, 63.88);

\path[draw=drawColor,line width= 0.4pt,line join=round,line cap=round] (274.98, 68.86) -- (274.98, 63.88);

\path[draw=drawColor,line width= 0.4pt,line join=round,line cap=round] (312.63, 68.86) -- (312.63, 63.88);

\node[text=drawColor,anchor=base,inner sep=0pt, outer sep=0pt, scale=  0.83] at ( 86.71, 50.94) {-0.6};

\node[text=drawColor,anchor=base,inner sep=0pt, outer sep=0pt, scale=  0.83] at (124.36, 50.94) {-0.4};

\node[text=drawColor,anchor=base,inner sep=0pt, outer sep=0pt, scale=  0.83] at (162.02, 50.94) {-0.2};

\node[text=drawColor,anchor=base,inner sep=0pt, outer sep=0pt, scale=  0.83] at (199.67, 50.94) {0.0};

\node[text=drawColor,anchor=base,inner sep=0pt, outer sep=0pt, scale=  0.83] at (237.32, 50.94) {0.2};

\node[text=drawColor,anchor=base,inner sep=0pt, outer sep=0pt, scale=  0.83] at (274.98, 50.94) {0.4};

\node[text=drawColor,anchor=base,inner sep=0pt, outer sep=0pt, scale=  0.83] at (312.63, 50.94) {0.6};

\path[draw=drawColor,line width= 0.4pt,line join=round,line cap=round] ( 58.90, 90.78) -- ( 58.90,300.70);

\path[draw=drawColor,line width= 0.4pt,line join=round,line cap=round] ( 58.90, 90.78) -- ( 53.92, 90.78);

\path[draw=drawColor,line width= 0.4pt,line join=round,line cap=round] ( 58.90,132.76) -- ( 53.92,132.76);

\path[draw=drawColor,line width= 0.4pt,line join=round,line cap=round] ( 58.90,174.75) -- ( 53.92,174.75);

\path[draw=drawColor,line width= 0.4pt,line join=round,line cap=round] ( 58.90,216.73) -- ( 53.92,216.73);

\path[draw=drawColor,line width= 0.4pt,line join=round,line cap=round] ( 58.90,258.71) -- ( 53.92,258.71);

\path[draw=drawColor,line width= 0.4pt,line join=round,line cap=round] ( 58.90,300.70) -- ( 53.92,300.70);

\node[text=drawColor,rotate= 90.00,anchor=base,inner sep=0pt, outer sep=0pt, scale=  0.83] at ( 46.95, 90.78) {20};

\node[text=drawColor,rotate= 90.00,anchor=base,inner sep=0pt, outer sep=0pt, scale=  0.83] at ( 46.95,132.76) {40};

\node[text=drawColor,rotate= 90.00,anchor=base,inner sep=0pt, outer sep=0pt, scale=  0.83] at ( 46.95,174.75) {60};

\node[text=drawColor,rotate= 90.00,anchor=base,inner sep=0pt, outer sep=0pt, scale=  0.83] at ( 46.95,216.73) {80};

\node[text=drawColor,rotate= 90.00,anchor=base,inner sep=0pt, outer sep=0pt, scale=  0.83] at ( 46.95,258.71) {100};

\node[text=drawColor,rotate= 90.00,anchor=base,inner sep=0pt, outer sep=0pt, scale=  0.83] at ( 46.95,300.70) {120};

\path[draw=drawColor,line width= 0.4pt,line join=round,line cap=round] ( 58.90, 68.86) --
	(340.43, 68.86) --
	(340.43,320.51) --
	( 58.90,320.51) --
	( 58.90, 68.86);
\end{scope}
\begin{scope}
\path[clip] ( 18.07, 18.07) rectangle (361.35,361.35);
\definecolor[named]{drawColor}{rgb}{0.00,0.00,0.00}

\node[text=drawColor,anchor=base,inner sep=0pt, outer sep=0pt, scale=  0.83] at (199.67, 31.02) {Gender-Faktor};

\node[text=drawColor,rotate= 90.00,anchor=base,inner sep=0pt, outer sep=0pt, scale=  0.83] at ( 27.03,194.69) {Häufigkeit Buben};
\end{scope}
\begin{scope}
\path[clip] ( 58.90, 68.86) rectangle (340.43,320.51);
\definecolor[named]{drawColor}{rgb}{0.83,0.83,0.83}

\path[draw=drawColor,line width= 0.4pt,line join=round,line cap=round] ( 86.71, 68.86) -- ( 86.71,320.51);

\path[draw=drawColor,line width= 0.4pt,line join=round,line cap=round] (124.36, 68.86) -- (124.36,320.51);

\path[draw=drawColor,line width= 0.4pt,line join=round,line cap=round] (162.02, 68.86) -- (162.02,320.51);

\path[draw=drawColor,line width= 0.4pt,line join=round,line cap=round] (199.67, 68.86) -- (199.67,320.51);

\path[draw=drawColor,line width= 0.4pt,line join=round,line cap=round] (237.32, 68.86) -- (237.32,320.51);

\path[draw=drawColor,line width= 0.4pt,line join=round,line cap=round] (274.98, 68.86) -- (274.98,320.51);

\path[draw=drawColor,line width= 0.4pt,line join=round,line cap=round] (312.63, 68.86) -- (312.63,320.51);

\path[draw=drawColor,line width= 0.4pt,line join=round,line cap=round] ( 58.90, 90.78) -- (340.43, 90.78);

\path[draw=drawColor,line width= 0.4pt,line join=round,line cap=round] ( 58.90,132.76) -- (340.43,132.76);

\path[draw=drawColor,line width= 0.4pt,line join=round,line cap=round] ( 58.90,174.75) -- (340.43,174.75);

\path[draw=drawColor,line width= 0.4pt,line join=round,line cap=round] ( 58.90,216.73) -- (340.43,216.73);

\path[draw=drawColor,line width= 0.4pt,line join=round,line cap=round] ( 58.90,258.71) -- (340.43,258.71);

\path[draw=drawColor,line width= 0.4pt,line join=round,line cap=round] ( 58.90,300.70) -- (340.43,300.70);
\end{scope}
\begin{scope}
\path[clip] (  0.00,  0.00) rectangle (361.35,361.35);
\definecolor[named]{drawColor}{rgb}{0.00,0.00,0.00}

\path[draw=drawColor,line width= 0.4pt,line join=round,line cap=round] ( 58.90, 68.86) --
	(340.43, 68.86) --
	(340.43,320.51) --
	( 58.90,320.51) --
	( 58.90, 68.86);
\end{scope}
\begin{scope}
\path[clip] ( 58.90, 68.86) rectangle (340.43,320.51);
\definecolor[named]{drawColor}{rgb}{0.00,0.00,0.00}

\path[draw=drawColor,line width= 0.4pt,line join=round,line cap=round] (214.15,157.95) circle (  1.55);

\path[draw=drawColor,line width= 0.4pt,line join=round,line cap=round] (170.70, 94.98) circle (  1.55);

\path[draw=drawColor,line width= 0.4pt,line join=round,line cap=round] (170.70,149.56) circle (  1.55);

\path[draw=drawColor,line width= 0.4pt,line join=round,line cap=round] (228.63,296.50) circle (  1.55);

\path[draw=drawColor,line width= 0.4pt,line join=round,line cap=round] (228.63,193.64) circle (  1.55);

\path[draw=drawColor,line width= 0.4pt,line join=round,line cap=round] (257.60,189.44) circle (  1.55);

\path[draw=drawColor,line width= 0.4pt,line join=round,line cap=round] (293.80,269.21) circle (  1.55);

\path[draw=drawColor,line width= 0.4pt,line join=round,line cap=round] (136.91, 78.18) circle (  1.55);

\path[draw=drawColor,line width= 0.4pt,line join=round,line cap=round] (214.15,160.05) circle (  1.55);

\path[draw=drawColor,line width= 0.4pt,line join=round,line cap=round] ( 83.81, 94.98) circle (  1.55);

\path[draw=drawColor,line width= 0.4pt,line join=round,line cap=round] (214.15,206.23) circle (  1.55);

\path[draw=drawColor,line width= 0.4pt,line join=round,line cap=round] (141.74, 82.38) circle (  1.55);

\path[draw=drawColor,line width= 0.4pt,line join=round,line cap=round] (257.60,101.27) circle (  1.55);

\path[draw=drawColor,line width= 0.4pt,line join=round,line cap=round] (301.04,157.95) circle (  1.55);

\path[draw=drawColor,line width= 0.4pt,line join=round,line cap=round] (231.05,166.35) circle (  1.55);

\path[draw=drawColor,line width= 0.4pt,line join=round,line cap=round] (156.22, 97.08) circle (  1.55);

\path[draw=drawColor,line width= 0.4pt,line join=round,line cap=round] (214.15,113.87) circle (  1.55);

\path[draw=drawColor,line width= 0.4pt,line join=round,line cap=round] (185.19,260.81) circle (  1.55);

\path[draw=drawColor,line width= 0.4pt,line join=round,line cap=round] (243.11,157.95) circle (  1.55);

\path[draw=drawColor,line width= 0.4pt,line join=round,line cap=round] (127.26,191.54) circle (  1.55);

\path[draw=drawColor,line width= 0.4pt,line join=round,line cap=round] (330.01,286.00) circle (  1.55);

\path[draw=drawColor,line width= 0.4pt,line join=round,line cap=round] (127.26,202.04) circle (  1.55);

\path[draw=drawColor,line width= 0.4pt,line join=round,line cap=round] (301.04,279.71) circle (  1.55);

\path[draw=drawColor,line width= 0.4pt,line join=round,line cap=round] ( 69.33,174.75) circle (  1.55);

\path[draw=drawColor,line width= 0.4pt,line join=round,line cap=round] (243.11,311.19) circle (  1.55);

\path[draw=drawColor,line width= 0.4pt,line join=round,line cap=round] (301.04,304.90) circle (  1.55);

\path[draw=drawColor,line width= 0.4pt,line join=round,line cap=round] (243.11,294.40) circle (  1.55);

\path[draw=drawColor,line width= 0.4pt,line join=round,line cap=round] (330.01,162.15) circle (  1.55);

\path[draw=drawColor,line width= 0.4pt,line join=round,line cap=round] (285.24,248.22) circle (  1.55);

\path[draw=drawColor,line width= 0.4pt,line join=round,line cap=round] (156.22,189.44) circle (  1.55);
\definecolor[named]{drawColor}{rgb}{0.00,0.80,0.00}

\path[draw=drawColor,line width= 0.4pt,line join=round,line cap=round] ( 69.33,112.42) --
	(330.01,249.40);
\end{scope}
\end{tikzpicture}
\begin{tikzpicture}[x=1pt,y=1pt]
\definecolor[named]{fillColor}{rgb}{1.00,1.00,1.00}
\path[use as bounding box,fill=fillColor,fill opacity=0.00] (0,0) rectangle (361.35,361.35);
\begin{scope}
\path[clip] (  0.00, 68.86) rectangle ( 18.07,320.51);
\definecolor[named]{drawColor}{rgb}{0.00,0.00,0.00}

\path[draw=drawColor,line width= 0.4pt,line join=round,line cap=round] (  0.67,111.47) --
	( 17.40,111.47) --
	( 17.40,193.81) --
	(  0.67,193.81) --
	(  0.67,111.47);

\path[draw=drawColor,line width= 0.4pt,line join=round,line cap=round] (  0.67,176.29) --
	( 17.40,176.29);

\path[draw=drawColor,line width= 0.4pt,line join=round,line cap=round] (  9.03, 78.18) --
	(  9.03,111.47);

\path[draw=drawColor,line width= 0.4pt,line join=round,line cap=round] (  9.03,193.81) --
	(  9.03,311.19);
\end{scope}
\begin{scope}
\path[clip] ( 58.90,  0.00) rectangle (340.43, 18.07);
\definecolor[named]{drawColor}{rgb}{0.00,0.00,0.00}

\path[draw=drawColor,line width= 0.4pt,line join=round,line cap=round] (156.22,  0.67) --
	(156.22, 17.40) --
	(257.60, 17.40) --
	(257.60,  0.67) --
	(156.22,  0.67);

\path[draw=drawColor,line width= 0.4pt,line join=round,line cap=round] (221.39,  0.67) --
	(221.39, 17.40);

\path[draw=drawColor,line width= 0.4pt,line join=round,line cap=round] ( 69.33,  9.03) --
	(156.22,  9.03);

\path[draw=drawColor,line width= 0.4pt,line join=round,line cap=round] (257.60,  9.03) --
	(330.01,  9.03);
\end{scope}
\begin{scope}
\path[clip] (  0.00,  0.00) rectangle (361.35,361.35);
\definecolor[named]{drawColor}{rgb}{0.00,0.00,0.00}

\path[draw=drawColor,line width= 0.4pt,line join=round,line cap=round] ( 86.71, 68.86) -- (312.63, 68.86);

\path[draw=drawColor,line width= 0.4pt,line join=round,line cap=round] ( 86.71, 68.86) -- ( 86.71, 63.88);

\path[draw=drawColor,line width= 0.4pt,line join=round,line cap=round] (124.36, 68.86) -- (124.36, 63.88);

\path[draw=drawColor,line width= 0.4pt,line join=round,line cap=round] (162.02, 68.86) -- (162.02, 63.88);

\path[draw=drawColor,line width= 0.4pt,line join=round,line cap=round] (199.67, 68.86) -- (199.67, 63.88);

\path[draw=drawColor,line width= 0.4pt,line join=round,line cap=round] (237.32, 68.86) -- (237.32, 63.88);

\path[draw=drawColor,line width= 0.4pt,line join=round,line cap=round] (274.98, 68.86) -- (274.98, 63.88);

\path[draw=drawColor,line width= 0.4pt,line join=round,line cap=round] (312.63, 68.86) -- (312.63, 63.88);

\node[text=drawColor,anchor=base,inner sep=0pt, outer sep=0pt, scale=  0.83] at ( 86.71, 50.94) {-0.6};

\node[text=drawColor,anchor=base,inner sep=0pt, outer sep=0pt, scale=  0.83] at (124.36, 50.94) {-0.4};

\node[text=drawColor,anchor=base,inner sep=0pt, outer sep=0pt, scale=  0.83] at (162.02, 50.94) {-0.2};

\node[text=drawColor,anchor=base,inner sep=0pt, outer sep=0pt, scale=  0.83] at (199.67, 50.94) {0.0};

\node[text=drawColor,anchor=base,inner sep=0pt, outer sep=0pt, scale=  0.83] at (237.32, 50.94) {0.2};

\node[text=drawColor,anchor=base,inner sep=0pt, outer sep=0pt, scale=  0.83] at (274.98, 50.94) {0.4};

\node[text=drawColor,anchor=base,inner sep=0pt, outer sep=0pt, scale=  0.83] at (312.63, 50.94) {0.6};

\path[draw=drawColor,line width= 0.4pt,line join=round,line cap=round] ( 58.90, 97.46) -- ( 58.90,307.69);

\path[draw=drawColor,line width= 0.4pt,line join=round,line cap=round] ( 58.90, 97.46) -- ( 53.92, 97.46);

\path[draw=drawColor,line width= 0.4pt,line join=round,line cap=round] ( 58.90,132.49) -- ( 53.92,132.49);

\path[draw=drawColor,line width= 0.4pt,line join=round,line cap=round] ( 58.90,167.53) -- ( 53.92,167.53);

\path[draw=drawColor,line width= 0.4pt,line join=round,line cap=round] ( 58.90,202.57) -- ( 53.92,202.57);

\path[draw=drawColor,line width= 0.4pt,line join=round,line cap=round] ( 58.90,237.61) -- ( 53.92,237.61);

\path[draw=drawColor,line width= 0.4pt,line join=round,line cap=round] ( 58.90,272.65) -- ( 53.92,272.65);

\path[draw=drawColor,line width= 0.4pt,line join=round,line cap=round] ( 58.90,307.69) -- ( 53.92,307.69);

\node[text=drawColor,rotate= 90.00,anchor=base,inner sep=0pt, outer sep=0pt, scale=  0.83] at ( 46.95, 97.46) {40};

\node[text=drawColor,rotate= 90.00,anchor=base,inner sep=0pt, outer sep=0pt, scale=  0.83] at ( 46.95,132.49) {60};

\node[text=drawColor,rotate= 90.00,anchor=base,inner sep=0pt, outer sep=0pt, scale=  0.83] at ( 46.95,167.53) {80};

\node[text=drawColor,rotate= 90.00,anchor=base,inner sep=0pt, outer sep=0pt, scale=  0.83] at ( 46.95,202.57) {100};

\node[text=drawColor,rotate= 90.00,anchor=base,inner sep=0pt, outer sep=0pt, scale=  0.83] at ( 46.95,237.61) {120};

\node[text=drawColor,rotate= 90.00,anchor=base,inner sep=0pt, outer sep=0pt, scale=  0.83] at ( 46.95,272.65) {140};

\node[text=drawColor,rotate= 90.00,anchor=base,inner sep=0pt, outer sep=0pt, scale=  0.83] at ( 46.95,307.69) {160};

\path[draw=drawColor,line width= 0.4pt,line join=round,line cap=round] ( 58.90, 68.86) --
	(340.43, 68.86) --
	(340.43,320.51) --
	( 58.90,320.51) --
	( 58.90, 68.86);
\end{scope}
\begin{scope}
\path[clip] ( 18.07, 18.07) rectangle (361.35,361.35);
\definecolor[named]{drawColor}{rgb}{0.00,0.00,0.00}

\node[text=drawColor,anchor=base,inner sep=0pt, outer sep=0pt, scale=  0.83] at (199.67, 31.02) {Gender-Faktor};

\node[text=drawColor,rotate= 90.00,anchor=base,inner sep=0pt, outer sep=0pt, scale=  0.83] at ( 27.03,194.69) {Häufigkeit Mädchen};
\end{scope}
\begin{scope}
\path[clip] ( 58.90, 68.86) rectangle (340.43,320.51);
\definecolor[named]{drawColor}{rgb}{0.83,0.83,0.83}

\path[draw=drawColor,line width= 0.4pt,line join=round,line cap=round] ( 86.71, 68.86) -- ( 86.71,320.51);

\path[draw=drawColor,line width= 0.4pt,line join=round,line cap=round] (124.36, 68.86) -- (124.36,320.51);

\path[draw=drawColor,line width= 0.4pt,line join=round,line cap=round] (162.02, 68.86) -- (162.02,320.51);

\path[draw=drawColor,line width= 0.4pt,line join=round,line cap=round] (199.67, 68.86) -- (199.67,320.51);

\path[draw=drawColor,line width= 0.4pt,line join=round,line cap=round] (237.32, 68.86) -- (237.32,320.51);

\path[draw=drawColor,line width= 0.4pt,line join=round,line cap=round] (274.98, 68.86) -- (274.98,320.51);

\path[draw=drawColor,line width= 0.4pt,line join=round,line cap=round] (312.63, 68.86) -- (312.63,320.51);

\path[draw=drawColor,line width= 0.4pt,line join=round,line cap=round] ( 58.90, 97.46) -- (340.43, 97.46);

\path[draw=drawColor,line width= 0.4pt,line join=round,line cap=round] ( 58.90,132.49) -- (340.43,132.49);

\path[draw=drawColor,line width= 0.4pt,line join=round,line cap=round] ( 58.90,167.53) -- (340.43,167.53);

\path[draw=drawColor,line width= 0.4pt,line join=round,line cap=round] ( 58.90,202.57) -- (340.43,202.57);

\path[draw=drawColor,line width= 0.4pt,line join=round,line cap=round] ( 58.90,237.61) -- (340.43,237.61);

\path[draw=drawColor,line width= 0.4pt,line join=round,line cap=round] ( 58.90,272.65) -- (340.43,272.65);

\path[draw=drawColor,line width= 0.4pt,line join=round,line cap=round] ( 58.90,307.69) -- (340.43,307.69);
\end{scope}
\begin{scope}
\path[clip] (  0.00,  0.00) rectangle (361.35,361.35);
\definecolor[named]{drawColor}{rgb}{0.00,0.00,0.00}

\path[draw=drawColor,line width= 0.4pt,line join=round,line cap=round] ( 58.90, 68.86) --
	(340.43, 68.86) --
	(340.43,320.51) --
	( 58.90,320.51) --
	( 58.90, 68.86);
\end{scope}
\begin{scope}
\path[clip] ( 58.90, 68.86) rectangle (340.43,320.51);
\definecolor[named]{drawColor}{rgb}{0.00,0.00,0.00}

\path[draw=drawColor,line width= 0.4pt,line join=round,line cap=round] (214.15,107.97) circle (  1.55);

\path[draw=drawColor,line width= 0.4pt,line join=round,line cap=round] (170.70, 78.18) circle (  1.55);

\path[draw=drawColor,line width= 0.4pt,line join=round,line cap=round] (170.70,109.72) circle (  1.55);

\path[draw=drawColor,line width= 0.4pt,line join=round,line cap=round] (228.63,227.10) circle (  1.55);

\path[draw=drawColor,line width= 0.4pt,line join=round,line cap=round] (228.63,113.22) circle (  1.55);

\path[draw=drawColor,line width= 0.4pt,line join=round,line cap=round] (257.60,111.47) circle (  1.55);

\path[draw=drawColor,line width= 0.4pt,line join=round,line cap=round] (293.80,174.54) circle (  1.55);

\path[draw=drawColor,line width= 0.4pt,line join=round,line cap=round] (136.91,218.34) circle (  1.55);

\path[draw=drawColor,line width= 0.4pt,line join=round,line cap=round] (214.15,311.19) circle (  1.55);

\path[draw=drawColor,line width= 0.4pt,line join=round,line cap=round] ( 83.81,192.06) circle (  1.55);

\path[draw=drawColor,line width= 0.4pt,line join=round,line cap=round] (214.15,274.40) circle (  1.55);

\path[draw=drawColor,line width= 0.4pt,line join=round,line cap=round] (141.74,130.74) circle (  1.55);

\path[draw=drawColor,line width= 0.4pt,line join=round,line cap=round] (257.60,162.28) circle (  1.55);

\path[draw=drawColor,line width= 0.4pt,line join=round,line cap=round] (301.04,218.34) circle (  1.55);

\path[draw=drawColor,line width= 0.4pt,line join=round,line cap=round] (231.05,186.80) circle (  1.55);

\path[draw=drawColor,line width= 0.4pt,line join=round,line cap=round] (156.22, 86.94) circle (  1.55);

\path[draw=drawColor,line width= 0.4pt,line join=round,line cap=round] (214.15, 90.45) circle (  1.55);

\path[draw=drawColor,line width= 0.4pt,line join=round,line cap=round] (185.19,188.56) circle (  1.55);

\path[draw=drawColor,line width= 0.4pt,line join=round,line cap=round] (243.11,100.96) circle (  1.55);

\path[draw=drawColor,line width= 0.4pt,line join=round,line cap=round] (127.26,195.56) circle (  1.55);

\path[draw=drawColor,line width= 0.4pt,line join=round,line cap=round] (330.01,188.56) circle (  1.55);

\path[draw=drawColor,line width= 0.4pt,line join=round,line cap=round] (127.26,185.05) circle (  1.55);

\path[draw=drawColor,line width= 0.4pt,line join=round,line cap=round] (301.04,102.71) circle (  1.55);

\path[draw=drawColor,line width= 0.4pt,line join=round,line cap=round] ( 69.33,172.79) circle (  1.55);

\path[draw=drawColor,line width= 0.4pt,line join=round,line cap=round] (243.11,193.81) circle (  1.55);

\path[draw=drawColor,line width= 0.4pt,line join=round,line cap=round] (301.04,190.31) circle (  1.55);

\path[draw=drawColor,line width= 0.4pt,line join=round,line cap=round] (243.11,178.05) circle (  1.55);

\path[draw=drawColor,line width= 0.4pt,line join=round,line cap=round] (330.01,128.99) circle (  1.55);

\path[draw=drawColor,line width= 0.4pt,line join=round,line cap=round] (285.24,241.12) circle (  1.55);

\path[draw=drawColor,line width= 0.4pt,line join=round,line cap=round] (156.22,137.75) circle (  1.55);
\definecolor[named]{drawColor}{rgb}{0.00,0.80,0.00}

\path[draw=drawColor,line width= 0.4pt,line join=round,line cap=round] ( 69.33,162.09) --
	(330.01,170.14);
\end{scope}
\end{tikzpicture}
\begin{tikzpicture}[x=1pt,y=1pt]
\definecolor[named]{fillColor}{rgb}{1.00,1.00,1.00}
\path[use as bounding box,fill=fillColor,fill opacity=0.00] (0,0) rectangle (361.35,361.35);
\begin{scope}
\path[clip] (  0.00, 68.86) rectangle ( 18.07,320.51);
\definecolor[named]{drawColor}{rgb}{0.00,0.00,0.00}

\path[draw=drawColor,line width= 0.4pt,line join=round,line cap=round] (  0.67,111.47) --
	( 17.40,111.47) --
	( 17.40,193.81) --
	(  0.67,193.81) --
	(  0.67,111.47);

\path[draw=drawColor,line width= 0.4pt,line join=round,line cap=round] (  0.67,176.29) --
	( 17.40,176.29);

\path[draw=drawColor,line width= 0.4pt,line join=round,line cap=round] (  9.03, 78.18) --
	(  9.03,111.47);

\path[draw=drawColor,line width= 0.4pt,line join=round,line cap=round] (  9.03,193.81) --
	(  9.03,311.19);
\end{scope}
\begin{scope}
\path[clip] ( 58.90,  0.00) rectangle (340.43, 18.07);
\definecolor[named]{drawColor}{rgb}{0.00,0.00,0.00}

\path[draw=drawColor,line width= 0.4pt,line join=round,line cap=round] (156.22,  0.67) --
	(156.22, 17.40) --
	(257.60, 17.40) --
	(257.60,  0.67) --
	(156.22,  0.67);

\path[draw=drawColor,line width= 0.4pt,line join=round,line cap=round] (221.39,  0.67) --
	(221.39, 17.40);

\path[draw=drawColor,line width= 0.4pt,line join=round,line cap=round] ( 69.33,  9.03) --
	(156.22,  9.03);

\path[draw=drawColor,line width= 0.4pt,line join=round,line cap=round] (257.60,  9.03) --
	(330.01,  9.03);
\end{scope}
\begin{scope}
\path[clip] (  0.00,  0.00) rectangle (361.35,361.35);
\definecolor[named]{drawColor}{rgb}{0.00,0.00,0.00}

\path[draw=drawColor,line width= 0.4pt,line join=round,line cap=round] ( 86.71, 68.86) -- (312.63, 68.86);

\path[draw=drawColor,line width= 0.4pt,line join=round,line cap=round] ( 86.71, 68.86) -- ( 86.71, 63.88);

\path[draw=drawColor,line width= 0.4pt,line join=round,line cap=round] (124.36, 68.86) -- (124.36, 63.88);

\path[draw=drawColor,line width= 0.4pt,line join=round,line cap=round] (162.02, 68.86) -- (162.02, 63.88);

\path[draw=drawColor,line width= 0.4pt,line join=round,line cap=round] (199.67, 68.86) -- (199.67, 63.88);

\path[draw=drawColor,line width= 0.4pt,line join=round,line cap=round] (237.32, 68.86) -- (237.32, 63.88);

\path[draw=drawColor,line width= 0.4pt,line join=round,line cap=round] (274.98, 68.86) -- (274.98, 63.88);

\path[draw=drawColor,line width= 0.4pt,line join=round,line cap=round] (312.63, 68.86) -- (312.63, 63.88);

\node[text=drawColor,anchor=base,inner sep=0pt, outer sep=0pt, scale=  0.83] at ( 86.71, 50.94) {-0.6};

\node[text=drawColor,anchor=base,inner sep=0pt, outer sep=0pt, scale=  0.83] at (124.36, 50.94) {-0.4};

\node[text=drawColor,anchor=base,inner sep=0pt, outer sep=0pt, scale=  0.83] at (162.02, 50.94) {-0.2};

\node[text=drawColor,anchor=base,inner sep=0pt, outer sep=0pt, scale=  0.83] at (199.67, 50.94) {0.0};

\node[text=drawColor,anchor=base,inner sep=0pt, outer sep=0pt, scale=  0.83] at (237.32, 50.94) {0.2};

\node[text=drawColor,anchor=base,inner sep=0pt, outer sep=0pt, scale=  0.83] at (274.98, 50.94) {0.4};

\node[text=drawColor,anchor=base,inner sep=0pt, outer sep=0pt, scale=  0.83] at (312.63, 50.94) {0.6};

\path[draw=drawColor,line width= 0.4pt,line join=round,line cap=round] ( 58.90, 97.46) -- ( 58.90,307.69);

\path[draw=drawColor,line width= 0.4pt,line join=round,line cap=round] ( 58.90, 97.46) -- ( 53.92, 97.46);

\path[draw=drawColor,line width= 0.4pt,line join=round,line cap=round] ( 58.90,132.49) -- ( 53.92,132.49);

\path[draw=drawColor,line width= 0.4pt,line join=round,line cap=round] ( 58.90,167.53) -- ( 53.92,167.53);

\path[draw=drawColor,line width= 0.4pt,line join=round,line cap=round] ( 58.90,202.57) -- ( 53.92,202.57);

\path[draw=drawColor,line width= 0.4pt,line join=round,line cap=round] ( 58.90,237.61) -- ( 53.92,237.61);

\path[draw=drawColor,line width= 0.4pt,line join=round,line cap=round] ( 58.90,272.65) -- ( 53.92,272.65);

\path[draw=drawColor,line width= 0.4pt,line join=round,line cap=round] ( 58.90,307.69) -- ( 53.92,307.69);

\node[text=drawColor,rotate= 90.00,anchor=base,inner sep=0pt, outer sep=0pt, scale=  0.83] at ( 46.95, 97.46) {40};

\node[text=drawColor,rotate= 90.00,anchor=base,inner sep=0pt, outer sep=0pt, scale=  0.83] at ( 46.95,132.49) {60};

\node[text=drawColor,rotate= 90.00,anchor=base,inner sep=0pt, outer sep=0pt, scale=  0.83] at ( 46.95,167.53) {80};

\node[text=drawColor,rotate= 90.00,anchor=base,inner sep=0pt, outer sep=0pt, scale=  0.83] at ( 46.95,202.57) {100};

\node[text=drawColor,rotate= 90.00,anchor=base,inner sep=0pt, outer sep=0pt, scale=  0.83] at ( 46.95,237.61) {120};

\node[text=drawColor,rotate= 90.00,anchor=base,inner sep=0pt, outer sep=0pt, scale=  0.83] at ( 46.95,272.65) {140};

\node[text=drawColor,rotate= 90.00,anchor=base,inner sep=0pt, outer sep=0pt, scale=  0.83] at ( 46.95,307.69) {160};

\path[draw=drawColor,line width= 0.4pt,line join=round,line cap=round] ( 58.90, 68.86) --
	(340.43, 68.86) --
	(340.43,320.51) --
	( 58.90,320.51) --
	( 58.90, 68.86);
\end{scope}
\begin{scope}
\path[clip] ( 18.07, 18.07) rectangle (361.35,361.35);
\definecolor[named]{drawColor}{rgb}{0.00,0.00,0.00}

\node[text=drawColor,anchor=base,inner sep=0pt, outer sep=0pt, scale=  0.83] at (199.67, 31.02) {Gender-Faktor};

\node[text=drawColor,rotate= 90.00,anchor=base,inner sep=0pt, outer sep=0pt, scale=  0.83] at ( 27.03,194.69) {Häufigkeit Mädchen};
\end{scope}
\begin{scope}
\path[clip] ( 58.90, 68.86) rectangle (340.43,320.51);
\definecolor[named]{drawColor}{rgb}{0.83,0.83,0.83}

\path[draw=drawColor,line width= 0.4pt,line join=round,line cap=round] ( 86.71, 68.86) -- ( 86.71,320.51);

\path[draw=drawColor,line width= 0.4pt,line join=round,line cap=round] (124.36, 68.86) -- (124.36,320.51);

\path[draw=drawColor,line width= 0.4pt,line join=round,line cap=round] (162.02, 68.86) -- (162.02,320.51);

\path[draw=drawColor,line width= 0.4pt,line join=round,line cap=round] (199.67, 68.86) -- (199.67,320.51);

\path[draw=drawColor,line width= 0.4pt,line join=round,line cap=round] (237.32, 68.86) -- (237.32,320.51);

\path[draw=drawColor,line width= 0.4pt,line join=round,line cap=round] (274.98, 68.86) -- (274.98,320.51);

\path[draw=drawColor,line width= 0.4pt,line join=round,line cap=round] (312.63, 68.86) -- (312.63,320.51);

\path[draw=drawColor,line width= 0.4pt,line join=round,line cap=round] ( 58.90, 97.46) -- (340.43, 97.46);

\path[draw=drawColor,line width= 0.4pt,line join=round,line cap=round] ( 58.90,132.49) -- (340.43,132.49);

\path[draw=drawColor,line width= 0.4pt,line join=round,line cap=round] ( 58.90,167.53) -- (340.43,167.53);

\path[draw=drawColor,line width= 0.4pt,line join=round,line cap=round] ( 58.90,202.57) -- (340.43,202.57);

\path[draw=drawColor,line width= 0.4pt,line join=round,line cap=round] ( 58.90,237.61) -- (340.43,237.61);

\path[draw=drawColor,line width= 0.4pt,line join=round,line cap=round] ( 58.90,272.65) -- (340.43,272.65);

\path[draw=drawColor,line width= 0.4pt,line join=round,line cap=round] ( 58.90,307.69) -- (340.43,307.69);
\end{scope}
\begin{scope}
\path[clip] (  0.00,  0.00) rectangle (361.35,361.35);
\definecolor[named]{drawColor}{rgb}{0.00,0.00,0.00}

\path[draw=drawColor,line width= 0.4pt,line join=round,line cap=round] ( 58.90, 68.86) --
	(340.43, 68.86) --
	(340.43,320.51) --
	( 58.90,320.51) --
	( 58.90, 68.86);
\end{scope}
\begin{scope}
\path[clip] ( 58.90, 68.86) rectangle (340.43,320.51);
\definecolor[named]{drawColor}{rgb}{0.00,0.00,0.00}

\path[draw=drawColor,line width= 0.4pt,line join=round,line cap=round] (214.15,107.97) circle (  1.55);

\path[draw=drawColor,line width= 0.4pt,line join=round,line cap=round] (170.70, 78.18) circle (  1.55);

\path[draw=drawColor,line width= 0.4pt,line join=round,line cap=round] (170.70,109.72) circle (  1.55);

\path[draw=drawColor,line width= 0.4pt,line join=round,line cap=round] (228.63,227.10) circle (  1.55);

\path[draw=drawColor,line width= 0.4pt,line join=round,line cap=round] (228.63,113.22) circle (  1.55);

\path[draw=drawColor,line width= 0.4pt,line join=round,line cap=round] (257.60,111.47) circle (  1.55);

\path[draw=drawColor,line width= 0.4pt,line join=round,line cap=round] (293.80,174.54) circle (  1.55);

\path[draw=drawColor,line width= 0.4pt,line join=round,line cap=round] (136.91,218.34) circle (  1.55);

\path[draw=drawColor,line width= 0.4pt,line join=round,line cap=round] (214.15,311.19) circle (  1.55);

\path[draw=drawColor,line width= 0.4pt,line join=round,line cap=round] ( 83.81,192.06) circle (  1.55);

\path[draw=drawColor,line width= 0.4pt,line join=round,line cap=round] (214.15,274.40) circle (  1.55);

\path[draw=drawColor,line width= 0.4pt,line join=round,line cap=round] (141.74,130.74) circle (  1.55);

\path[draw=drawColor,line width= 0.4pt,line join=round,line cap=round] (257.60,162.28) circle (  1.55);

\path[draw=drawColor,line width= 0.4pt,line join=round,line cap=round] (301.04,218.34) circle (  1.55);

\path[draw=drawColor,line width= 0.4pt,line join=round,line cap=round] (231.05,186.80) circle (  1.55);

\path[draw=drawColor,line width= 0.4pt,line join=round,line cap=round] (156.22, 86.94) circle (  1.55);

\path[draw=drawColor,line width= 0.4pt,line join=round,line cap=round] (214.15, 90.45) circle (  1.55);

\path[draw=drawColor,line width= 0.4pt,line join=round,line cap=round] (185.19,188.56) circle (  1.55);

\path[draw=drawColor,line width= 0.4pt,line join=round,line cap=round] (243.11,100.96) circle (  1.55);

\path[draw=drawColor,line width= 0.4pt,line join=round,line cap=round] (127.26,195.56) circle (  1.55);

\path[draw=drawColor,line width= 0.4pt,line join=round,line cap=round] (330.01,188.56) circle (  1.55);

\path[draw=drawColor,line width= 0.4pt,line join=round,line cap=round] (127.26,185.05) circle (  1.55);

\path[draw=drawColor,line width= 0.4pt,line join=round,line cap=round] (301.04,102.71) circle (  1.55);

\path[draw=drawColor,line width= 0.4pt,line join=round,line cap=round] ( 69.33,172.79) circle (  1.55);

\path[draw=drawColor,line width= 0.4pt,line join=round,line cap=round] (243.11,193.81) circle (  1.55);

\path[draw=drawColor,line width= 0.4pt,line join=round,line cap=round] (301.04,190.31) circle (  1.55);

\path[draw=drawColor,line width= 0.4pt,line join=round,line cap=round] (243.11,178.05) circle (  1.55);

\path[draw=drawColor,line width= 0.4pt,line join=round,line cap=round] (330.01,128.99) circle (  1.55);

\path[draw=drawColor,line width= 0.4pt,line join=round,line cap=round] (285.24,241.12) circle (  1.55);

\path[draw=drawColor,line width= 0.4pt,line join=round,line cap=round] (156.22,137.75) circle (  1.55);
\definecolor[named]{drawColor}{rgb}{0.00,0.80,0.00}

\path[draw=drawColor,line width= 0.4pt,line join=round,line cap=round] ( 69.33,162.09) --
	(330.01,170.14);
\end{scope}
\end{tikzpicture}


\end{figure}

\section{Mädchen bevorzugen Bücher mit wenig Figuren am Cover}

Somit bleibt von den bis jetzt angesprochen Merkmalen nur mehr die
Anzahl der Figuren am Cover. Zu unserer Überraschung besteht ein
negativer linearer Zusammenhang zwischen der Häufigkeit der Leserinnen
und der Anzahl der Figuren am Cover. Das heißt, umso weniger Figuren am
Cover sind umso höher ist die Wahrscheinlichkeit, dass das Buch von
einem Mädchen gelesen wurde. In unseren ersten Überlegungen hatten wir
eher damit gerechnet, dass Mädchen mehrere Figuren bevorzugen würden.

Um zu verstehen, wie es zu diesem Merkmal kommt, ist es wieder sinnvoll
die Entstehung dieses Merkmals genauer zu beleuchten. Dieses Merkmal
entsteht, wie auch schon die Helligkeit, ohne den direkten Einfluss der
Verfasserin bzw. des Verfassers. Die Grafikabteilung des Verlags,
übersetzt hier wieder Inhalt in Design. Wobei wir vermuten, dass zwei
Aspekte der Geschichte für die Anzahl der Figuren wichtig ist.
Einerseits halten wir es für entscheidend, ob es sich um einen
Multiprotagonisten handelt, wie z.B bei der \emph{Knickerbockerbande}
oder den \emph{Wilden Hühnern}. Andererseits glauben wir, dass die Ebene
auf der die Geschichte stattfindet, ob es viel \emph{psychologisches}
also z.B. \emph{Inneren Monolog} gibt, oder ob sich die meisten Probleme
auf soziales Handeln beziehen. Diese These wird auch davon gestützt,
dass die stärkste Korrelation der Anzahl der Figuren von dem Merkmal
\emph{Innerer Monolog} ausgeht ($R=0{,}36; p=0{,}06$).

\chapter{Fazit}

Unter Gender verstehen \inparencite[126]{West1987} vom Geschlecht
abhängiges Verhalten. Wir haben in unserer Arbeit gezeigt, dass es
Gender auch bei Kinderbüchern gibt. Auch Kinderbücher \emph{verhalten}
sich abhängig vom Geschlecht und zwar abhängig vom Geschlecht der
Lesenden.Mädchen und Buben lesen unterschiedliche Bücher und diese
Bücher unterscheiden sich in ihrem Verhalten. So konsumieren Mädchen und
Buben unterschiedliches Verhalten. Eine Analyse dieses Verhaltens anhand
der Hauptfiguren hat gezeigt, dass das Gender der Bücher auch mit den
Geschlechterstereotypen zusammenhängt. Umso höher der Anteil an
Leserinnen um so femininer handelt die Hauptfigur.

Das es dieses Gender geben kann, setzt voraus, dass dieser Unterschied
auch bei der Entscheidung ob man ein Buch liest gemacht werden kann.
Dass dies möglich ist, wurde durch den zweiten Teil der Untersuchung
gezeigt.

Geht man davon aus, dass konsumiertes Verhalten auf die Leserinnen und
Leser abfärbt und deren Verhalten beeinflusst, könnte man vermuten, dass
Kinderbücher auf diese Art und Weise geschlechtsstereotypes Verhalten
bei Kindern verstärkt wird.

Durch das Zeigen, wie Stereotypen mit Kindern verknüpft werden, lassen
sich auch neue Ansätze für Gendermainstreaming-Maßnahmen in Bezug auf
Kinderbücher ableiten.

Wir möchten noch einmal darauf hinweisen, dass wir nicht die Wirkung von
Büchern auf Kinder untersucht haben sondern nur das geschlechtsabhängige
\emph{Verhalten} von den Büchern selbst: das Gender von Kinderbüchern.



% \nocite{Knister1994}
\pagebreak
\singlespacing
\onecolumn
% \printbibliography%[heading=subbibliography]

\nocite{BrezinaK}


\printbibliography[notkeyword={kb},heading=bibliography, title={Literaturverzeichnis}]


  
 \printbibliography[keyword={kb}, heading=subbibliography, title={Analysierte Kinderbücher}]


\appendix
\chapter{Anhang}






% latex table generated in R 2.15.2 by xtable 1.7-0 package
% Fri Jan 25 13:16:08 2013
\begin{sidewaystable}[ht]
\begin{center}
\scalebox{0.6}{
\begin{tabular}{rllllrrlrrrrr}
  \hline
 & buchtite & buchtyp & autorin & autsex & maedchen & buben & hfignam & hell & buchst & seiten & figanz & wm \\ 
  \hline
1 & Franz                                                                                                                                                                                                                                                           & Reihe & Christine Noestlinger                                                                                                                                                                                                                                           & weiblich & 83.00 & 60.00 & Franz                                                                                                                                                                                                                                                           & 168.90 &  & 55.00 & 2.00 & -0.16 \\ 
  2 & Conni                                                                                                                                                                                                                                                           & Reihe & Julia Boehmerz                                                                                                                                                                                                                                                  & weiblich & 94.00 & 22.00 & Conni                                                                                                                                                                                                                                                           & 186.59 &  & 115.00 & 3.00 & -0.62 \\ 
  3 & Pinocchio                                                                                                                                                                                                                                                       & Buch & Carlo Collodi                                                                                                                                                                                                                                                   & maennlich & 96.00 & 68.00 & Pinocchio                                                                                                                                                                                                                                                       & 99.24 & 21.00 & 288.00 & 4.00 & -0.17 \\ 
  4 & Peter Pan                                                                                                                                                                                                                                                       & Buch & James M. Barrie                                                                                                                                                                                                                                                 & maennlich & 90.00 & 73.00 & Wendy (Peter)                                                                                                                                                                                                                                                   & 144.30 & 14.00 & 72.00 & 2.00 & -0.10 \\ 
  5 & Prinzessing Lillifee                                                                                                                                                                                                                                            & Reihe & Monika Finsterbusch                                                                                                                                                                                                                                             & weiblich & 109.00 & 14.00 & Lillifee                                                                                                                                                                                                                                                        & 179.40 & 46.00 & 96.00 & 6.00 & -0.77 \\ 
  6 & Mini                                                                                                                                                                                                                                                            & Reihe & Christine Noestlinger                                                                                                                                                                                                                                           & weiblich & 59.00 & 16.00 & Mini                                                                                                                                                                                                                                                            & 150.23 &  & 64.00 & 4.00 & -0.57 \\ 
  7 & Das Wutmonster                                                                                                                                                                                                                                                  & Buch & Britta Schwarz                                                                                                                                                                                                                                                  & weiblich & 34.00 & 23.00 & Marvin                                                                                                                                                                                                                                                          & 182.93 &  & 32.00 & 3.00 & -0.19 \\ 
  8 & Sams                                                                                                                                                                                                                                                            & Kleinserie & Paul Maar                                                                                                                                                                                                                                                       & maennlich & 63.00 & 67.00 & Sams                                                                                                                                                                                                                                                            & 161.61 & 27.00 & 208.00 & 8.00 & 0.03 \\ 
  9 & Baumhausgeschichten                                                                                                                                                                                                                                             & Buch & Martin Kleinhichte                                                                                                                                                                                                                                              & maennlich & 29.00 & 22.00 &                                                                                                                                                                                                                                                                 & 146.23 &  & 43.00 & 5.00 & -0.14 \\ 
  10 & Die Olchis                                                                                                                                                                                                                                                      & Reihe & Erhard Dietl                                                                                                                                                                                                                                                    & maennlich & 47.00 & 48.00 &                                                                                                                                                                                                                                                                 & 165.97 &  & 57.00 & 8.00 & 0.01 \\ 
  11 & Der Raeuber Hotzenplotz                                                                                                                                                                                                                                         & Kleinserie & Ottfried Preuszler                                                                                                                                                                                                                                              & maennlich & 92.00 & 101.00 & Kasperl                                                                                                                                                                                                                                                         & 141.31 & 46.00 & 124.00 & 1.00 & 0.05 \\ 
  12 & Die Geggis                                                                                                                                                                                                                                                      & Buch & Mira Lobe                                                                                                                                                                                                                                                       & weiblich & 36.00 & 31.00 & Rokko und Gil                                                                                                                                                                                                                                                   & 114.90 & 29.00 & 32.00 & 2.00 & -0.07 \\ 
  13 & Der kleine Drache Kokosnuss                                                                                                                                                                                                                                     & Reihe & Ingo Siegner                                                                                                                                                                                                                                                    & maennlich & 46.00 & 52.00 & Kokosnuss                                                                                                                                                                                                                                                       & 147.30 & 61.00 & 80.00 & 2.00 & 0.06 \\ 
  14 & Hexe Lilli                                                                                                                                                                                                                                                      & Reihe & Knister Ludger Jochmann                                                                                                                                                                                                                                         & neutral & 162.00 & 53.00 & Lilli                                                                                                                                                                                                                                                           & 173.04 & 68.00 & 92.00 & 1.00 & -0.51 \\ 
  15 & Pippi Langstrumpf                                                                                                                                                                                                                                               & Kleinreihe & Astrid Lindgren                                                                                                                                                                                                                                                 & weiblich & 141.00 & 75.00 & Pippi Langstrumpf                                                                                                                                                                                                                                               & 133.87 & 17.00 & 208.00 & 2.00 & -0.31 \\ 
  16 & 5 Freunde                                                                                                                                                                                                                                                       & Reihe & Enid Blyton                                                                                                                                                                                                                                                     & weiblich & 114.00 & 118.00 & Anne, Georg, Julius, Richard                                                                                                                                                                                                                                    & 107.90 &  & 183.00 & 5.00 & 0.02 \\ 
  17 & Das Tiger- Team                                                                                                                                                                                                                                                 & Reihe & Thomas Brezina                                                                                                                                                                                                                                                  & maennlich & 49.00 & 69.00 & Biggy, Patrick, Luk                                                                                                                                                                                                                                             & 85.76 &  & 160.00 & 4.00 & 0.17 \\ 
  18 & Der kleine Eisbaer                                                                                                                                                                                                                                              & Kleinreihe & Hans de Beer                                                                                                                                                                                                                                                    & maennlich & 91.00 & 56.00 & Lars                                                                                                                                                                                                                                                            & 225.01 & 66.00 & 32.00 & 2.00 & -0.24 \\ 
  19 & Der kleine Ritter Trenk                                                                                                                                                                                                                                         & Kleinreihe & Kirsten Boie                                                                                                                                                                                                                                                    & weiblich & 42.00 & 52.00 & Trenk von Tausendschlag                                                                                                                                                                                                                                         & 148.13 & 39.00 & 280.00 & 2.00 & 0.11 \\ 
  20 & Harry Potter                                                                                                                                                                                                                                                    & Kleinserie & Joanne K. Rowling                                                                                                                                                                                                                                               & weiblich & 95.00 & 125.00 & Harry Potter                                                                                                                                                                                                                                                    & 113.93 & 52.00 & 336.00 & 3.00 & 0.14 \\ 
  21 & Gregs Tagebuch                                                                                                                                                                                                                                                  & Kleinreihe & Jeff Kinney                                                                                                                                                                                                                                                     & maennlich & 86.00 & 117.00 & Greg                                                                                                                                                                                                                                                            & 124.40 & 71.00 & 224.00 & 1.00 & 0.15 \\ 
  22 & Die Knickerbockerbande                                                                                                                                                                                                                                          & Reihe & Thomas Brezina                                                                                                                                                                                                                                                  & maennlich & 48.00 & 67.00 & Poppi, Dominik, Axel, Lilo                                                                                                                                                                                                                                      & 96.56 &  & 188.00 & 5.00 & 0.17 \\ 
  23 & Die wilden Huehner                                                                                                                                                                                                                                              & Kleinserie & Cornelia Funkenerxe                                                                                                                                                                                                                                             & weiblich & 77.00 & 25.00 & Sprotte, Melanie, Frieda, Trude                                                                                                                                                                                                                                 & 130.89 &  & 175.00 & 8.00 & -0.51 \\ 
  24 & Der Regenbogenfisch                                                                                                                                                                                                                                             & Kleinreihe & Marcus Pfister                                                                                                                                                                                                                                                  & maennlich & 122.00 & 95.00 & Regenbogenfisch                                                                                                                                                                                                                                                 & 99.63 & 45.00 & 14.00 & 1.00 & -0.12 \\ 
  25 & Das magische Baumhaus                                                                                                                                                                                                                                           & Reihe & Mary P. Osborn                                                                                                                                                                                                                                                  & weiblich & 84.00 & 105.00 &                                                                                                                                                                                                                                                                 & 144.60 & 58.00 & 89.00 & 2.00 & 0.11 \\ 
  26 & Die Wilden Fuszballkerle                                                                                                                                                                                                                                        & Serie & Joachim Masannek                                                                                                                                                                                                                                                & maennlich & 43.00 & 110.00 & Leon                                                                                                                                                                                                                                                            & 49.50 & 88.00 & 160.00 & 12.00 & 0.44 \\ 
  27 & Die drei ???                                                                                                                                                                                                                                                    & Reihe & Christoph Dittert                                                                                                                                                                                                                                               & maennlich & 93.00 & 122.00 & Justus Jonas,Peter Shaw, Bob Andrews                                                                                                                                                                                                                            & 57.47 & 40.00 & 126.00 & 1.00 & 0.13 \\ 
  28 & Die kleine Hexe                                                                                                                                                                                                                                                 & Buch & Ottfried Preuszler                                                                                                                                                                                                                                              & maennlich & 109.00 & 52.00 & kleine Hexe                                                                                                                                                                                                                                                     & 57.00 & 22.00 & 127.00 & 1.00 & -0.35 \\ 
  29 & Grueffelo                                                                                                                                                                                                                                                       & Kleinreihe & Axel Scheffler                                                                                                                                                                                                                                                  & maennlich & 58.00 & 54.00 & Maus                                                                                                                                                                                                                                                            & 125.24 & 41.00 & 24.00 & 2.00 & -0.04 \\ 
  30 & Tom Turbo                                                                                                                                                                                                                                                       & Reihe & Thomas Brezina                                                                                                                                                                                                                                                  & maennlich & 92.00 & 113.00 & Tom Turbo, Karo, Klaro                                                                                                                                                                                                                                          & 162.60 & 69.00 & 192.00 & 4.00 & 0.10 \\ 
   \hline
\end{tabular}
}
\caption{Datentabelle}
\label{merkmale}
\end{center}
\end{sidewaystable}% latex table generated in R 2.15.2 by xtable 1.7-0 package
% Fri Jan 25 13:16:08 2013
\begin{sidewaystable}[ht]
\begin{center}
\scalebox{0.6}{
\begin{tabular}{rrrrrrrrrrrrrrllllllrr}
  \hline
 & unterw & abh & konk & akt & sicher & aggr & mut & stark & rational & streng & ego & emo & unlog & phant & abent & grup & imon & quest & hfigsex & alter & gender \\ 
  \hline
1 & 1.00 & 1.00 & 1.00 & 1.50 & 1.50 & 1.00 & 1.00 & 1.00 & 1.00 & 1.00 & 1.00 & 1.00 & 2.00 & nicht vorhanden & Alltag & vorhanden & vorhanden & nicht vorhanden & maennlich & 8.00 & -0.69 \\ 
  2 & 1.00 & 1.50 & 1.00 & 1.50 & 1.00 & 1.00 & 1.00 & 1.00 & 1.50 & 1.00 & 1.00 & 1.00 & 2.00 & nicht vorhanden & Alltag & vorhanden & nicht vorhanden & nicht vorhanden & weiblich & 7.00 & -0.62 \\ 
  3 & 1.00 & 1.00 & 2.00 & 2.00 & 2.00 & 1.00 & 1.00 & 1.00 & 1.00 & 1.00 & 2.00 & 1.00 & 1.00 & vorhanden & Abenteuer & vorhanden & nicht vorhanden & nicht vorhanden & maennlich & 7.00 & -0.38 \\ 
  4 & 2.00 & 1.00 & 2.00 & 1.00 & 2.00 & 1.00 & 2.00 & 1.00 & 1.00 & 1.00 & 1.00 & 1.00 & 1.00 & vorhanden & Abenteuer & vorhanden & vorhanden & nicht vorhanden & weiblich & 7.00 & -0.38 \\ 
  5 &  & 2.00 & 1.00 &  & 2.00 & 1.00 & 2.00 &  & 1.00 & 1.00 & 1.00 & 1.00 &  & vorhanden & Abenteuer & nicht vorhanden &  & vorhanden & weiblich & 4.00 & -0.33 \\ 
  6 & 1.50 & 1.50 & 1.00 & 2.00 & 1.50 & 1.00 & 1.50 & 1.50 & 1.00 & 1.00 & 1.00 & 1.00 & 2.00 & nicht vorhanden & Alltag & vorhanden & vorhanden & nicht vorhanden & weiblich & 8.00 & -0.31 \\ 
  7 & 1.00 & 1.00 & 1.00 & 1.50 & 1.00 & 2.00 & 1.50 & 1.50 & 1.50 & 1.50 & 2.00 & 1.00 & 1.50 & vorhanden & Alltag & vorhanden & nicht vorhanden & nicht vorhanden & maennlich & 4.00 & -0.23 \\ 
  8 & 2.00 & 1.00 & 1.00 & 2.00 & 2.00 & 1.00 & 2.00 & 1.00 & 1.00 & 1.00 & 2.00 & 1.00 & 1.00 & vorhanden & Alltag & nicht vorhanden & nicht vorhanden & nicht vorhanden & neutral & 8.00 & -0.23 \\ 
  9 & 1.50 & 1.50 & 1.00 & 2.00 & 1.50 & 1.00 & 1.50 & 1.50 & 1.50 & 1.00 & 1.50 & 1.00 & 2.00 & nicht vorhanden & Alltag & nicht vorhanden & nicht vorhanden & nicht vorhanden & unbestimmbar & 6.00 & -0.15 \\ 
  10 & 1.50 & 1.50 & 1.00 & 1.50 & 1.50 & 1.00 & 2.00 & 2.00 & 1.50 & 1.00 & 1.00 & 1.50 & 1.50 & vorhanden & Alltag & nicht vorhanden & nicht vorhanden & nicht vorhanden & unbestimmbar & 6.00 & -0.15 \\ 
  11 & 2.00 & 2.00 & 1.00 & 1.00 & 2.00 & 1.00 & 2.00 & 1.00 & 2.00 & 1.00 & 1.00 & 1.00 & 2.00 & vorhanden & Abenteuer & nicht vorhanden & nicht vorhanden & vorhanden & maennlich & 6.00 & -0.08 \\ 
  12 & 2.00 & 2.00 & 1.00 & 2.00 & 2.00 & 2.00 & 2.00 & 2.00 & 1.00 & 1.00 & 1.00 & 1.00 & 1.00 & vorhanden & Abenteuer & vorhanden & vorhanden & nicht vorhanden & maennlich & 5.00 & 0.08 \\ 
  13 & 2.00 & 2.00 & 1.00 & 2.00 & 2.00 & 1.00 & 1.00 & 2.00 & 1.00 & 1.00 & 2.00 & 1.00 & 2.00 & vorhanden & Abenteuer & nicht vorhanden & nicht vorhanden & vorhanden & maennlich & 6.00 & 0.08 \\ 
  14 & 2.00 & 2.00 & 1.00 & 2.00 & 2.00 & 1.00 & 2.00 & 2.00 & 1.00 & 1.00 & 1.00 & 1.00 & 2.00 & vorhanden & Alltag & nicht vorhanden & nicht vorhanden & vorhanden & weiblich & 6.00 & 0.08 \\ 
  15 & 2.00 & 2.00 & 1.00 & 2.00 & 2.00 & 2.00 & 2.00 & 2.00 & 1.00 & 1.00 & 1.00 & 1.00 & 1.00 & vorhanden & Alltag & nicht vorhanden & nicht vorhanden & nicht vorhanden & weiblich & 8.00 & 0.08 \\ 
  16 & 1.50 & 2.00 & 1.00 & 2.00 & 2.00 & 1.50 & 1.50 & 1.50 & 2.00 & 1.00 & 1.50 & 1.00 & 2.00 & nicht vorhanden & Abenteuer & nicht vorhanden & nicht vorhanden & vorhanden & unbestimmbar & 8.00 & 0.15 \\ 
  17 & 2.00 & 2.00 & 1.00 & 2.00 & 2.00 & 1.00 & 2.00 & 1.50 & 2.00 & 1.00 & 1.00 & 1.00 & 2.00 & nicht vorhanden & Abenteuer & nicht vorhanden & nicht vorhanden & vorhanden & unbestimmbar & 8.00 & 0.15 \\ 
  18 & 2.00 & 2.00 & 1.00 & 2.00 & 2.00 & 1.00 & 2.00 & 2.00 &  & 1.00 & 1.00 & 1.00 & 2.00 & vorhanden & Abenteuer & vorhanden & nicht vorhanden & nicht vorhanden & maennlich & 3.00 & 0.17 \\ 
  19 & 2.00 & 2.00 & 1.00 & 2.00 & 2.00 & 1.00 & 2.00 & 2.00 & 2.00 & 1.00 & 1.00 & 1.00 & 2.00 & vorhanden & Abenteuer & nicht vorhanden & nicht vorhanden & vorhanden & maennlich & 6.00 & 0.23 \\ 
  20 & 2.00 & 1.00 & 2.00 & 2.00 & 2.00 & 2.00 & 1.00 & 2.00 & 2.00 & 1.00 & 1.00 & 1.00 & 2.00 & vorhanden & Abenteuer & vorhanden & nicht vorhanden & vorhanden & maennlich & 10.00 & 0.23 \\ 
  21 & 2.00 & 1.00 & 2.00 & 2.00 & 2.00 & 2.00 & 1.00 & 1.00 & 1.00 & 2.00 & 2.00 & 1.00 & 2.00 & nicht vorhanden & Alltag & nicht vorhanden & vorhanden & nicht vorhanden & maennlich & 10.00 & 0.23 \\ 
  22 & 2.00 & 2.00 & 1.00 & 2.00 & 2.00 & 1.50 & 2.00 & 2.00 & 2.00 & 1.00 & 1.00 & 1.00 & 2.00 & nicht vorhanden & Abenteuer & nicht vorhanden & nicht vorhanden & vorhanden & unbestimmbar & 9.00 & 0.31 \\ 
  23 & 2.00 & 1.50 & 1.50 & 2.00 & 1.50 & 2.00 & 1.50 & 2.00 & 1.50 & 1.50 & 1.50 & 1.00 & 2.00 & nicht vorhanden & Alltag & nicht vorhanden & nicht vorhanden & vorhanden & weiblich & 10.00 & 0.31 \\ 
  24 & 2.00 & 2.00 & 2.00 & 2.00 & 1.00 & 1.00 & 1.00 &  &  & 2.00 & 2.00 & 2.00 & 2.00 & vorhanden & Alltag & vorhanden & vorhanden & nicht vorhanden & neutral & 4.00 & 0.45 \\ 
  25 & 2.00 & 2.00 & 1.00 & 2.00 & 2.00 & 2.00 & 2.00 & 2.00 & 2.00 & 1.00 & 1.00 &  & 2.00 & vorhanden & Abenteuer & nicht vorhanden &  &  & unbestimmbar & 10.00 & 0.50 \\ 
  26 & 2.00 & 1.00 & 2.00 & 2.00 & 2.00 & 2.00 & 2.00 & 2.00 & 2.00 & 1.00 & 2.00 & 1.00 & 2.00 & nicht vorhanden & Alltag & nicht vorhanden & nicht vorhanden & nicht vorhanden & maennlich & 8.00 & 0.54 \\ 
  27 & 2.00 & 2.00 & 2.00 & 2.00 & 2.00 & 1.00 & 2.00 & 2.00 & 2.00 & 1.00 & 1.00 & 2.00 & 2.00 & nicht vorhanden & Abenteuer & nicht vorhanden & nicht vorhanden & vorhanden & maennlich & 10.00 & 0.54 \\ 
  28 & 2.00 & 2.00 & 2.00 & 2.00 & 2.00 & 2.00 & 2.00 & 2.00 & 2.00 & 1.00 & 1.00 & 1.00 & 2.00 & vorhanden & Alltag & nicht vorhanden & vorhanden & vorhanden & weiblich & 6.00 & 0.54 \\ 
  29 & 2.00 & 2.00 & 2.00 & 2.00 & 2.00 & 2.00 & 2.00 & 1.00 & 1.00 & 2.00 & 2.00 & 2.00 & 2.00 & vorhanden & Alltag & nicht vorhanden & nicht vorhanden & nicht vorhanden & neutral & 3.00 & 0.69 \\ 
  30 & 2.00 & 2.00 & 2.00 & 2.00 & 2.00 & 2.00 & 2.00 & 2.00 & 2.00 & 1.00 & 1.00 & 2.00 & 2.00 & vorhanden & Abenteuer & nicht vorhanden & nicht vorhanden & vorhanden & neutral & 7.00 & 0.69 \\ 
   \hline
\end{tabular}
}
\caption{Datentabelle (Fortsetzung)}
\label{merkmale}
\end{center}
\end{sidewaystable}





%\end{refsection}

%\chapter{Anderes Projetkt}
%\begin{refsection}
% %\nocite{latour2010}
% \blindtext
%  \printbibliography[heading=subbibliography]
%\end{refsection}
%
%\begin{refsection}
%  \blinddocument
%\end{refsection}
%\blinddocument

%\singlespacing
%\printbibliography	% Literaturliste

\end{document}% Gender





